\chapter{Conclusions}

In this work rare decays were analysed in order to look for hints of new physics using data recorded by the LHCb 
detector at centre-of-mass energies of 7 and 8 TeV and corresponding to a total integrated luminosity of 3.0~\invfb.

First, a measurement of the differential branching fraction of the rare $\Lb\to\Lz\mumu$ decay was performed together
with the first measurement of angular observables for this decay.
Evidence for the signal was found for the first time in the \qsq region below the square of the \jpsi mass in
in the $0.1 < \qsq < 2.0$~\gevgevcccc interval, where an enhanced yield is expected due to the vicinity of the photon pole. 
Due to a larger data sample and a better control of systematic, the uncertainty of the measurement in the $15 < \qsq < 20$~\gevgevcccc
interval are reduced by approximately a factor of three with respect to the previous LHCb measurements. 
The branching fraction measurements are compatible with SM predictions in the high-\qsq region, above the square
 of the \jpsi mass, and lie below the predictions in the low-\qsq region. In the angular analysis of $\Lb\to\Lz\mumu$ decays
 two forward-backward asymmetries, in the dimuon and $p\pi$ systems, were measured. The measurements
 of the $A_{\rm FB}^h$ observable are in good agreement with the SM predictions while for the $A_{\rm FB}^\ell$ observable
 measurements are consistently above the SM predictions.
 
Secondly, an analysis is set up to test flavour universality between electrons and muons exploiting rare decays.
A set of requirements is defined to select the rare and normalisation, $\Bz\to\Kstarz\ll$ and $\Bz\to\Kstarz(\jpsi\to\ll)$ modes
in both electron and muons channels, which includes the definition of a multivariate classifier. 
Efficiencies are obtained for this selection for all channels.
A study of the possible backgrounds to these channels is performed, which results in
a set of requirements to lower their yield in the selected samples and a set of PDFs to model
the remaining contributions in the invariant mass fits. Preliminary fits to the 4-body invariant mass distributions
of all channels are performed. Finally, a preliminary study of the systematic uncertainties is presented and
a set a procedure to obtain the result and check its robustness are defined. The results are currently blinded.


