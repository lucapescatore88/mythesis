\chapter{Conclusions}
\label{sec:conclusions}

In this work, two rare decays - one baryonic and one mesonic - are analysed in order to look for hints of new physics using data collected by the LHCb 
detector at centre of mass energies of 7 and 8 TeV and corresponding to a total integrated luminosity of 3.0~\invfb.

A measurement of the differential branching fraction of the rare $\Lb\to\Lz\mumu$ decay is performed together
with the first measurement of angular observables for this decay.
Evidence for the signal is found for the first time in the \qsq region between the two charmonium resonances and 
below the square of the \jpsi mass, in particular in the $[0.1,2.0]$~\gevgevcccc interval, 
where an increased yield is expected due to the photon pole. 
Thanks to a larger data sample and a better control of systematic effects, uncertainties on the measurements in the $[15,20]$~\gevgevcccc
interval are reduced by approximately a factor of three with respect to the previous LHCb measurements~\cite{LHCb-PAPER-2013-025}. 
The branching fraction measurements are compatible with SM predictions in the high-\qsq region, above the square
 of the \jpsi mass, but lie below the predictions in the low-\qsq region. Furthermore, an angular analysis of $\Lb\to\Lz\mumu$ decays
 is performed, where two forward-backward asymmetries, in the $p\pi$ system, \afbh, and in the dimuon system, \afbl, are measured. 
 The measurements of the $\afbh$ observable are in good agreement with the SM predictions, while for the $\afbl$ observable
 they are consistently above the predictions. Following the publication of these studies improved theoretical calculations 
 became available, which are reported in Appendix~\ref{app:newpredictions} and show a better agreement with the measurement. 
 Theoretical values are now compatible with the branching fraction measurements at low-\qsq and overestimate the experimental 
 values at high-\qsq. The situation regarding the angular observables is unchanged but the 
 significance of the existing discrepancies is enhanced due to the reduced uncertainties on the predicted values.
 
Secondly, an analysis to test flavour universality between electrons and muons exploiting rare decays is carried out.
Selection requirements are defined to select the rare and normalisation modes, $\Bz\to\Kstarz\ll$ 
and $\Bz\to\Kstarz(\jpsi\to\ll)$, in both electron and muons channels; this includes the definition of a 
multivariate classifier. 
A study of backgrounds is performed, which results in
a set of requirements to lower their yields in the selected samples and a set of PDFs to model
the remaining contributions in the invariant mass fits. The efficiency of the selection requirements is evaluated 
and fits to the 4-body invariant mass distributions are performed for all channels. Finally, a study of the systematic uncertainties is presented and
a procedure to calculate the result and validate its robustness is defined. The results are currently blinded,
pending completion of the review within the LHCb Collaboration; minimal changes are anticipated 
and publication is expected in the near future.


