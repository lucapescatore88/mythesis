\chapter{Systematics uncertainties on angular observables and results}

In the following sections are described the five main sources of systematic uncertainties
that are considered for the angular observables measurement and, finally, results are
reported in Sec.\ref{sec:afb_results}. Results are shown only for \qsq intervals
where the signal significance, shown in Tab.~\ref{tab:Lb_rawYield} is more than 3 standard
deviations. This includes all of the \qsq intervals above the \jpsi resonance and the lowest 
\qsq interval, where an increased yield is due to the presence of the photon pole.



\section{Non-flat angular efficiency}

The angular efficiency is non-flat as a function of $\cos\theta_\ell$ and $\cos \theta_h$.
Therefore, while integrating the full angular distribution, terms that cancel with perfect efficiency
may remain and generate a bias in the final result. In order to deal with this effect simulated events are
generated in a two-dimensional ($\cos\theta_\ell$,$\cos \theta_h$) space according to the
theoretical distribution described by Eq.~\ref{eq:2DcosThetaLandB} multiplied by the two-dimensional efficiency
histograms obtained from simulation and reported in Fig.~\ref{fig:2DcosThetaLandBeff}.
Then one-dimensional projections are taken and fit using the default 1D efficiency functions.
In Fig.~\ref{fig:effBias} deviations from the true generated value $\Delta x = x_{true} - x_{measured}$ are shown.
Since the mean of these distributions is non zero by more then 3$\sigma$, they are taken as a systematic uncertainties.

\begin{figure}[h!]
\includegraphics[width=0.9\textwidth]{Lmumu/figs/efficiencies/2D/2Deff_upper_cosThetaB_vs_cosThetaL_DD.pdf}
\includegraphics[width=0.9\textwidth]{Lmumu/figs/efficiencies/2D/2Deff_upper_cosThetaB_vs_cosThetaL_LL.pdf}
\caption{Angular acceptance as a function of $\cos\theta_\ell$ and $\cos\theta_h$ for long (left) and
downstream (right) candidates, integrated over the full available \qsq range.}
\label{fig:2DcosThetaLandBeff}
\end{figure}

\begin{figure}
\centering
\includegraphics[width=0.48\textwidth]{Lmumu/figs/fLsys_efficiency.pdf}
\includegraphics[width=0.48\textwidth]{Lmumu/figs/afbsys_efficiency.pdf}
\includegraphics[width=0.48\textwidth]{Lmumu/figs/afbBsys_efficiency.pdf}
\caption{Deviations of the observables' values obtained fitting simulated events
 generated with a 2D distribution multiplied by a 2D efficiency and fitting 1D projections
 with respect to generated values. From left to for $f_{\rm L}$ (top left), 
 $A_{\rm FB}^\ell$ (top right) and $A_{\rm FB}^h$ (bottom). }
\label{fig:effBias}
\end{figure}




\section{Resolution}

The angular resolution could bias the observables measurement 
%If the angular distribution is asymmetric, a non-zero resolution may yield to
generating an asymmetric migration of events.
This is especially important in the $\cos \theta_h$ case, because this has worse resolution
and a considerably asymmetric distribution. To study this systematic simulated experiments are used,
where events are generated following the measured distributions (including efficiencies).
The generated events are then smeared by the angular resolution (gaussian smearing).
To be conservative the case with biggest angular resolution, downstream events, is always used.
%downstream events, as reported in table \ref{tab:resolutions}.
Finally, the smeared and not-smeared distributions are fitted with the same PDF.  
The average deviation from the default values are reported in Tab.~\ref{tab:resolSys}
as a function of \qsq and assigned as systematic uncertainties.

\begin{table}[h]
\centering
\caption{Values of simulated $\cos\theta_\ell$ and $\cos\theta_\Lambda$ 
resolutions and systematic uncertainties on angular observables due to
the finite resolution in bins of \qsq.}
\begin{tabular}{c|c|c|c|c|c}
 \qsq [\gevgevcccc] &  $\sigma_\ell$    &  $\sigma_\Lambda$   & $\Delta A_{\rm FB}^\ell$ &  $\Delta f_{\rm L}$ & $\Delta A_{\rm FB}^h$ \\ \hline
0.1--2.0  & 0.0051 & 0.061 & 0.0011 & -0.0022 & -0.007 \\ 
11.0--12.5 & 0.0055 & 0.067 & 0.0016 & -0.0051 & -0.013 \\
15.0--16.0 & 0.0059 & 0.070 & 0.0006 & -0.0054 & -0.010 \\
16.0--18.0 & 0.0064 & 0.070 & 0.0014 & -0.0077 & -0.010 \\
18.0--20.0 & 0.0081 & 0.074 & 0.0014 & -0.0062 & -0.010 \\
\hline
15.0--20.0 & 0.0066 & 0.072 & 0.0013 & -0.0076 & -0.011 \\
\end{tabular}
\label{tab:resolSys}
\end{table}


\section{Efficiency description}

An imprecise determination of the reconstruction and selection efficiency can introduce extra oddity
and therefore bias the measurement. To asses this effect the kinematic re-weighting described in \ref{sec:kinWeight}
is removed from the simulation and the efficiency is determined again.
Simulated events are then fit using the same theoretical PDF and multiplied by the efficiency
function obtained with and without kinematical weights. As in the previous cases the average bias 
is taken as systematic uncertainty. Results are shown in Tab.~\ref{tab:AfbeffSys}.

\begin{table}[h]
\centering
\caption{Values systematic erros due to limited knowledge of the efficiency
function on the three angular observables in bins of \qsq}
\begin{tabular}{c|c|c|c}
 \qsq [\gevgevcccc]  & $A_{\rm FB}^h$   & $A_{\rm FB}^\ell$ & $f_{\rm L}$ \\ \hline
0.1--2.0    &  0.0093 & 0.0020  & 0.0440 \\ 
11.0--12.5  &  0.0069 & 0.0069  & 0.0027 \\
15.0--16.0  &  0.0109 & 0.0018  & 0.0046 \\
16.0--18.0  &  0.0159 & 0.0012  & 0.0043 \\
18.0--20.0  &  0.0148 & 0.0030  & 0.0017 \\
\hline
15.0-20.0  &  0.0138 & 0.0002  & 0.0046 \\
\end{tabular}
\label{tab:AfbeffSys}
\end{table}

Furthermore, for the effect of the limited simulated statistics is taken into account and found
to be negligible with respect to other sources.
%This is not already taken in account
%by the Feldman-Cousins plug-in method ( see \ref{sec:FeldmanCousins}) since the efficiency parameters are always fixed
%in the fit. We generate toy MC events using the measured values of angular observables and the default efficiency. 
%Then we vary the efficiency with its errors and we refit with the new efficiency model. In each toy we generate a number
%of events comparable to the one we have in data in bins of \qsq. The average deviation over 1000 toys is taken as systematic,
%values are reported in table \ref{tab:stateffsys}. N.B.: Efficiency parameters are always fixed when fitting the angular distributions. 
%
%Finally, the effect induced by a non-zero production polarisation in the efficiency description
%was investigated by varying the polarisation parameter in the model used to re-weight the simulation
%from $- \sigma$ to $+ \sigma$ for the central measured value, as done for the branching ratio analysis \ref{sec:BRpolsys}. 
%Then again simulated experiments are generated and fit using efficiency models obtained with different values
%of the polarisation. No significant effect is found. 

\begin{table}[h]
\centering
\caption{Values of systematic unceertainties due to the statistics of the simulated
samples on the three angular observables in bins of \qsq.}
\begin{tabular}{c|c|c|c}
 \qsq [\gevgevcccc]  & $A_{FB}^\ell$   & $f_{L}$ & $A_{FB}^h$ \\ \hline
0.1--2.0    &  0.00151 & 0.00170  & 0.00213 \\
11.0--12.5  &  0.00121 & 0.00154  & 0.00196 \\
15.0--16.0  &  0.00004 & 0.00017  & 0.00103 \\
16.0--18.0  &  0.00065 & 0.00246  & 0.00417 \\
18.0--20.0  &  0.00023 & 0.00372  & 0.00162 \\
\hline
15.0--20.0  &  0.00039 & 0.00091  & 0.00137 \\
\end{tabular}
\label{tab:stateffsys}
\end{table}


\section{Background parameterisation}
\label{sec:bkgShapeSys}

There is a certain degree of arbitrariety in the choice of a parameterisation for the background,
especially in bins with low statistics. To asses possible biases due to this simulated experiments
are generated using the shapes from real data fits and the same statistics observed in data for each \qsq bin.
Each toy is fit with two models: the default one, a ``line times efficiency" function and
the efficiency function alone, corresponding to the assumption that background distributions
are originally flat and only modified by the interaction with the detector. 
The average bias the generated experiments is taken as systematic uncertainties.
Results are reported in Tab.~\ref{tab:bkgParamSys}.

%Finally, we consider also the bias which may be cause by fixing the background fraction in the fit by fitting the same toys with
%a model where the background fraction is fixed to a random value normally distributed around the measured one. This systematic is negligible
%compared to the previous one but we add it anyway to the final systematic.
%The total result calculated as the square root sum of the 2 compoents is reported in table \ref{tab:bkgParamSys}.

\begin{center}
\begin{table}[h]
\centering
\caption{ Values of systematics due to the choice of background parameterisation in bins of \qsq. }
\begin{tabular}{c|c|c|c}
\qsq [\gevgevcccc] & $A_{FB}^\ell$     & $f_L$      & $A_{FB}^h$   \\ \hline
0.1--2.0         &  0.003	 &   0.049	  &  0.053		\\
11.0--12.5		&  0.045     &   0.034	  &  0.035     \\
15.0--16.0 	&  0.010     &   0.038    &  0.026     \\
16.0--18.0 	&  0.026     &   0.036    &  0.022     \\
18.0--20.0 	&  0.011     &   0.031    &  0.025     \\
\hline
15.0--20.0		&  0.007     &   0.014    &  0.017     \\
\end{tabular}
\label{tab:bkgParamSys}
\end{table}
\end{center}




\section{Polarisation}

To study the effect of a non-zero \Lb production polarisation simulated events are generated 
using the distributions~\ref{eq:lepton2D} and \ref{eq:hadron2D} as a function of our angular
observable and $\cos\theta$, which is sensitive to polarisation.
Similarly to the procedure used for the branching ratio measurement, events are generated using
values of the polarisation corresponding to $\pm \sigma$ from the LHCb measurement~\cite{Aaij:2013oxa}.
In the theoretical distributions $\cos \theta$ is always odd therefore with perfect efficiency it always
drops out by integrating over $\cos\theta$. Therefore the generated distributions must also contain
the information of the two-dimensional efficiency. No significant bias is found.
%In Fig.~\ref{fig:2Deffs} are reported 2D efficiencies
%as a function of $\cos \theta_\ell$ and $\cos \theta_\Lambda$ versus $\cos \theta$. and in Fig.~\ref{fig:Afbpolsys} are reported distributions of the absolute difference between the observable value obtained fitting generated events with the two polarisation values. 
%We then plot the the biggest difference between the plus or minus $\sigma$ case and the central value.
%For the integrated high \qsq region we obtain the following average deviations: $\Delta f_L = 0.0016 \pm 0.0012$, $\Delta A_{FB}^\ell = 0.00048 \pm 0.00074$ and $\Delta A_{FB}^h = 0.00014 \pm 0.00075$.
%Since all average differences are consistent with zero within less than 1.5$\sigma$ we do not assign extra systematics for polarisation.

%\begin{figure}
%\centering
%\includegraphics[width=0.45\textwidth]{Lmumu/figs/2Deff_tot_cosThetaL_vs_cosTheta_All.pdf}
%\includegraphics[width=0.45\textwidth]{Lmumu/figs/2Deff_upper_cosThetaB_vs_cosTheta_All.pdf}
%\caption{2-dimensional efficiencies obtained from weighted MC for $\cos \theta_\ell$ and $\cos\theta_\Lambda$ versus $\cos\theta$.}
%\label{fig:2Deffs}
%\end{figure}
%\begin{figure}
%\centering
%\includegraphics[width=0.45\textwidth]{Lmumu/figs/fLsys_polarisation.pdf}
%\includegraphics[width=0.45\textwidth]{Lmumu/figs/afbsys_polarisation.pdf}
%\includegraphics[width=0.45\textwidth]{Lmumu/figs/afbBsys_polarisation.pdf}
%\caption{Plots show the absolute difference of the observables' values obtained fitting toy MC generated with polarisation set to $-0.03$ and $+0.15$. From left to right for $f_L$, $A_{FB}^\ell$ and $A_{FB}^h$. }
%\label{fig:Afbpolsys}
%\end{figure}




\section{$J/\psi$ cross-check}

To cross-check the fitting procedure this is applied on the high statistics \Lb\to\jpsi\Lz sample.
To select these events the same selection as for the branching fraction is used (see Sec.~\ref{sec:Lb_selection})
with the addition of a strong PID cut on the proton (\verb!PID!$p > 10$), needed to reduce the
\KS\jpsi background. This is particularly important for the $cosTheta_h$ fit, since 
the \KS events are not distributed in a flat way in this variable would therefore can bias the fit.
In Fig.~\ref{fig:Jpsimass_angular} invariant mass plots after this cut are reported, which can be comparend with
Fig.~\ref{fig:Lb_totalFit}. %One can see that the \KS background is reduced.
After the PID cut there are 0.2\% of \KS events left in the downstream sample and a fraction
compatible with zero in the long sample.
The signal fit model is the same used for the rare case and described in \ref{sec:angfit}.
For the background instead the higher statistics allows to leave more freedom to the fit.
Therefore a second-order Chebyschev polynomial is used, where the two parameters are
free to float. As for the rare case the background fractions are gaussian-constrained
to what found in the invariant mass fit. In Figs. \ref{fig:AngFitJpsi}, \ref{fig:AngFitBJpsi} 
fitted angular distributions are reported for the \jpsi channel. The measured values of the observables
are $A_{\rm FB}^\ell = -0.002^{+0.011}_{-0.011}$, $A_{\rm FB}^h = -0.402^{+0.010}_{-0.009}$
and $f_{\rm L} = 0.485^{+0.019}_{-0.020}$, where the errors are 60\% Feldman Cousins confidence intervals.
The lepton side asymmetry as expected is measured to be zero.

\begin{figure}
\centering
\includegraphics[width=0.48\textwidth]{Lmumu/figs/Jpsi_default_LL_log_fitAndRes.pdf}
\includegraphics[width=0.48\textwidth]{Lmumu/figs/Jpsi_default_DD_log_fitAndRes.pdf}
\caption{Invariant mass distribution of \Lb\ra\jpsi\Lz long (left) and downstream (right)
candidates with an extra proton PID cut to remove \KS background. }
\label{fig:Jpsimass_angular}
\end{figure}
%
\begin{figure}[h]
\centering
\includegraphics[width=0.48\textwidth]{Lmumu/figs/AngularDistribs/Fitted/Afb_DD_jpsi.pdf}
\includegraphics[width=0.48\textwidth]{Lmumu/figs/AngularDistribs/Fitted/Afb_LL_jpsi.pdf}
\caption{Fitted angular distribution as a function of $\cos\theta_\ell$ for \Lb\to\jpsi\Lz candidates
reconstructed using downstream (left) and long (right) tracks. }
\label{fig:AngFitJpsi}
\end{figure}
%
\begin{figure}[h]
\centering
\includegraphics[width=0.48\textwidth]{Lmumu/figs/AngularDistribs/Fitted/AfbB_DD_jpsi.pdf}
\includegraphics[width=0.48\textwidth]{Lmumu/figs/AngularDistribs/Fitted/AfbB_LL_jpsi.pdf}
\caption{Fitted angular distribution as a function of $\cos\theta_h$ for \Lb\to\jpsi\Lz candidates
reconstructed using downstream (left) and long (right) tracks.  }
\label{fig:AngFitBJpsi}
\end{figure}







\section{Results}
\label{sec:afb_results}

In Figs.~\ref{fig:AngFit} and \ref{fig:AngFitB} are reported fits to angular distributions
for the 15-20 \gevgevcccc ~\qsq interval.
%The LL and DD distributions are fitted simultaneously and therefore we only have two free parameters
%in the fit, $f_L$ and $A_{FB}^\ell$, for the lepton side and one, $A_{FB}^h$, for the hadron one.
%
Tab.~\ref{tab:angresults} reports measured values of $A_{\rm FB}^\ell$, $A_{\rm FB}^h$ and $f_{\rm L}$.
, with the asymmetries shown in Fig.~\ref{fig:Afb_results}. The statistical uncertainties on these tables 
are obtained using the likelihood-ratio ordering method described in Sec.~\ref{sec:FeldmanCousins}, where only one of the two
observables at a time is treated as the parameter of interest. In Fig.~\ref{fig:contours} the statistical uncertainties
on $A_{\rm FB}^\ell$ and $f_{\rm L}$ are also reported as two-dimensional 68\;\% confidence level (CL) regions,
where the likelihood-ratio ordering method is applied by varying both observables and therefore taking
correlations into account. Total systematic uncertainties correspond to the square root sum of the
single considered sources. The SM predictions on the plots are obtained from Ref.~\cite{Detmold:2012vy}.  

\begin{figure}[h]
\centering
\includegraphics[width=0.48\textwidth]{Lmumu/figs/AngularDistribs/Fitted/Afb_DD_q2_1500_2000.pdf}
\includegraphics[width=0.48\textwidth]{Lmumu/figs/AngularDistribs/Fitted/Afb_LL_q2_1500_2000.pdf}
\caption{Fitted angular distributions as a function of $\cos\theta_\ell$ for downstream
 (left) and long (right) candidates in the 15--20 \gevgevcccc ~\qsq interval.  }
\label{fig:AngFit}
\end{figure}

\begin{figure}[h]
\centering
\includegraphics[width=0.48\textwidth]{Lmumu/figs/AngularDistribs/Fitted/AfbB_DD_q2_1500_2000.pdf}
\includegraphics[width=0.48\textwidth]{Lmumu/figs/AngularDistribs/Fitted/AfbB_LL_q2_1500_2000.pdf}
\caption{Fitted angular distributions as a function of $\cos\theta_h$ for downstream
 (left) and long (right) candidates in the 15--20 \gevgevcccc ~\qsq interval.  }
 \label{fig:AngFitB}
\end{figure}



\begin{table}[tbp]
\centering
\caption{Measured values of leptonic and hadronic angular observables,
  where the first uncertainties are statistical and the second
  systematic.}
\label{tab:angresults}
\renewcommand{\arraystretch}{1.2}
\begin{tabular}{c|ccc}
 \qsq interval  [\gevgevcccc]   &            $A_{\rm FB}^\ell$      &       $f_{\rm L}$ 						&  $A_{\rm FB}^h$                    \\ \hline

0.1 -- 2.0   & $\phantom{-\,}0.37 \; ^{+\;0.37}_{-\;0.48} \,\pm\, 0.03$  	&   $0.56 \; ^{+\;0.23}_{-\;0.56}\,\pm\, 0.08$ 		& $-\;0.12 \; ^{+\;0.31}_{-\;0.28}\,\pm\, 0.15$	\\
11.0 -- 12.5 & $\phantom{-\,}0.01 \; ^{+\;0.19}_{-\;0.18} \,\pm\, 0.06$  	&   $0.40 \; ^{+\;0.37}_{-\;0.36}\,\pm\, 0.06$		& $-\;0.50 \; ^{+\;0.10}_{-\;0.00}\,\pm\, 0.04$	 \\
15.0 -- 16.0 & $-\,0.10 \; ^{+\;0.18}_{-\;0.16} \,\pm\, 0.03$  			&   $0.49 \; ^{+\;0.30}_{-\;0.30} \,\pm\, 0.05$ 	& $-\;0.19 \; ^{+\;0.14}_{-\;0.16}\,\pm\, 0.03$	\\	
16.0 -- 18.0 & $-\,0.07 \; ^{+\;0.13}_{-\;0.12} \,\pm\, 0.04$  			&   $0.68 \; ^{+\;0.15}_{-\;0.21} \,\pm\, 0.05$ 	& $-\;0.44 \; ^{+\;0.10}_{-\;0.05}\,\pm\, 0.03$	\\
18.0 -- 20.0 & $\phantom{-\,}0.01 \; ^{+\;0.15}_{-\;0.14} \,\pm\; 0.04$  	&   $0.62 \; ^{+\;0.24}_{-\;0.27}\,\pm\, 0.04$ 		& $-\;0.13 \; ^{+\;0.09}_{-\;0.12}\,\pm\, 0.03$	\\ \hline
15.0 -- 20.0 & $-\,0.05 \; ^{+\;0.09}_{-\;0.09} \,\pm\, 0.03$  			&   $0.61 \; ^{+\;0.11}_{-\;0.14} \,\pm\, 0.03$ 	& $-\;0.29 \; ^{+\;0.07}_{-\;0.07}\,\pm\, 0.03$	\\
\end{tabular}
\end{table}

\begin{figure}[ptb]
\centering
\includegraphics[width=0.48\textwidth]{Lmumu/figs/paper/figure8a.pdf}
\includegraphics[width=0.48\textwidth]{Lmumu/figs/paper/figure8b.pdf}
\caption{Measured values of (left) the leptonic and (right) the hadronic
  forward-backward asymmetries in bins of \qsq.
  Data points are only shown for \qsq intervals where a statistically
  significant signal yield is found, see text for details.
  The (red) triangle represents the values for the $15 < \qsq < 20$ \gevgevcccc
  interval. Standard Model predictions are obtained from Ref.~\cite{Meinel:2014wua}.}
\label{fig:Afb_results}
\end{figure}



\begin{figure}[h]
\centering
\includegraphics[width=0.48\textwidth]{Lmumu/figs/paper/figure9.pdf}
\includegraphics[width=0.48\textwidth]{Lmumu/figs/paper/figure10a.pdf}
\includegraphics[width=0.48\textwidth]{Lmumu/figs/paper/figure10b.pdf}
\includegraphics[width=0.48\textwidth]{Lmumu/figs/paper/figure10c.pdf}
\includegraphics[width=0.48\textwidth]{Lmumu/figs/paper/figure10d.pdf}
\includegraphics[width=0.48\textwidth]{Lmumu/figs/paper/figure10e.pdf}
\caption{Two-dimensional 68\,\% CL regions (black) as a
  function of $A_{\rm FB}^\ell$ and $f_{\rm L}$.  The shaded areas
  represent the regions in which the PDF is positive over the complete $\cos
  \theta_{\ell}$ range. The best fit points are indicated by the (blue) stars. }
\label{fig:contours}
\end{figure}
