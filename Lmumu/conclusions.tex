\section{Conclusions}
  A measurement of the differential branching fraction of the
  \decay{\Lb}{\Lz\mumu} decay is performed using data, corresponding
  to an integrated luminosity of 3.0\invfb, recorded by the \lhcb
  detector at centre-of-mass energies of 7 and 8\tev. Signal is
  observed for the first time at a significance of more than three
  standard deviations in two \qsq intervals: $0.1 < \qsq < 2.0$
  \gevgevcccc, close to the photon pole, and between the charmonium
  resonances. No significant signal is observed in the $1.1 < \qsq <
  6.0$ \gevgevcccc range. The uncertainties of the measurements in the
  region $15 < \qsq < 20$ \gevgevcccc are reduced by a factor of
  approximately three relative to previous \lhcb
  measurements~\cite{LHCB-PAPER-2013-025}.  The improvements in the
  results, which supersede those of Ref.~\cite{LHCB-PAPER-2013-025},
  are due to the larger data sample size and a better control of
  systematic uncertainties.  The measurements are compatible with the
  predictions of the Standard Model in the high-\qsq region and lie
  below the~predictions~in~the~low-\qsq~region.

  The first measurement of angular observables for the
  \decay{\Lb}{\Lz\mumu} decay is reported, in the form of two
  forward-backward asymmetries, in the dimuon and $p\pi$ systems and
  the fraction of longitudinally polarised dimuons.  The measurements
  of the $A_{\rm FB}^h$ observable are in good agreement with the
  predictions of the SM, while for the $A_{\rm FB}^\ell$ observable
  measurements are consistently above the prediction.


