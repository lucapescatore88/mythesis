\section{Conclusions}
  
  A measurement of the differential branching fraction of the
  \decay{\Lb}{\Lz\mumu} decay is performed using data, corresponding
  to an integrated luminosity of 3.0~\invfb, recorded by the \lhcb
  detector at centre-of-mass energies of 7 and 8~\tev. The signal, previously
  observed only above the square of the \psitwos mass, is now
  observed at a significance of more than three standard deviations in two \qsq intervals: $0.1 < \qsq < 2.0$
  \gevgevcccc, where an increased yield if expected due to the photon pole, and between the charmonium
  resonances. No signal is observed in the $1.1 < \qsq < 6.0$ \gevgevcccc range, which is the theoretically 
  cleaner interval to look for new physics effects. The uncertainties on the measurements in the
  region $15 < \qsq < 20$ \gevgevcccc are reduced by a factor of
  approximately three relative to previous \lhcb
  measurements~\cite{LHCB-PAPER-2013-025}.  These improvements 
  %in the results %, which supersede those of Ref.~\cite{LHCB-PAPER-2013-025},
  are due to the larger data sample size and a better control of
  systematic uncertainties.  The measurements are compatible with the
  predictions of the Standard Model in the high-\qsq region and lie
  below the predictions in the low-\qsq region.

  Furthermore, the first measurement of angular observables for the
  \decay{\Lb}{\Lz\mumu} decay is reported, in the form of two
  forward-backward asymmetries, in the dimuon and $p\pi$ systems and
  the fraction of longitudinally polarised dimuons.  The measurements
  of the \afbh observable are in good agreement with the
  predictions of the SM, while for the \afbl observable
  measurements are consistently above the prediction.
  
  The publication of the results included in this thesis triggered intrest in the
  theory community, which produced improved lattice calculations and predictions~\cite{Detmold:2016pkz}.
  Results with the new predictions overlaid are reported in Appendix~\ref{app:newLbpredictions}.
  The latest predictions for the \fl and \afbh observables continue to agree with the
  measurements while a 3.3$\sigma$ local tension is found for \afbl at high \qsq.
  
  
  


