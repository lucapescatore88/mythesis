\section{Yield extraction}

Extended unbinned maximum likelihood fits are used to extract the yields of the rare and resonant channels.
The likelihood has the form:
%
\begin{equation}
\mathcal{L}=e^{-(N_\mathrm{S}+N_\mathrm{C}+N_{\mathrm{B}})}\times\prod_{i=1}^{N}\left[
N_\mathrm{S}P_{\mathrm{S}}(m_i)+N_\mathrm{C}P_\mathrm{C}(m_i)+N_{\mathrm{B}}P_{\mathrm{B}}(m_i)\right]
\end{equation}
\noindent
where $N_\mathrm{S}$, $N_\mathrm{C}$ and $N_\mathrm{B}$ are respectively the numbers of signal, 
combinatorial and \KS background candidates and the $P_i(m_i)$ are the corresponding probability density functions (PDF).
The fit variable is the 4-body $m(p\pi\mu\mu)$ invariant mass obtained from
a kinematical fit of the full decay chain in which each particle is constrained to point to its
assigned origin vertex and the invariant mass of the $p\pi$ system is constrained to be equal to
the world average of the \Lz baryon mass~\cite{PDG2014}. In the resonant case a further constraint is imposed, namely 
that the dimuon mass is equal to the known \jpsi mass. This method allows the mass resolution to be improved giving
better defined peaks and therefore a more stable fit. For brevity, in the following these variables are
simply referred to as ``invariant mass".

\subsection{Fit description}
\label{sec:Lb_fit}

The fit is performed via the following steps:
\begin{itemize}
\item simulated distributions are fitted to extract initial parameters values;
\item the resonant data sample is fitted;
\item the rare sample is fitted with the values of some parameters fixed to those obtained in the previous cases.
\end{itemize}
In the first step, simulated $\Lb\to\jpsi\Lz$ distributions are fitted using the signal PDF alone.
This is done separately for downstream and long candidates. Figure~\ref{fig:Lb_jpsiMCfit} shows 
distributions of candidates selected in the resonant sample with the fit function overlaid.
%
The signal is described as the sum of two Crystal Ball functions (CB) with
common mean ($m_0$) and tail slope ($n$). This is also known as a Double Crystal Ball (DCB) function.
A single Crystal Ball~\cite{Skwarnicki:1986xj} is a probability
density function commonly used to model processes involving energy loss. In particular it is used
to describe resonances that have radiative tails. This function 
combines a Gaussian core with a power-law tail of slope $n$ that takes effect beyond some 
threshold $\alpha$ away from the peak value. This asymmetric function has the form
%
\begin{equation}
C(x;\alpha,n,\bar{x},\sigma) = N \cdot
\begin{cases}
\exp \left( -\frac{(x - \bar{x})^2}{2\sigma} \right)  & \mbox{   if   } \frac{(x - \bar{x})}{\sigma} > \alpha, \\
A\left( B - \frac{(x - \bar{x})}{\sigma} \right)^{-n} & \mbox{   if   } \frac{(x - \bar{x})}{\sigma} < \alpha,
\end{cases}
\end{equation}
%
where for normalisation and continuity
%
\begin{equation}
\label{CB}
\begin{array}{lcr}
A = \left( \frac{c}{|\alpha|} \right))^n \cdot \exp(- \frac{\alpha^2}{2}) & \text{ and } & B = \frac{n}{|\alpha|} - |\alpha|.
\end{array}
\end{equation}
%
The full PDF for the resonant signal channel,  $P_\mathrm{S}(m)$, is therefore:
%
\begin{equation}
P_\mathrm{S}(m;m_0,\alpha_1,\alpha_2,f,n) = f \cdot \text{CB}(m;m_0,\sigma_1,\alpha_1,n)+(1-f) \cdot \text{CB}(m;m_0,\sigma_2,\alpha_2,n), \nonumber
\end{equation}
%
where $f$ is the relative fraction of candidates falling into the first CB function.

\begin{figure}
\centering
\includegraphics[width=0.49\textwidth]{Lmumu/figs/MassFits/fitLb2JpsiL_DD_MC.pdf}
\includegraphics[width=0.49\textwidth]{Lmumu/figs/MassFits/fitLb2JpsiL_LL_MC.pdf}
\caption{Invariant mass distribution of $\Lb\ra\jpsi\Lz$ downstream (left) and long (right) candidates.
The points show simulated data and the blue line is the signal fit function.}
\label{fig:Lb_jpsiMCfit}
\end{figure}

In a second step, the fit to the resonant channel data sample is performed.
For this fit the tail slope parameter, $n$, which is highly correlated
with $\alpha_1$ and $\alpha_2$, is fixed to the value found in the fit to simulated candidates.
In this fit two background components are modelled: the combinatorial background,
parameterised with an exponential and the background from $\Bz\ra\jpsi\KS$ decays.
The shape used to describe the \KS background is obtained from a $\Bz\ra\jpsi\KS$ simulated
sample that has satisfied the full selection. The invariant distribution of these candidates
is fitted with a DCB function, which is then used to model the \KS background
in the $\Lb\to\jpsi\Lz$ fit. The fit to the simulated $\Bz\ra\jpsi\KS$ events
is reported in Fig.~\ref{fig:KSbkgFit}. When the \KS shape is introduced in the fit to the data, all
of its parameters are fixed. This is particularly important when fitting long candidates, because the 
contribution from the \KS peak is smaller and therefore the values of the parameters would 
not be adequately constrained by data. 
To take into account possible differences between simulation and data in the definition of the
absolute mass scale, an offset is introduced by adding a shift to the central value 
of the DCB, $m_0 \ra m_0 + m'$, where $m'$ is left free to vary in the fit.
In summary, the free parameters in the fit to the resonant $\Lb\to\jpsi\Lz$ sample
are the yields of the signal and the combinatorial and \KS backgrounds, the slope
of the exponential and the horizontal shift of the \KS shape. Note that all the parameters
of the PDFs used to fit the long and downstream samples are independent.

\begin{figure}
\centering
\includegraphics[width=0.6\textwidth]{Lmumu/figs/MassFits/fitKS_bkg.pdf}
\caption{Invariant mass distribution of simulated $\Bz\ra\jpsi\KS$ events passing the
full $\Lz\mu\mu$ selection with the fit function, a Double Crystal Ball, overlaid. }
\label{fig:KSbkgFit}
\end{figure}

Finally, the fit to the rare $\Lb\to\Lz\mumu$ data sample is carried out. In this case the fits to the long
and downstream samples are performed simultaneously to obtain a more stable convergence. 
For this fit the signal is modelled with the same shape used in the resonant case as there is no physical
reason why they should be different. This method is also useful to reduce systematic uncertainties
as the result will be given as a ratio between rare and resonant quantities.
However, the small candidate yields expected in the rare samples do not allow many 
parameters to be reliably extracted from the fits.
%,especially when dividing data in \qsq bins.
Therefore, all parameters of the signal shape are fixed to
the ones derived from the fit to the $\jpsi\Lz$ channel. However, to account for possible differences, 
arising from a different resolution in the various \qsq regions, a scale factor is applied
to the widths of the two Gaussian cores of the signal DCB: $\sigma_1 \rightarrow c(\qsq)\cdot \sigma_1$
and $\sigma_2 \rightarrow c(\qsq)\cdot \sigma_2$, where the same scale factor, $c$, is applied to both widths
but it is allowed to vary for each \qsq region. 
These factors are fixed to values obtained by fitting rare $\Lb\to\Lz\mumu$ simulated events in each \qsq bin and comparing
the widths with those obtained from the fit to the resonant simulated sample, namely
\begin{equation}
c = \sigma_{\mumu}^{MC} / \sigma_{\jpsi}^{MC}.
\end{equation}
These values are found to be $\sim 1.9$ for downstream candidates and $\sim 2.3$ for long candidates,
corresponding to the fact that in the resonant case a further constraint on the dimuon mass
is used, which improves the resolution by a factor of $\sim2$. The dependence of the scaling factor on \qsq 
is found to be small. For the fits to the long and downstream samples the parameters are always separately fixed to the
corresponding $\jpsi\Lz$ fits; in this analysis shape parameters are never shared between the two candidate categories.

The modelled background components are, also in the rare case,
the combinatorial background, described with an exponential function, and the \KS background. 
The slope of the background is visibly different depending on the \qsq interval. This is partly due to the 
fact that at high \qsq the combinatorial background changes slope because of a kinematical limit at low 4-body
masses imposed by the \qsq requirements. The exponential slopes are therefore left as independent
parameters in each \qsq interval. % and for the downstream and long samples.
The background component from $\Bz\to\KS\mumu$ decays is modelled using the same shapes used
for the resonant channel. However, in this case the mass offset, $m'$, is fixed to that found
for the resonant channel. The expected level of misreconstructed $\Bz\ra\KS\mumu$
candidates is small and does not allow its yield to be determined reliably. Therefore, 
this is fixed to the yield of $\Bz\ra\jpsi\KS$ decays rescaled by the expected ratio
of branching fractions between the resonant and rare channels. The \qsq distribution of $\Bz\ra\KS\mumu$ 
simulated events is used to predict the yield as a function of \qsq. Table~\ref{tab:KSprediction} reports the 
number of predicted $\Bz\ra\KS\mumu$ candidates in each \qsq interval obtained with the following formula:
\begin{equation}
N_{\KS\mumu}(\qsq) = N_{\jpsi\KS}\frac{B(\Bz\ra\KS\mumu)}{B(\Bz\ra\KS\jpsi)}\cdot \frac{1}{\varepsilon_{rel}} \cdot B(\jpsi\ra\mumu) \frac{N(\qsq)_{MC}}{N^{tot}_{MC}} 
\end{equation}
where $N(\qsq)_{MC}$ is the number of simulated rare candidates falling in a \qsq interval after full selection and $N^{tot}_{MC}$ 
is the total number of simulated events and \mbox{$\varepsilon_{rel} = \varepsilon_{\mu\mu} / \varepsilon_{\jpsi}$} is the relative selection efficiency between the two channels. 
%The \KS\mumu contribution is then completely taken out to study systematic
%uncertainties as described in Sec.~\ref{sec:Lb_sys}.
%Only for the 6-8 \gevgevcccc bin, a background component coming from the residual of the \jpsi radiative tail is
%added, modelled using a the shape obtained studying simulated events and smoothed using the \verb!RooKeysPdf! method of \verb!RooFit!.
%This is then removed from the final fit becuase it returns zero yield.

\begin{table}
\centering
\caption{Predicted numbers of $\Bz\ra\KS\mumu$ events in each considered \qsq interval.}
\begin{tabular}{$c^c^c}
	\rowstyle{\bfseries}
\boldmath{\qsq} [\boldmath{\gevgevcccc}]   & Downstream & Long \\ \hline
\phantom{0}0.1 -- 2.0\phantom{0} & 0.9 & 0.1 \\
\phantom{0}2.0 -- 4.0\phantom{0} & 0.9 & 0.1 \\
\phantom{0}4.0 -- 6.0\phantom{0} & 0.8 & 0.1 \\
\phantom{0}6.0 -- 8.0\phantom{0} & 1.1 & 0.1 \\
11.0 -- 12.5 & 1.9 & 0.2 \\
15.0 -- 16.0 & 1.1 & 0.1 \\
16.0 -- 18.0 & 2.0 & 0.2 \\
18.0 -- 20.0 & 1.1 & 0.1 \\ \hline
1.1 -- 6.0 & 2.1 & 0.1 \\
15.0 -- 20.0 & 4.2 & 0.5 \\ 
\end{tabular}
\label{tab:KSprediction}
\end{table}

As the fit to the rare sample is performed simultaneously on long and downstream candidates,
their two yields are not free to vary separately but are parameterised
as a function of the common branching fraction using the following formula:
%
\begin{equation}
N(\Lz\mumu)_{k}  = \left[ \frac{\mathrm{d}\mathcal{B}(\Lz\mumu)/\mathrm{d}\qsq}{\mathcal{B}(\jpsi\Lz)} \right]  \cdot
N(\jpsi\Lz)_{k} \cdot \varepsilon^{\mathrm{rel}}_{k} \cdot \frac {\Delta\qsq} { \mathcal{B}(\jpsi\to\mumu) },
\label{eq:relYield}
\end{equation}
%
where $k = $(LL,DD), $\Delta\qsq$ is the width of the \qsq interval and the only free parameter is the ratio of the branching 
fraction of the rare decay to that of the \jpsi channel, $\mathcal{B}^{\rm rel}$. The value of the branching fraction 
of the $\jpsi\to\mumu$ decay is taken to be \mbox{$(5.93 \pm 0.06)\cdot 10^{-2}$~\cite{PDG2014}} and $\varepsilon^{\rm rel}$ 
corresponds to the relative efficiencies of the rare and resonant channels obtained in Sec.~\ref{sec:Lb_eff}. 
In this formula the efficiencies and the normalisation yield appear as constants, namely 
$N(\Lz\mumu)_{k} = C_k \cdot \mathcal{B}^{\rm rel}$. 
%These constants are then varied in order to obtain 
%systematic uncertainties on the final result as described in Sec.~\ref{sec:Lb_sys}.


\subsection{Fit results}



\begin{figure}
\centering
\includegraphics[width=0.75\textwidth]{Lmumu/figs/MassFits/Lb2JpsiL__DD_data.pdf}
\includegraphics[width=0.75\textwidth]{Lmumu/figs/MassFits/Lb2JpsiL__LL_data.pdf}
\includegraphics[width=0.49\textwidth]{Lmumu/figs/MassFits/Lb2JpsiL_DD_data_log_fitAndRes.pdf}
\includegraphics[width=0.49\textwidth]{Lmumu/figs/MassFits/Lb2JpsiL_LL_data_log_fitAndRes.pdf}
\caption{Invariant mass distributions of $\Lb\ra\jpsi\Lz$ downstream (top) and long (middle) candidates
selected with high \qsq requirements.
Bottom plots are the same as the upper ones but shown in logarithmic scale. Black points show data.
The blue solid line represents the total fit function, the black dashed line the signal, the red dashed line
the combinatorial background and the green dashed line the $\Bz\ra\KS\mumu$ background.}
\label{fig:Lb_totalFit}
\end{figure}
%
Figures~\ref{fig:Lb_totalFit} and~\ref{fig:Lb_totalFit_low} show fitted invariant mass distributions for
the normalisation channel, selected with the high-\qsq and low-\qsq requirements respectively.
%The $\chi^2$ value of the fit is $126$ for LL and $112$ for DD both with 140 degrees of freedom, which corresponds to probability of 80\% and 95\%.
Table~\ref{tab:Lb_rawYieldJpsi} reports the measured yields of $\Lb\ra\jpsi\Lz$ candidates found using the low- 
and \mbox{high-\qsq} selections. Values for the signal shape parameters are given in Fig.~\ref{fig:Lb_totalFit}.
Fits to the rare $\Lb\ra\Lz\mumu$ samples are shown in Fig.~\ref{fig:Lb_Lmumu} for the integrated
$15 < \qsq < 20$ and $1.1 < \qsq < 6.0$~\gevgevcccc ~\qsq intervals, while
%The exponential slopes, the scale factors multiplied to the widths and the number of combinatorial events
%found from these fits are reported in Tab.~\ref{tab:Lb_rareParam}.
fitted invariant mass distributions for each individual \qsq interval considered are given in Figs.~\ref{fig:Lb_differentialFitDD}
and~\ref{fig:Lb_differentialFitLL} for downstream and long candidates respectively.
The yields of rare candidates obtained from the fit are listed in Tab.~\ref{tab:Lb_rawYield} together with their significances.
Most candidates are found in the downstream sample, which comprises \mbox{$\sim80\%$} of the total yield.
Note that, since the fit is simultaneous to the two candidate categories, their yields 
%are not parameters free to vary independently but 
are correlated via the branching ratio.
The statistical significance of the observed signal yields is evaluated as the change in the logarithm 
of the likelihood function, $\sqrt{2\Delta\ln{\mathcal{L}}}$, when the signal component
is excluded from the fit, relative to the nominal fit in which it is present.

\begin{table}
\centering
\caption{Number of \decay{\Lb}{\jpsi\Lz} candidates in the downstream and
  long categories found using the for low- and
  high-\qsq requirements; uncertainties are statistical only.}
\begin{tabular}{$l^c^c}
\rowstyle{\bfseries}
Selection & Long & Downstream					\\ \hline
high-\qsq	& $4313 \pm 70$	 	&  $11\,497 \pm 123$ \\
low-\qsq	& $3363 \pm 59$ 	&  $\phantom{0}\,7225 \pm 89\phantom{0}$  \\
 \hline
\end{tabular}
\label{tab:Lb_rawYieldJpsi}
\end{table}

\begin{table}
\centering
\caption{Signal yields, $N_\mathrm{S}$, obtained from the
  invariant mass fit to \decay{\Lb}{\Lz\mumu} candidates in each \qsq interval
  together with their statistical significances. 
  The \mbox{8 -- 11} and \mbox{12.5 -- 15.0}~\gevgevcccc ~\qsq intervals are excluded
  from the study as they are dominated by resonant decays via charmonium resonances.}
\begin{tabular}{$l^c^c^c^c}
\rowstyle{\bfseries} 
\boldmath{\qsq} [\boldmath{\gevgevcccc}]  & DD & LL & Tot. yield & Significance \\ \hline
\phantom{x}0.1 -- 2.0\phantom{x}   &  $\phantom{x}6.9 \pm 2.2$  &  $\phantom{xx}9.1 \pm 3.0\phantom{x}$	 &  $16.0\pm5.3$            		&  4.4 \\
\phantom{x}2.0 -- 4.0\phantom{x}   &  $\phantom{x}1.8 \pm 1.7$  &  $\phantom{xx}3.0 \pm 2.8\phantom{x}$ 	 &  $\phantom{x}4.8\pm4.7$  	 &  1.2 \\
\phantom{x}4.0 -- 6.0\phantom{x}   &  $\phantom{x}0.4 \pm 0.9$  &  $\phantom{xx}0.6 \pm 1.4\phantom{x}$	 &  $\phantom{x}0.9\pm2.3$  	 &  0.5 \\
\phantom{x}6.0 -- 8.0\phantom{x}   &  $\phantom{x}4.3 \pm 2.0$   &  $\phantom{xx}7.2 \pm 3.3\phantom{x}$	 &  $11.4\pm5.3$            		&  2.7 \\
11.0 -- 12.5  				     &  $14.6 \pm 2.9$  		    &  $\phantom{x}42.8 \pm 8.5\phantom{x}$  &  $\phantom{x.}60\pm12\phantom{.}$ &  6.5 \\
15.0 -- 16.0  			             &  $13.5 \pm 2.2$  		    &  $\phantom{x}43.5 \pm 7.2\phantom{x}$   &  $\phantom{x.}57\pm9\phantom{x.}$             			 &  8.7 \\
16.0 -- 18.0  				    &  $28.6 \pm 3.3$  		    &  $\phantom{x}88.8 \pm 10.1$	 	       &  $\phantom{.}118\pm13\phantom{.}$              			 &  13  \\
18.0 -- 20.0  				    &  $22.4 \pm 2.6$  		    &  $\phantom{x}78.0 \pm 8.9\phantom{x}$ &  $\phantom{.}100\pm11\phantom{.}$    &  14  \\
\hline
\phantom{x}1.1 -- 6.0\phantom{x}    	&  $\phantom{x}3.6 \pm 2.4$  &  $\phantom{xx}5.7 \pm 3.8\phantom{x}$	 &  $\phantom{x}9.4\pm6.3$  			&  1.7 \\
15.0 -- 20.0  					&  $64.6 \pm 4.7$  			&  $209.6 \pm 15.3$ 					&  $\phantom{.}276\pm20\phantom{.}$              		&  21  \\
\end{tabular}
\label{tab:Lb_rawYield}
\end{table}

\begin{figure}
\centering
\includegraphics[width=0.49\textwidth]{Lmumu/figs/MassFits/Lb2JpsiL__lowSel_DD_data.pdf}
\includegraphics[width=0.49\textwidth]{Lmumu/figs/MassFits/Lb2JpsiL__lowSel_LL_data.pdf}
\caption{Invariant mass distribution of $\Lb\ra\jpsi\Lz$ for downstream (left) and long (right) candidates
 selected with low-\qsq requirements.}
\label{fig:Lb_totalFit_low}
\end{figure}
%
%
\begin{figure}
\centering
\includegraphics[width=0.7\textwidth]{Lmumu/figs/paper/figure13.pdf}
\includegraphics[width=0.7\textwidth]{Lmumu/figs/paper/figure2.pdf}
\caption{Invariant mass distributions of $\Lb\ra\Lz\mumu$ candidates in the integrated $0.1 - 6.0$ (top)
and $15 - 20$~\gevgevcccc (bottom) ~\qsq intervals. Points show data combining long and downstream candidates together.
The blue solid line represents the total fit function and the dashed red line the combinatorial background.}
%\includegraphics[width=0.54\textwidth]{Lmumu/figs/MassFits/Lb2Lmumu_DD_lowQ2_fitAndRes.pdf}
%\includegraphics[width=0.54\textwidth]{Lmumu/figs/MassFits/Lb2Lmumu_LL_lowQ2_fitAndRes.pdf}
%\includegraphics[width=0.54\textwidth]{Lmumu/figs/MassFits/Lb2Lmumu_DD_highQ2_fitAndRes.pdf}
%\includegraphics[width=0.54\textwidth]{Lmumu/figs/MassFits/Lb2Lmumu_LL_highQ2_fitAndRes.pdf}
%\caption{Invariant mass distribution of $\Lb\ra\Lz\mumu$ candidates in the integrated 0.1--6.0 (top)
%\gevgevcccc ~\qsq interval for downstream (left) and long (right) candidates. The points show data, the blue line
%and 15.0--20.0 (bottom) \gevgevcccc \qsq intervals.
%The blue solid line represents the total fit function and the dashed red line the combinatorial background.}
\label{fig:Lb_Lmumu}
\end{figure}
%
\begin{figure}
\centering
\includegraphics[width=1.\textwidth]{Lmumu/figs/MassFits/q2_fits_DD_plot2.pdf}
\includegraphics[width=1.\textwidth]{Lmumu/figs/MassFits/q2_fits_DD_plot1.pdf}
\caption{Invariant mass distributions of rare $\Lb\ra\Lz\mumu$ downstream candidates in the considered \qsq intervals.
 %[0.1,2], [2,4], [4,6], [6,8], [11,12.5], [15,16], [16,18], [18,20] 
 }
\label{fig:Lb_differentialFitDD}
\end{figure}

\begin{figure}
\centering
\includegraphics[width=1.\textwidth]{Lmumu/figs/MassFits/q2_fits_LL_plot2.pdf}
\includegraphics[width=1.\textwidth]{Lmumu/figs/MassFits/q2_fits_LL_plot1.pdf}
\caption{Invariant mass distributions of rare $\Lb\ra\Lz\mumu$ long candidates in the considered \qsq intervals.
 %[0.1,2], [2,4], [4,6], [6,8], [11,12.5], [15,16], [16,18], [18,20] 
 }
\label{fig:Lb_differentialFitLL}
\end{figure}



%\begin{table}
%\centering
%\caption{Values of exponential slope, $b$, and scale factors, $c$ found from the fit
%to the resonant data sample and fixed from the fit to the rare data sample.}
%\begin{tabular}{|c|c|c|}
%\hline
%Parameter  			 & Downstream & Long	\\ 
%\hline
%\multicolumn{3}{|c|}{ 15.0--20.0 \gevgevcccc}  \\
%\hline

%$b$ 				& $0.0006 \pm 0.0003$		 		& 	$0.0012 \pm 0.0008$        \\
%$c$ 				& $1.9027 \pm 0.0001$ 				&	$2.2910 \pm 0.0001$        \\
%%$N_{exp}$ 			& $393^{+23}_{-22}$	&	 $64^{+9}_{-8}$       \\

%\hline
%\multicolumn{3}{|c|}{ 1.1--6.0 \gevgevcccc}  \\

%\hline
%$b$ 				& $-0.0026 \pm 0.0004$				&	$-0.0036 \pm 0.0011$	 \\
%$c$ 				& $1.9208 \pm 0.0001$				&	$2.3504 \pm 0.0001$        \\
%%$N_{exp}$ 			& $203^{+15}_{-14}$	&	$34^{+7}_{-6}$       \\
%\hline

%\end{tabular}
%\label{tab:Lb_rareParam}
%\end{table}




\clearpage

