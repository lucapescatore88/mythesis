\chapter{Efficiency}
\label{sec:Lb_eff}

The efficiency for each of the decays is calculated according to the formula
\begin{equation}
\varepsilon=\varepsilon(Geom)\varepsilon(Det|Geom)\varepsilon(Reco|Det)\epsilon(MVA|Reco)\varepsilon(Trig|MVA).
\end{equation}
In this expression the first term gives the efficiency to have final state particles in the LHCb acceptance.
The second term handles the possibility of \Lz escaping the detector or interacting with it and therefore
never decaying into $p\pi$. This term is referred to as ``Detection" efficiency.
The third term carries information about the reconstruction and stripping efficiency
which keep these together given that boundaries between them are completely artificial.
The fourth part deals with the efficiency of the Neural Network for those events which passed the pre--selection . 
Finally, the last term handles the trigger efficiency.
%Trigger efficiency is calculated before the MVA one because the training is performed on untriggered data to save statistics.
Most of the efficiency components are evaluated using the simulated events described in Sec.~\ref{sec:Lb_simulation}.
Only the efficiency of cut on the PID of the proton, present in the stripping, is separately derived
with a data--driven method because the simulation does not provide a good description of PID variables.
For complete information, all absolute efficiencies for the two decays \Lb\to\Lz\mumu and \Lb\to\jpsi\Lz are
separately listed in the next sections. However, for the analysis itself only relative efficiency,
$\epsilon(\Lb\to\Lz\mumu)/\epsilon(\Lb\to\jpsi\Lz)$, is used. 


\section{Geometric acceptance}
\label{sec:Lb_geomAcc}
In order to save disk space and time, simulated samples contain only events in which the final muons
are in LHCb acceptance, and therefore can be reconstructed. This corresponds to requirement for each
of the muons to be in an interval $10 < \theta < 400$ mrad, where $\theta$ is the angle between
the muon momentum and the beam line. The efficiency of this requirement is obtained by using 
a separate simulated sample where events are generated in the full space.
In Tab.~\ref{tab:Lb_geom_eff} the efficiencies due to the geometrical acceptance are listed
in bins of \qsq for \Lb\to\Lz\mumu decays.
%
\begin{table}
\centering
\caption{Absolute geometrical acceptance in bins of \qsq from MC simulation. Errors shown are statistical only.}
\begin{tabular}{lc}\hline
\qsq [\gevgevcccc]     & Geom. acc.   \\ \hline
0.1--2.0 	&  $0.2359 \pm 0.0008$  \\
2.0--4.0 	&  $0.2098 \pm 0.0007$  \\
4.0--6.0 	&  $0.2008 \pm 0.0007$  \\
6.0--8.0 	&  $0.1960 \pm 0.0008$  \\
%9.1--10.1 	&  $0.1927 \pm 0.0012$  \\
11.0--12.5 	&  $0.1897 \pm 0.0010$  \\
15.0--16.0 	&  $0.1896 \pm 0.0015$  \\
16.0--18.0 	&  $0.1872 \pm 0.0012$  \\
18.0--20.0 	&  $0.1870 \pm 0.0016$  \\
\hline
1.1--6.0 	&  $0.2072 \pm 0.0005$  \\
15.0--20.0 	&  $0.1876 \pm 0.0008$  \\

\hline
\end{tabular}
\label{tab:Lb_geom_eff}
\end{table}


\section{Reconstruction and neural network efficiencies}

The efficiency to reconstruct the decays together with the stripping selection is
evaluated from simulated data. This component does not include the efficiency
of the PID cut that appears in Tab.~\ref{tab:Lb_stripping}, which is kept separate
because PID variables are not well described by the simulatation and therefore a
data-driven method is used instead (see Sec.~\ref{sec:PIDeff}).
%For the evaluation we use the most recent LHCb measurement of \Lb
%lifetime of $1.482\pm0.03$ ps\cite{LHCb--PAPER--2013--032} and \Lb polarisation $0.06\pm0.09$ \cite{Aaij:2013oxa}.
In Tab.~\ref{tab:Lb_recoEff} the reconstruction efficiency is reported in bins of \qsq and for long and downstream candidates.
In the table the efficiency is subdivided in ``Detection" and ``Reconstruction and Stripping" efficiencies.
In fact since \Lz is a long lived particles there is a non-negligible probability that it interacts in the detector
or escapes from it and therefore never decays in proton and pion. The efficiency for this to happen is what
is called "Detection" efficiency. Reconstruction and Stripping" efficiency include the efficiency of
reconstructing tracks and the efficiency for events passing the stripping cuts.

\begin{table}
\centering
\caption{Absolute Detection and reconstruction plus stripping efficiencies.
Reconstruction efficiency is given separately for DD and LL events. Errors shown are statistical only. }
\begin{tabular}{lccc}\hline
\qsq [\gevgevcccc] & Detection & Reco and Strip (DD) & Reco and Strip (LL) \\ \hline
0.1--2.0 	&  $0.8793 \pm 0.0005$	&  $0.0519 \pm 0.0006$	&  $0.0194 \pm 0.0004$  \\
2.0--4.0 	&  $0.8850 \pm 0.0004$	&  $0.0664 \pm 0.0006$	&  $0.0195 \pm 0.0004$  \\
4.0--6.0 	&  $0.8902 \pm 0.0004$	&  $0.0717 \pm 0.0007$	&  $0.0209 \pm 0.0004$  \\
6.0--8.0 	&  $0.8962 \pm 0.0005$	&  $0.0756 \pm 0.0007$	&  $0.0212 \pm 0.0004$  \\
%9.1--10.1 	&  $0.9022 \pm 0.0007$	&  $0.0787 \pm 0.0011$	&  $0.0220 \pm 0.0006$  \\
11.0--12.5 	&  $0.9084 \pm 0.0006$	&  $0.0799 \pm 0.0009$	&  $0.0221 \pm 0.0005$  \\
15.0--16.0 	&  $0.9187 \pm 0.0009$	&  $0.0736 \pm 0.0012$	&  $0.0179 \pm 0.0007$  \\
16.0--18.0 	&  $0.9247 \pm 0.0007$	&  $0.0696 \pm 0.0010$	&  $0.0169 \pm 0.0005$  \\
18.0--20.0 	&  $0.9318 \pm 0.0009$	&  $0.0600 \pm 0.0011$	&  $0.0136 \pm 0.0006$  \\
\hline
1.1--6.0 	&  $0.8868 \pm 0.0003$	&  $0.0684 \pm 0.00041$	&  $0.0202 \pm 0.0002$  \\
15.0--20.0 	&  $0.9260 \pm 0.0005$	&  $0.0669 \pm 0.00063$	&  $0.0159 \pm 0.0003$  \\

\hline
\end{tabular}
\label{tab:Lb_recoEff}
\end{table}

%\subsection{Neural Networks efficiency}

The MVA efficiency is again evaluated from simulated samples. Results are shown in Tab.~\ref{tab:Lb_mvaEff} in bins of \qsq.
The sudden jump in efficiency before and after $\sim 9 $ \gevcc is due to the fact that
a different optimisation is performed for the MVA cut in the low and high \qsq regions.

\begin{table}
\centering
\caption{Neural network selection efficiency. Errors shown are statistical only.}
\begin{tabular}{lcc} \hline
\qsq [\gevgevcccc] & MVA eff. (DD) & MVA eff. (LL)\\ \hline
0.1--2.0 	&  $0.623 \pm 0.008$	&  $0.813 \pm 0.011$  \\
2.0--4.0 	&  $0.583 \pm 0.007$	&  $0.757 \pm 0.011$  \\
4.0--6.0 	&  $0.584 \pm 0.007$	&  $0.776 \pm 0.011$  \\
6.0--8.0 	&  $0.588 \pm 0.007$	&  $0.778 \pm 0.011$  \\
%9.1--10.1 	&  $0.904 \pm 0.006$	&  $0.948 \pm 0.008$  \\
11.0--12.5 	&  $0.888 \pm 0.005$	&  $0.944 \pm 0.007$  \\
15.0--16.0 	&  $0.882 \pm 0.007$	&  $0.929 \pm 0.012$  \\
16.0--18.0 	&  $0.847 \pm 0.007$	&  $0.928 \pm 0.009$  \\
18.0--20.0 	&  $0.831 \pm 0.009$	&  $0.889 \pm 0.016$  \\
\hline
1.1--6.0 	&  $0.584 \pm 0.005$	&  $0.772 \pm 0.007$  \\
15.0--20.0 	&  $0.849 \pm 0.005$	&  $0.917 \pm 0.007$  \\
\hline
\end{tabular}
\label{tab:Lb_mvaEff}
\end{table}

%This efficiency component was crosschecked using \jpsi real data. In these events the peak is well visible even before
%the MVA cut. Therefore it is possible to fit the peak to subtract the background and then do the same procedure after the MVA cut.
%Counting events in an interval of 20 $MeV/c^2$ around the \jpsi invariant mass and (5605,5635) in \Lb invariant mass
%we obtain the following efficiencies: 86.1\% for DD events and 93.1\% for LL.
%Given the uncertainty on the background subtraction prodecure we quintify the uncertainty on this procedure
%in $\sim 4\%$ by varing mass windows and look at the change in the efficiency obtained. The pure statistical error
%on these values is 0.7\%. Therefore comparing numbers in table \ref{tab:jpsiEff}, obtained using \jpsi\Lz MC, we find them compatible within 1 sigma.   



\section{Trigger efficiency}
\label{sec:Lb_trigger_eff}

The trigger efficiency is again calculated on a simulated sample for events which are accepted by the full selection.
Using the resonant channel it is possible to crosscheck on data the efficiency obtained using the simulation with the data driven TISTOS method.
In LHCb triggered events can fall in two categories: events triggered by a track which is part of a signal candidate, 
Trigger On Signal (TOS), or by other tracks in the event, Trigger Independent of Signal (TIS). All trigger lines used
for this analysis are required to be TOS. As the TIS and TOS categories are not exclusive the TIS sample provides a control
sample which can be used to obtain the efficiency for TOS trigger. This is calculated with the formula:
\begin{equation}
\varepsilon_{TOS} = \frac{TOS \mbox{ and } TIS}{TIS}
\end{equation}
Using the data--driven method an efficiency of ($70 \pm 5$)\% is obtained, while this is calculated to be
($73.33 \pm 0.02$)\% using the simulation. Results are therefore compatible within $1\sigma$. 


\begin{table}
\centering
\caption{Absolute trigger efficiencies for selected events as determined
from the simulation separately for LL and DD events.}
\begin{tabular}{lcc} \hline
\qsq [\gevgevcccc] & Trigger eff. (DD) & Trigger eff. (LL)\\ \hline
0.1--2.0 	&  $0.560 \pm 0.008$	&  $0.577 \pm 0.012$  \\
2.0--4.0 	&  $0.606 \pm 0.006$	&  $0.651 \pm 0.010$  \\
4.0--6.0 	&  $0.623 \pm 0.006$	&  $0.674 \pm 0.010$  \\
6.0--8.0 	&  $0.669 \pm 0.006$	&  $0.706 \pm 0.010$  \\
%9.1--10.1 	&  $0.700 \pm 0.007$	&  $0.722 \pm 0.013$  \\
11.0--12.5 	&  $0.744 \pm 0.006$	&  $0.738 \pm 0.011$  \\
15.0--16.0 	&  $0.818 \pm 0.008$	&  $0.826 \pm 0.015$  \\
16.0--18.0 	&  $0.836 \pm 0.006$	&  $0.860 \pm 0.011$  \\
18.0--20.0 	&  $0.857 \pm 0.008$	&  $0.863 \pm 0.015$  \\
\hline
1.1--6.0 	&  $0.610 \pm 0.004$	&  $0.653 \pm 0.007$  \\
15.0--20.0 	&  $0.839 \pm 0.004$	&  $0.853 \pm 0.008$  \\
\hline
\end{tabular}
\label{tab:Lb_triggerEfficiency}
\end{table}


\section{PID efficiency}
\label{sec:PIDeff}
For long tracks a PID cut on protons (\verb!PID!p$ > -5$) is used. The simulation is known not to
describe particle ID well and therefore data-driven method is used to obtain this efficiency component.
This is done using the \verb!PIDCalib! package, which uses decays where particles
can be identified due to their kinematic properties. In the case of protons a sample of
\Lz particles was used where the proton can be identified because it always has the highest momentum.
The package allows to divide the phase space in bins of variables relevant for PID
performances, in this analysis momentum and pseudorapidity are used.
Using the calibration sample the efficiency is derived in each two-dimensional bin.
To take in account that the decay channel under study could have different kinematical distributions
than the calibration sample these efficiency tables are used to re-weight the simulation.
Absolute PID efficiencies are shown in Tab.~\ref{tab:Lb_PIDabs} and the relative efficiency 
over the resonant channel are reported in Tab.~\ref{tab:Lb_PIDrel}.

\begin{table}
\centering
\caption{Absolute PID efficiencies in \qsq bins}
\begin{tabular}{lc} \hline
\qsq [\gevgevcccc]	&       PID efficiency      \\  \hline
0.1--2.0     &  $97.32 \pm 0.012$   \\
2.0--4.0     &  $97.42 \pm 0.012$   \\
4.0--6.0     &  $97.59 \pm 0.011$   \\
6.0--8.0     &  $97.70 \pm 0.010$   \\
11.0--12.5   &  $98.04 \pm 0.009$   \\
15.0--16.0   &  $98.31 \pm 0.006$   \\
16.0--18.0   &  $98.10 \pm 0.005$   \\
18.0--20.0   &  $98.11 \pm 0.001$  \\
\hline
1.1--6.0     &  $97.49 \pm 0.007$   \\
15.0--20.0   &  $98.17 \pm 0.003$  \\
\jpsi       &  $97.89 \pm 0.005$   \\
\hline
\end{tabular}
\label{tab:Lb_PIDabs}
\end{table}



\section{Relative efficiencies}

In the previous sections absolute efficiencies values were given for the rare channel in different \qsq bins.
In this section the corresponding relative efficiencies with respecte to the \Lb\to\jpsi\Lz channel are reported, which will be used for
the differential branching ratio calculation. In Tab.~\ref{tab:jpsiEff} the absolute efficiency values
for the \jpsi channel are also reported. Relative geometric efficiencies are listed in Tab.~\ref{tab:relativeGeometric},
In Tabs.~\ref{tab:allRelativeEffDD} and \ref{tab:allRelativeEffLL} relative reconstruction, trigger and mva efficiencies 
are shown separately for downstream and long candidates. Since these three components are obtained from the same simulated sample
their statistical errors are correlated. Therefore the total of the three is also reported as a single efficiency
and labeled ``Full Selection" in the table. Finally, the relative PID efficiency is reported in Tab.~\ref{tab:Lb_PIDrel}.
Figure~\ref{fig:Lb_relativeEff} shows the values in these tables in graphical form.
Finally, in Tab.~\ref{tab:Lb_effSummary} is reported the total of all relative efficiencies, which will be then used
to correct the raw yields and calculate the differential branching fraction.
Uncertainties reflect statistics of both rare and resonant samples, while systematic uncertainties are discussed in next chapter.

\begin{table}
\centering
\caption{Absolute efficiency values for $\Lb\to\jpsi\Lz$. Errors shown are statistical only.}
\begin{tabular}{lcc} \hline
Efficiency		& 	Downstream				& 	Long				\\  \hline		
Geometric 		&   	\multicolumn{2}{c}{	$0.1818 \pm 0.0003$ } 	\\	
Detection 		&   	\multicolumn{2}{c}{	$0.9017 \pm 0.0003$ } 	\\
Reconstruction 	& 	$0.0724 \pm 0.0004$   & $0.0203 \pm 0.0002$     \\
MVA 			&	$0.882 \pm 0.002$   & $0.942 \pm 0.002$     \\
Triger 			&	$0.697 \pm 0.003$   & $0.734 \pm 0.005$     \\ \hline
Full Selection	&	$0.0445 \pm 0.0003$   & $0.0140 \pm 0.0002$     \\ \hline
Total  			&	$0.00729 \pm 0.00005$   & $0.00230 \pm 0.00003$    	\\
\end{tabular}
\label{tab:jpsiEff}
\end{table}


\begin{table}
\centering
\caption{Relative geometric efficiency and Detection efficiency between
$\Lb\to\Lz\mumu$ and $\Lb\to\jpsi\Lz$ decays.
Uncertainty reflects statistics of both samples.}
\begin{tabular}{lcc} \hline
\qsq [\gevgevcccc] & Geometric & Detection  \\ \hline
0.1--2.0 	&  $1.2976 \pm 0.0050$ 	&  $0.9751 \pm 0.0006$  \\
2.0--4.0 	&  $1.1541 \pm 0.0043$ 	&  $0.9814 \pm 0.0005$  \\
4.0--6.0 	&  $1.1043 \pm 0.0044$ 	&  $0.9872 \pm 0.0006$  \\
6.0--8.0 	&  $1.0778 \pm 0.0045$ 	&  $0.9939 \pm 0.0006$  \\
%9.1--10.1 	&  $1.0596 \pm 0.0065$ 	&  $1.0005 \pm 0.0008$  \\
11.0--12.5 	&  $1.0431 \pm 0.0058$ 	&  $1.0074 \pm 0.0007$  \\
15.0--16.0 	&  $1.0426 \pm 0.0084$ 	&  $1.0188 \pm 0.0010$  \\
16.0--18.0 	&  $1.0296 \pm 0.0068$ 	&  $1.0255 \pm 0.0008$  \\
18.0--20.0 	&  $1.0288 \pm 0.0087$ 	&  $1.0333 \pm 0.0010$  \\
\hline
1.1--6.0 	&  $1.1396 \pm 0.0031$ 	&  $0.9835 \pm 0.0004$  \\
15.0--20.0 	&  $1.0320 \pm 0.0048$ 	&  $1.0269 \pm 0.0006$  \\
\hline
\end{tabular}
\label{tab:relativeGeometric}
\end{table}


\begin{table}
\centering
\caption{Relative efficiencies between $\Lb\to\Lz\mumu$ and $\Lb\to\jpsi\Lz$ decays for long events.
Uncertainty reflects statistics of both samples.}
\begin{tabular}{lccccc} \hline
\qsq [\gevgevcccc]      & Reco and strip          & MVA                 & Trigger         & Full Selection \\
\hline
0.1--2.0 	&  $0.96 \pm 0.02$ 	&  $0.863 \pm 0.012$ 	&  $0.79 \pm 0.02$ 	&  $0.65 \pm 0.02$  \\
2.0--4.0 	&  $0.97 \pm 0.02$ 	&  $0.803 \pm 0.012$ 	&  $0.89 \pm 0.02$ 	&  $0.69 \pm 0.02$  \\
4.0--6.0 	&  $1.04 \pm 0.02$ 	&  $0.824 \pm 0.012$ 	&  $0.92 \pm 0.02$ 	&  $0.79 \pm 0.02$  \\
6.0--8.0 	&  $1.05 \pm 0.02$ 	&  $0.825 \pm 0.012$ 	&  $0.96 \pm 0.02$ 	&  $0.84 \pm 0.02$  \\
%9.1--10.1 	&  $1.08 \pm 0.03$ 	&  $1.007 \pm 0.009$ 	&  $0.98 \pm 0.02$ 	&  $1.07 \pm 0.04$  \\
11.0--12.5 	&  $1.10 \pm 0.03$ 	&  $1.002 \pm 0.008$ 	&  $1.01 \pm 0.02$ 	&  $1.10 \pm 0.03$  \\
15.0--16.0 	&  $0.89 \pm 0.03$ 	&  $0.987 \pm 0.013$ 	&  $1.13 \pm 0.02$ 	&  $0.98 \pm 0.04$  \\
16.0--18.0 	&  $0.84 \pm 0.03$ 	&  $0.985 \pm 0.010$ 	&  $1.17 \pm 0.02$ 	&  $0.97 \pm 0.03$  \\
18.0--20.0 	&  $0.67 \pm 0.03$ 	&  $0.944 \pm 0.017$ 	&  $1.18 \pm 0.02$ 	&  $0.75 \pm 0.04$  \\
\hline
1.1--6.0 	&  $1.00 \pm 0.02$ 	&  $0.820 \pm 0.008$ 	&  $0.89 \pm 0.01$ 	&  $0.73 \pm 0.02$  \\
15.0--20.0 	&  $0.78 \pm 0.02$ 	&  $0.973 \pm 0.008$ 	&  $1.16 \pm 0.01$ 	&  $0.89 \pm 0.02$  \\

\hline
\end{tabular}
\label{tab:allRelativeEffLL}
\end{table}

\begin{table}
\centering
\caption{Relative efficiencies between $\Lb\to\Lz\mumu$ and $\Lb\to\jpsi\Lz$ decays for downstream events.
Uncertainty reflects statistics of both samples.}
\begin{tabular}{lcccccc} \hline
\qsq [\gevgevcccc]      & Reco and strip          & MVA                 & Trigger       & Full Selection      \\
\hline
0.1--2.0 	&  $0.721 \pm 0.009$ 	&  $0.706 \pm 0.010$ 	&  $0.805 \pm 0.011$ 	&  $0.410 \pm 0.009$  \\
2.0--4.0 	&  $0.920 \pm 0.010$ 	&  $0.661 \pm 0.008$ 	&  $0.870 \pm 0.010$ 	&  $0.529 \pm 0.010$  \\
4.0--6.0 	&  $0.997 \pm 0.010$ 	&  $0.662 \pm 0.008$ 	&  $0.895 \pm 0.010$ 	&  $0.590 \pm 0.011$  \\
6.0--8.0 	&  $1.050 \pm 0.011$ 	&  $0.665 \pm 0.008$ 	&  $0.960 \pm 0.010$ 	&  $0.671 \pm 0.012$  \\
%9.1--10.1 	&  $1.092 \pm 0.015$ 	&  $1.025 \pm 0.007$ 	&  $1.005 \pm 0.011$ 	&  $1.125 \pm 0.021$  \\
11.0--12.5 	&  $1.112 \pm 0.014$ 	&  $1.007 \pm 0.006$ 	&  $1.069 \pm 0.009$ 	&  $1.197 \pm 0.019$  \\
15.0--16.0 	&  $1.019 \pm 0.018$ 	&  $1.000 \pm 0.009$ 	&  $1.175 \pm 0.012$ 	&  $1.197 \pm 0.026$  \\
16.0--18.0 	&  $0.968 \pm 0.014$ 	&  $0.961 \pm 0.008$ 	&  $1.200 \pm 0.010$ 	&  $1.115 \pm 0.020$  \\
18.0--20.0 	&  $0.832 \pm 0.016$ 	&  $0.943 \pm 0.010$ 	&  $1.231 \pm 0.012$ 	&  $0.966 \pm 0.023$  \\
\hline
1.1--6.0 	&  $0.950 \pm 0.007$ 	&  $0.663 \pm 0.005$ 	&  $0.876 \pm 0.007$ 	&  $0.551 \pm 0.007$  \\
15.0--20.0 	&  $0.929 \pm 0.010$ 	&  $0.963 \pm 0.005$ 	&  $1.204 \pm 0.007$ 	&  $1.077 \pm 0.014$  \\

\hline
\end{tabular}
\label{tab:allRelativeEffDD}
\end{table}


\begin{table}
\centering
\caption{Relative PID efficiencies in \qsq bins}
\begin{tabular}{lc} \hline
\qsq [\gevgevcccc]	&     Rel.  PID Eff.            \\  \hline
0.1--2.0    & $0.99418 \pm 0.00013$  \\
2.0--4.0    & $0.99523 \pm 0.00013$   \\
4.0--6.0    & $0.99699 \pm 0.00012$   \\
6.0--8.0    & $0.99805 \pm 0.00011$   \\
11.0--12.5  & $1.00151 \pm 0.00010$   \\
15.0--16.0  & $1.00431 \pm 0.00008$   \\
16.0--18.0  & $1.00215 \pm 0.00008$   \\
518.0--20.0  & $1.00226 \pm 0.00005$   \\
\hline
1.1--6.0    & $0.99589 \pm 0.00009$   \\
15.0--20.0  & $1.00281 \pm 0.00006$  \\
\hline
\end{tabular}
\label{tab:Lb_PIDrel}
\end{table}



%\begin{table}
%\centering
%\begin{tabular}{lcc} \hline\hline
%\qsq bin & Tot. rel. eff. DD &  Tot. rel. eff. LL \\ \hline

%0.1--2.0 	&  $0.51821 \pm 0.01208$	&  $0.82385 \pm 0.02846$  \\
%2.0--4.0 	&  $0.59863 \pm 0.01200$	&  $0.78194 \pm 0.02444$  \\
%4.0--6.0 	&  $0.64365 \pm 0.01253$	&  $0.85667 \pm 0.02594$  \\
%6.0--8.0 	&  $0.71846 \pm 0.01360$	&  $0.89726 \pm 0.02670$  \\
%9.1--10.1 	&  $1.19267 \pm 0.02374$	&  $1.13929 \pm 0.04046$  \\
%11.0--12.5 	&  $1.25735 \pm 0.02163$	&  $1.15949 \pm 0.03701$  \\
%15.0--16.0 	&  $1.27185 \pm 0.02971$	&  $1.04571 \pm 0.04716$  \\
%16.0--18.0 	&  $1.17743 \pm 0.02295$	&  $1.01962 \pm 0.03663$  \\
%18.0--20.0 	&  $1.02651 \pm 0.02607$	&  $0.79500 \pm 0.03937$  \\
%\hline
%1.1--6.0 	&  $0.61800 \pm 0.00855$	&  $0.81966 \pm 0.01804$  \\
%15.0--20.0 	&  $1.14181 \pm 0.01622$	&  $0.94460 \pm 0.02518$  \\

%\hline
%\end{tabular}
%\caption{Total relative efficiency between $\Lb\to\Lz\mumu$ and $\Lb\to\jpsi\Lz$ decays. For downstream events and long events.}
%\label{tab:totalRelativeEff}
%\end{table}
	

\begin{figure}
\centering
\includegraphics[width=0.48\textwidth]{Lmumu/figs/efficiencies/BR/effvsq2_DD_geom.pdf}
\includegraphics[width=0.48\textwidth]{Lmumu/figs/efficiencies/BR/effvsq2_DD_det.pdf}
\includegraphics[width=0.48\textwidth]{Lmumu/figs/efficiencies/BR/effvsq2_DD_reco.pdf}
\includegraphics[width=0.48\textwidth]{Lmumu/figs/efficiencies/BR/effvsq2_LL_reco.pdf}
\includegraphics[width=0.48\textwidth]{Lmumu/figs/efficiencies/BR/effvsq2_DD_mva.pdf}
\includegraphics[width=0.48\textwidth]{Lmumu/figs/efficiencies/BR/effvsq2_LL_mva.pdf}
\includegraphics[width=0.48\textwidth]{Lmumu/figs/efficiencies/BR/effvsq2_DD_trig.pdf}
\includegraphics[width=0.48\textwidth]{Lmumu/figs/efficiencies/BR/effvsq2_LL_trig.pdf}
\caption{Relative efficiencies as a function of \qsq. geometric efficiency (a), 
detection efficiency (b), reconstruction efficiency for DD (c) and LL (d) events, 
trigger efficiency for DD (e) and LL (f) and MVA efficiency for DD (g) and LL (h).}
\label{fig:Lb_relativeEff}
\end{figure}

\clearpage
