\section{Importance of inputs in NeuroBayes}
\label{ap:nnImportance}

All quantities related to the importance of the input
are calculated based on the correlation of
the variables to the classification truth. To calculate
the  correlation significance,
one calculates for each variable the correlation to the truth
defined as
$$
\rho^{ti}=\frac{\frac{1}{n}\sum_{j=1}^{n}\left(x^{t}_{j}-<x^{t}>\right)\cdot\left(x^{i}_{j}-<x^i>\right)}{\sqrt{V[x^t]V[x^i]}},
$$
where $t$ denotes the truth, $i$ the correlated
variable, $<x^i>$ the expectation
value for the given variable, $V[x^i]$ its variance and $n$ is
number of events in training sample. Then the
correlation significance for a given variable is
$\rho^{ti}\cdot\sqrt{n}$. For the significance loss, we first
decorrelate all input quantities and than calculate the
correlation between the truth and each decorrelated variable
$\tilde\rho^{ti}$. The total correlation is then defined by
$$
\rho_{TN}^2=\sum_{i=1}^{N}\tilde\rho^{ti2},
$$
where $N$ is the number of variables. An analogous procedure is
repeated without the considered variable to calculate
$\rho_{TN-1}^2$. The significance loss is then given by
$$
\left(\rho_{TN}^2-\rho_{TN-1}^2\right)\cdot\sqrt{n}.
$$
Intuitively, the correlation significance is proportional to
the amount of information which is provided by a given quantity
without all others while the significance loss corresponds to
the amount of information lost for the neural network if we
remove the given quantity while we keep all others.

