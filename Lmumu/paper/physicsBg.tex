\section{Peaking backgrounds}
\label{sec:physbg}
 In addition to combinatorial background formed from the random
 combination of particles, backgrounds due to specific decays are
 studied using fully reconstructed samples of simulated \bquark hadron
 decays in which the final state includes two muons.  For the
 \decay{\Lb}{\jpsi\Lz} channel, the only significant contribution is
 from \decay{\Bz}{\jpsi\KS} decays, with \decay{\KS}{\pip\pim} where
 one of the pions is misidentified as a proton.  This decay contains a
 long-lived \KS meson and therefore has the same topology as the
 \decay{\Lb}{\jpsi\Lz} mode. This contribution leads to a broad shape
 that peaks below the \Lb mass region, which is taken into account in
 the mass fit.

 For the \decay{\Lb}{\Lz\mumu} channel two sources of peaking
 background are identified. The first of these is
 \decay{\Lb}{\jpsi\Lz} decays in which an energetic photon is radiated
 from either of the muons; this constitutes a background in the \qsq
 region just below the square of the \jpsi mass and in a mass region
 significantly below the \Lb mass.  These events do not contribute
 significantly in the \qsq intervals chosen for the analysis.  The
 second source of background is due to \decay{\Bz}{\KS\mumu} decays,
 where \decay{\KS}{\pip\pim} and one of the pions is misidentified as
 a proton.  This contribution is estimated by scaling the number of
 \decay{\Bz}{\jpsi\KS} events found in the \decay{\Lb}{\jpsi\Lz} fit
 by the ratio of the world average branching fractions for the decay
 processes \decay{\Bz}{\KS\mumu} and \decay{\Bz}{\jpsi(\to\mumu)\KS}
 \cite{Agashe:2014kda}.  Integrated over \qsq this is estimated to
 yield fewer than ten events, which is small relative to the expected
 total background level.
