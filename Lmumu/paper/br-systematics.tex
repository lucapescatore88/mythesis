\section{Systematic uncertainties on the branching fraction}
\label{sec:systemtics}
\subsection{Yields}
\label{sec:systematics_yields}
 Three sources of systematic uncertainty on the measured yields are
 considered for both the \decay{\Lb}{\jpsi\Lz} and the
 \decay{\Lb}{\Lz\mumu} decay modes: the shape of the signal PDF, the
 shape of the background PDF and the choice of the fixed parameters
 used in the fits to data.

 For both decays, the default signal PDF is replaced by the sum of two
 Gaussian functions. All parameters of the Gaussian functions are
 allowed to vary to take into account the effect of fixing
 parameters. The shape of the background function is changed by
 permitting the \KS\mumu peaking background yield, which is fixed to
 the value obtained from simulation the nominal fit, to vary.  For the
 resonant channel, the \jpsi\KS peaking background shape is changed by
 fixing the global shift to zero.  Finally, simulated experiments are
 performed using the default model, separately for each \qsq interval,
 generating the same number of events as observed in data.  Each
 distribution is fitted with the default model and the modified
 PDFs. The average deviation over the ensemble of simulated
 experiments is assigned as the systematic uncertainty.  The relative
 change in signal yield due to the choice of signal PDF varies between
 0.6\,\% and 4.6\,\% depending on \qsq, while the change due to the
 choice of background PDF is in the range between 1.1\,\% and
 2.5\,\%. The \qsq intervals that are most affected are those in which
 a smaller number of candidates is observed and therefore there are
 fewer constraints to restrict potentially different PDFs.  The
 systematic uncertainties on the yield in each \qsq interval are
 summarised in Table~\ref{tab:brsys}, where the total is the sum in
 quadrature of the individual components.

\subsection{Relative efficiencies}
 \label{sec:systematics_eff}
 The dominant systematic effect is that related to the current
 knowledge of the angular structure and the \qsq dependence of the
 decay channels.  The uncertainty due to the finite size of simulated
 samples is comparable to that from other sources. The total
 systematic uncertainties on the efficiencies, calculated as the sums
 in quadrature of the individual components described below,
 are summarised in Table~\ref{tab:brsys}.


\subsubsection{Decay structure and production polarisation}
 The main factors that affect the detection efficiencies are the
 angular structure of the decays and the production polarisation
 ($P_b$). Although these arise from different parts of the process,
 the efficiencies are linked and are therefore treated together.

 For the \decay{\Lb}{\Lz\mumu} decay, the impact of the limited
 knowledge of the production polarisation, $P_b$, is estimated by
 comparing the default efficiency, obtained in the unpolarised
 scenario, with those in which the polarisation is varied within its
 measured uncertainties, using the most recent LHCb measurement, $P_b
 = 0.06 \pm 0.09$\cite{LHCb-PAPER-2012-057}.  The larger of these
 differences is assigned as the systematic uncertainty from this
 source. This yields a $\sim 0.5\,\%$ uncertainty on the efficiency of
 downstream candidates and $\sim 1.2\,\%$ for long candidates.  No
 significant \qsq dependence is found.

 To assess the systematic uncertainty due to the limited knowledge of
 the decay structure, the efficiency corresponding to the default
 model \cite{Gutsche:2013pp,Aliev:2005np,Buras:1994dj} is compared to
 that of a model containing an alternative set of form factors based
 on a lattice QCD calculation \cite{Detmold:2012vy}. The larger of the
 full difference or the statistical precision is assigned as the
 systematic uncertainty.
 
 For the \decay{\Lb}{\jpsi\Lz} mode, the default angular distribution
 is based on that observed in Ref.  \cite{LHCb-PAPER-2012-057}. The
 angular distribution is determined by the production polarisation and
 four complex decay amplitudes. The central values from
 Ref.~\cite{LHCb-PAPER-2012-057} are used for the nominal result. To
 assess the sensitivity of the \decay{\Lb}{\jpsi\Lz} mode to the
 choice of decay model, the production polarisation and decay
 amplitudes are varied within their uncertainties, taking into account
 correlations.

 To assess the potential impact that physics beyond the SM might have
 on the detection efficiency, the $C_7$ and $C_9$ Wilson coefficients
 are modified by adding a non-SM contribution ($C_i \rightarrow C_i +
 C_i^{'}$). The $C_i^{'}$ added are inspired to maintain compatibility
 with the recent LHCb result for the $P'_5$ observable
 \cite{Descotes-Genon:2013wba} and indicate a change at the level of
 $\sim 7$\,\% in the 0.1--2.0 \qsq interval, and 2--3\,\% in other
 regions.  No systematic is assigned as a result of this study.

 \subsubsection{Reconstruction efficiency for the \Lz baryon}
 The \Lz baryon is reconstructed from either long or downstream
 tracks, and their relative proportions differ in data and simulation.
 This proportion does not depend significantly on \qsq and therefore
 possible effects cancel in the ratio with the normalisation channel.
 Furthermore, since the analysis is performed separately for long and
 downstream candidates, it is not necessary to assign a systematic
 uncertainty to account for a potential effect due to the different
 fractions of candidates of the two categories observed in data and
 simulation.  To allow for residual differences between data and
 simulation that do not cancel completely in the ratio between signal
 and normalisation modes, systematic uncertainties of 0.8\,\% and
 1.2\,\% are estimated for the low-\qsq and high-\qsq regions,
 respectively, using the same data-driven method as in
 Ref.~\cite{LHCb-PAPER-2014-006}.


 \subsubsection{Production kinematics and lifetime of the \Lb baryon}
  In \decay{\Lb}{\jpsi\Lz} decays a small difference is observed
  between data and simulation in the momentum and transverse momentum
  distributions of the \Lb baryon produced. Simulated data are
  reweighted to reproduce these distributions in data and the relative
  efficiencies are compared to those obtained using events that are
  not reweighted.  This effect is less than 0.1\,\%, which is
  negligible with respect to other sources.

  Finally, the \Lb baryon lifetime used throughout corresponds to the
  most recent \lhcb measurement, $1.479\pm0.019$\ps
  \cite{LHCb-PAPER-2014-003}.  The associated systematic uncertainty
  is estimated by varying the lifetime value by one standard deviation
  and negligible differences are found.


 \begin{table}[tbp]
 \centering
 \caption{Systematic uncertainties as a function of \qsq, assigned for
   yields and efficiencies. Values reported are the sums in quadrature
   of all contributions evaluated within each category. }
 \label{tab:brsys}
 \renewcommand{\arraystretch}{1.3}
 \begin{tabular}{ccc}
  %%Nigel Watson 20150212 $q^2$ interval [\gevgevcccc] & \multicolumn{2}{c}{\text{ } Syst. on
  %%Nigel Watson 20150212    eff. [\%] \text{ }} & Syst. on yields [\%] \\ \hline
 $q^2$ interval [\gevgevcccc]  & Syst.\ on yields [\%] & Syst.\ on
  eff. [\%] \\ \hline

0.1 -- 2.0  	 	& 3.4  &	 $_{-3.6}^{+2.2}$ 	  \\
2.0 -- 4.0  	 	& 3.8  &	 $_{-4.1}^{+2.2}$ 	  \\
4.0 -- 6.0  	 	& 6.6  &	 $_{-14.3}^{+17.2}$   \\
6.0 -- 8.0  	 	& 2.0  &	 $_{-3.1}^{+2.1}$ 	  \\
11.0 -- 12.5  	 	& 3.2  &	 $_{-5.2}^{+3.7}$ 	  \\
15.0 -- 16.0  	 	& 2.8  &	 $_{-2.8}^{+3.1}$ 	  \\
16.0 -- 18.0  	 	& 1.4  &	 $_{-4.1}^{+3.0}$ 	  \\
18.0 -- 20.0  	 	& 2.5  &	 $_{-2.3}^{+3.9}$ 	  \\
\hline
1.1 -- 6.0  	 	& 4.2  &	 $_{-4.6}^{+2.2}$ 	  \\
15.0 -- 20.0       	& 1.0  &    $_{-2.9}^{+2.0}$      \\

%%%%% Old table with no WC systematics
%0.1 -- 2.0       & 3.4 & $^{+6.4}_{-7.3}$\\
%2.0 -- 4.0       & 3.8 & $^{+4.9}_{-5.4}$\\
%4.0 -- 6.0       & 6.6 & $^{+1.9}_{-2.0}$\\
%6.0 -- 8.0       & 2.0 & $^{+3.8}_{-4.1}$\\
%11.0 -- 12.5     & 3.2 & $^{+3.2}_{-3.4}$\\
%15.0 -- 16.0     & 2.8 & $^{+3.5}_{-3.8}$\\
%16.0 -- 18.0     & 1.4 & $^{+3.1}_{-3.3}$\\
%18.0 -- 20.0     & 2.5 & $^{+3.1}_{-3.3}$\\
%\hline                                 
%1.1 --  6.0      & 4.2 & $^{+5.3}_{-6.0}$\\
%15.0 -- 20.0     & 1.0 & $^{+1.8}_{-1.9}$\\
\end{tabular}
\end{table}


