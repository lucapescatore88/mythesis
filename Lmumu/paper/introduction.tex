\section{Introduction}
The decay \decay{\Lb}{\Lz\mumu} is a rare (\decay{\bquark}{\squark})
flavour-changing neutral current process that, in the Standard Model
(SM), proceeds through electroweak loop (penguin and \Wpm box)
diagrams.  As non-SM particles may also contribute to the decay
amplitudes, measurements of this and similar decays can be used to
search for physics beyond the SM. To date, emphasis has been placed on
the study of rare decays of mesons rather than baryons, in part due to
the theoretical complexity of the latter \cite{Mannel:1997xy}.  In the
particular system studied in this analysis, the decay products include
only a single long-lived hadron, simplifying the theoretical modelling
of hadronic physics in the final state.

The study of \Lb baryon decays is of considerable interest for several
reasons.  Firstly, as the \Lb baryon {has non-zero spin}, there is the
potential to improve the limited understanding of the helicity
structure of the underlying Hamiltonian, which cannot be extracted
from meson decays \cite{Hiller:2007ur,Mannel:1997xy}. Secondly, as the
\Lb baryon may be considered as consisting of a heavy quark combined
with a light diquark system, the hadronic physics differs
significantly from that of the \B meson decay.  A further motivation
specific to the \decay{\Lb}{\Lz\mumu} channel is that the polarisation
of the \Lz baryon is preserved in the \decay{\Lz}{\proton\pim}
decay\footnote{The inclusion of charge-conjugate modes is implicit
  throughout.}, giving access to complementary information to that
available from meson decays \cite{Boer:2014kda}.

Theoretical aspects of the \decay{\Lb}{\Lz\mumu} decay have been
considered both in the SM and in some of its extensions
\cite{Boer:2014kda,Aslam:2008hp,Wang:2008sm,Huang:1998ek,Chen:2001ki,Chen:2001zc,
  Chen:2001sj,Zolfagharpour:2007eh,Mott:2011cx,Aliev:2010uy,Mohanta:2010eb,Sahoo:2011yb,Detmold:2012vy,Gutsche:2013pp}.
Although based on the same effective Hamiltonian as that for the
corresponding mesonic transitions, the hadronic form factors for the
\Lb baryon case are less well-known due to the less stringent
experimental constraints.  This leads to a large spread in the
predicted branching fractions.  The decay has a non-trivial angular
structure which, in the case of unpolarised \Lb production, is
described by the helicity angles of the muon and proton, the angle
between the planes defined by the \Lz decay products and the two
muons, and the square of the dimuon invariant mass, \qsq.  In
theoretical investigations, the differential branching fraction, and
forward-backward asymmetries for both the dilepton and the hadron
systems of the decay, have received particular attention
\cite{Boer:2014kda,Mott:2011cx,Detmold:2012vy,Meinel:2014wua,Gutsche:2013pp}.
Different treatments of form factors are used depending on the \qsq
region and can be tested by comparing predictions with data as a
function of \qsq.
 
In previous observations of the decay \decay{\Lb}{\Lz\mumu}
\cite{Aaltonen:2011qs,LHCB-PAPER-2013-025}, evidence for signal had
been limited to \qsq values above the square of the mass of the
\psitwos resonance. This region will be referred to as ``high-\qsq'',
while that below the \psitwos will be referred to as ``low-\qsq''.  In
this paper an updated measurement by \lhcb of the differential
branching fraction for the rare decay \decay{\Lb}{\Lz\mumu}, and the
first angular analysis of this decay mode, are reported.
Non-overlapping \qsq intervals in the range 0.1--20.0\gevgevcccc, and
theoretically motivated ranges 1.1--6.0 and 15.0--20.0\gevgevcccc ~\cite{Boer:2014kda,Beylich:2011aq,Beneke:2000wa}, are
used.  The rates are normalised with respect to the tree-level
\decay{\bquark}{\cquark\cquarkbar\squark} decay \decay{\Lb}{\jpsi\Lz},
where \decay{\jpsi}{\mumu}. This analysis uses $pp$ collision data,
corresponding to an integrated luminosity of 3.0\invfb, collected
during 2011 and 2012 at centre-of-mass energies of 7 and 8\tev,
respectively.


