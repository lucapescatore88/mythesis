\section{Candidate selection}

 Candidate \decay{\Lb}{\Lz\mumu} (signal mode) and
 \decay{\Lb}{\jpsi\Lz} (normalisation mode) decays are reconstructed
 from a \Lz baryon candidate and either a dimuon or a \jpsi meson
 candidate, respectively.  The \decay{\Lb}{\jpsi\Lz} mode, with the
 \jpsi meson reconstructed via its dimuon decay, is a convenient
 normalisation process because it has the same final-state particles
 as the signal mode. 
 Signal and normalisation channels are distinguished by the
 \qsq interval in which they fall. 

 The dimuon candidates are formed from two
 well-reconstructed oppositely charged particles that are
 significantly displaced from any PV, identified as muons and
 consistent with originating from a common vertex.
 
 Candidate \Lz decays are reconstructed in the
 \decay{\Lz}{\proton\pim} mode from two oppositely charged tracks that
 either both include information from the VELO (\textit{long}
 candidates), or both do not include information from the VELO
 (\textit{downstream} candidates). The \Lz candidates must also have a
 vertex fit with a good \chisq, a decay time of at least 2\ps and an
 invariant mass within 30\mevcc of the known \Lz
 mass~\cite{Agashe:2014kda}.  For long candidates, charged particles
 must have $\pt>0.25\gevc$ and a further requirement is imposed on the
 particle identification (PID) of the proton using a likelihood
 variable that combines information from the RICH detectors and the
 calorimeters.

 Candidate \Lb decays are formed from \Lz and dimuon candidates that
 have a combined invariant mass in the interval 5.3--7.0\gevcc and
 form a good-quality vertex that is well-separated from any PV.
 Candidates pointing to the PV with which they are associated are
 selected by requiring that the angle between the \Lb momentum vector
 and the vector between the PV and the \Lb decay vertex, $\theta_D$,
 is less than 14\mrad.  After the \Lb candidate is built, a kinematic
 fit \cite{Hulsbergen:2005pu} of the complete decay chain is performed
 in which the proton and pion are constrained such that the
 $\proton\pim$ invariant mass corresponds to the known \Lz baryon
 mass, and the \Lz and dimuon systems are constrained to originate
 from their respective vertices.

 The final selection is based on a neural network classifier
 \cite{Feindt:2006pm,feindt-2004}, exploiting 15 variables carrying
 kinematic, candidate quality and particle identification information.
 Both the track parameter resolutions and kinematic properties are different
 for downstream and long \Lz decays and therefore a separate training is
 performed for each category.  The signal sample used to train the neural
 network consists of simulated \decay{\Lb}{\Lz\mumu} events, while the
 background is taken from data in the upper sideband of the \Lb
 candidate mass spectrum, between 6.0 and 7.0\gevcc. Candidates with a
 dimuon mass in either the \jpsi or \psitwos regions ($\pm 100$\mevcc
 intervals around their known masses)
 are excluded from the training samples.  The variable that provides the greatest
 discrimination in the case of long candidates is the \chisq from the
 kinematic fit.  For downstream candidates, the \pt of the \Lz
 candidate is the most powerful variable.  Other variables that
 contribute significantly are: the PID information for muons; the
 separation of the muons, the pion and the \Lb\ candidate from the PV;
 the distance between the \Lz and \Lb decay vertices; and the pointing
 angle, $\theta_D$.
 
 The requirement on the response of the neural network classifier is
 chosen separately for low- and high-\qsq candidates using two
 different figures of merit.  In the low-\qsq region, where the signal
 has not been previously established, the figure of merit $\varepsilon
 /(\sqrt{N_{\mathrm{B}}} + a/2)$~\cite{Punzi:2003bu} is used, where
 $\varepsilon$ and $N_{\mathrm{B}}$ are the signal efficiency and the
 expected number of background decays and $a$ is the target
 significance; a value of $a = 3$ is used.  In contrast, for the
 high-\qsq region the figure of merit
 $N_{\mathrm{S}}/\sqrt{N_{\mathrm{S}}+N_{\mathrm{B}}}$ is maximised,
 where $N_{\mathrm{S}}$ is the expected number of signal candidates.
 To ensure an appropriate normalisation of $N_{\mathrm{S}}$, the
 number of \decay{\Lb}{\jpsi\Lz} candidates that satisfy the
 preselection is scaled by the measured ratio of branching fractions
 of \decay{\Lb}{\Lz\mumu} to \decay{\Lb}{\jpsi(\to\mumu)\Lz}
 decays~\cite{LHCB-PAPER-2013-025}, and the \decay{\jpsi}{\mumu}
 branching fraction~\cite{Agashe:2014kda}.  The value of
 $N_\mathrm{B}$ is determined by extrapolating the number of candidate
 decays found in the background training sample into the signal
 region.  Relative to the preselected event sample, the neural network
 retains approximately 96\,\% (66\,\%) of downstream candidates and
 97\,\% (82\,\%) of long candidates for the selection at high (low)
 \qsq.
