\section{Systematic uncertainties on angular observables}
\subsection{Angular correlations}
 To derive Eqs.~\ref{eq:afbLTh} and \ref{eq:afbBTh}, a uniform
 efficiency is assumed. However, non-uniformity is observed,
 especially as a function of $\cos\theta_h$ (see
 Fig.~\ref{fig:AngEff}).  Therefore, while integrating over the full
 angular distribution, terms that would cancel with constant
 efficiency may remain and generate a bias in the measurement of these
 observables. To assess the impact of this potential bias, simulated
 experiments are generated in a two-dimensional ($\cos
 \theta_\ell$,$\cos \theta_h$) space according to the theoretical
 distribution multiplied by a two-dimensional efficiency histogram.
 Projections are then made and are fitted with the default
 one-dimensional efficiency functions. The average deviations from the
 generated parameters are assigned as systematic uncertainties. The
 magnitudes of these are found to be $-0.032$ for $A_{\rm FB}^\ell$,
 $0.013$ for $A_{\rm FB}^h$ and $0.028$ for $f_{\rm L}$, independently
 of \qsq.  In most \qsq intervals this is the dominant source of
 systematic uncertainty.


\subsection{Resolution}
 Resolution effects may induce an asymmetric migration of events
 between bins and therefore generate a bias in the measured value of
 the forward-backward asymmetries.  To study this systematic effect, a
 map of the angular resolution function is created using simulated
 events by comparing reconstructed quantities with those in the
 absence of resolution effects.  Simulated experiments are then
 generated according to the measured angular distributions and smeared
 using the angular resolution maps.  The simulated events, before and
 after smearing by the angular resolution function, are fitted with
 the default PDF.  The average deviations from the default values are
 assigned as systematic uncertainties.  These are larger for the
 $A_{\rm FB}^h$ observable because the resolution is poorer for
 $\cos\theta_h$ and the distribution is more asymmetric, yielding a
 net migration effect.  The uncertainties from this source are in the
 ranges $[0.011,0.016]$ for $A_{\rm FB}^\ell$, $[-0.001,-0.007]$ for
 $A_{\rm FB}^h$ and $[0.002,0.008]$ for $f_{\rm L}$, depending on
 \qsq.

\subsection{Angular acceptance}
 An imprecise determination of the efficiency due to data-simulation
 discrepancies could bias the $A_{\rm FB}$ measurement.  To estimate
 the potential impact arising from this source, the kinematic
 reweighting described in Sec.~\ref{sec:systematics_eff} is removed
 from the simulation.  Simulated samples are fitted using the same
 theoretical PDF multiplied by the efficiency function obtained with
 and without kinematical reweighting.  The average biases evaluated
 from simulated experiments are assigned as systematic uncertainties.
 These are larger for sparsely populated \qsq intervals and vary in
 the intervals $[0.009,0.016]$ for $A_{\rm FB}^\ell$, $[0.001,0.007]$
 for $A_{\rm FB}^h$ and $[0.002,0.044]$ for $f_{\rm L}$, depending on
 \qsq.

 The effect of the limited knowledge of the \Lb polarisation is
 investigated by varying the polarisation within its measured
 uncertainties, in the same way as for the branching fraction
 measurement. No significant effect is found and therefore no
 contribution is assigned.

\subsection{Background parametrisation}
\label{sec:bkgShapeSys}
As there is ambiguity in the choice of parametrisation for the
background model, in particular for regions with low statistical
significance in data, simulated experiments are generated from a PDF
corresponding to the best fit to data, for each \qsq interval. Each
simulated sample is fitted with two models: the nominal fit model,
consisting of the product of a linear function and the signal
efficiency, and an alternative model formed from a constant function
multiplied by the efficiency shape.  The average deviations are taken
as systematic uncertainties.  These are in the ranges $[0.003,0.045]$
for $A_{\rm FB}^\ell$, $[0.017,0.053]$ for $A_{\rm FB}^h$ and
$[0.014,0.049]$ for $f_{\rm L}$, depending on \qsq.

