\section{Angular observables}
\label{ap:angObservables}

In the analysis we measure two angular observables using fit to one dimensional angular projections
of muon and proton helicity angle. Here we list couple of details on how those are obtained, based
on Ref.~\cite{Gutsche:2013pp}. 

For unpolarized production, most general angular distribution can be written as 
\begin{eqnarray}
\label{bjoint3}
W(\theta_\ell,\theta_\Lambda,\chi)  &\propto& 
\sum_{\lambda_1,\lambda_{2},\lambda_j,\lambda'_j,J,J',m,m',\lambda_{\Lambda},
\lambda'_{\Lambda},\lambda_{p}} 
h^{m}_{\lambda_1\lambda_2}(J)h^{m'}_{\lambda_1\lambda_2}(J')
e^{i(\lambda_{j}-\lambda'_{j})\chi}
\nonumber\\ 
&\times&
\delta_{\lambda_{j}-\lambda_{\Lambda},\lambda'_{j}-\lambda'_{\Lambda}}
\delta_{JJ'}
d^J_{\lambda_j,\lambda_1-\lambda_{2}}(\theta_\ell)
d^{J'}_{\lambda'_j,\lambda_1-\lambda_{2}}(\theta_\ell)
H^{m}_{\lambda_{\Lambda}\lambda_{j}}(J)
H^{m'\dagger}_{\lambda'_{\Lambda}\lambda'_{j}}(J')
\nonumber \\
&\times& 
d^{1/2}_{\lambda_{\Lambda}\lambda_{p}}(\theta_\Lambda)
d^{1/2}_{\lambda'_{\Lambda}\lambda_{p}}(\theta_\Lambda)
h^{B}_{\lambda_{p}0}h^{B\,\dagger}_{\lambda_{p}0}\,,
\end{eqnarray}
where $\theta_\ell$, $\theta_\Lambda$ correspond to lepton and proton helicity angle, $\chi$ is angle between
dimuon and \Lz decay planes (for unpolarized production we are sensitive only to difference in
azimuthal angles), $d^J_{i,j}$ are Wigner d-functions and $h$, $h^B$ and $H$ are helicity amplitudes
for virtual dimuon, \Lz and \Lb decays. The sum runs over all possible helicities with dimuon being
allowed in spin 0 and 1 states ($J$ and $J'$). Index $m$/$m'$ runs over vector ($1$ in following)
and axial-vector ($2$) current contributions.

Substituting for dimuon amplitudes and by integrating over three angles, one can obtain differential branching fraction (eq. $11$ in
Ref.~\cite{Gutsche:2013pp})
\begin{eqnarray}
\label{bjoint00}
\frac{d\Gamma(\Lambda_b \to \Lambda \,\ell^{+}\ell^{-})}{d q^2}=
\frac{v^{2}}{2}\cdot\bigg( U^{11+22} + L^{11+22} \bigg)
+\frac{2m_\ell^{2}}{q^{2}}\cdot\frac{3}{2}\cdot
\bigg( U^{11} + L^{11} + S^{22} \bigg)\,, 
\end{eqnarray}
where we have adopted the notations
$d\Gamma_X^{mm'}/d q^2=X^{mm'}$ and $X^{11+22}=X^{11}+X^{22}$.
The bilinear expressions $H^{mm'}_{X}$ ($X=U,L,S$) are defined by
\begin{equation}
\label{helcom2}
\qquad
\begin{array}{lr}
\mbox{$ H^{mm'}_U = 
{\rm Re}(H^{m}_{\frac{1}{2}1} H^{\dagger m'}_{\frac{1}{2}1}) + 
{\rm Re}(H^{m}_{-\frac{1}{2}-1} H^{\dagger m'}_{-\frac{1}{2}-1}) $}  & 
\hfill\mbox{ \rm unpolarized-transverse}\,, 
\\
\mbox{$ H^{mm'}_L = 
{\rm Re}(H^{m}_{\frac{1}{2}0} H^{\dagger m'}_{\frac{1}{2}0}) + 
{\rm Re}(H^{m}_{-\frac{1}{2}0} H^{\dagger m'}_{-\frac{1}{2}0}) $}     & 
\hfill\mbox{ \rm longitudinal}\,, 
\\
\mbox{$ H^{mm'}_S =  
{\rm Re}(H^{m}_{\frac{1}{2}t}H^{\dagger m'}_{\frac{1}{2}t}) +  
{\rm Re}(H^{m}_{-\frac{1}{2}t}H^{\dagger m'}_{-\frac{1}{2}t})$} &  
\hfill\mbox{ \rm scalar}\,. 
\end{array}\\
\end{equation}
The lepton helicity angle $\theta_\ell$ distribution is given in Ref.~\cite{Gutsche:2013pp} by eq. 15
as
\begin{eqnarray}
\label{costheta2}
\frac{d\Gamma(\Lambda_{b}\to \Lambda \,\ell^{+}\ell^{-})}{dq^2d\cos\theta_\ell} 
&=&\,
v^{2}\cdot\bigg[\frac{3}{8}\,(1+\cos^2\theta_\ell)\cdot
\frac{1}{2} U^{11+22}  
\ + \ \frac{3}{4}\,\sin^2\theta_\ell\cdot
\frac{1}{2} L^{11+22} \bigg]\label{distr2}\nonumber\\[2mm]
&-&\,v \cdot\frac{3}{4}\cos\theta_\ell\cdot P^{12} 
\ + \ \frac{2m_{\ell}^{2}}{q^{2}}\cdot \frac{3}{4}\cdot
\bigg[ U^{11}+ L^{11} + S^{22} \bigg]\,.
\end{eqnarray}
Here in addition to amplitudes combinations entering differential branching fraction one more
parity--odd term contributes
\begin{equation}
\label{helcom2a}
\qquad\begin{array}{lr}
\mbox{$H^{mm'}_P =  
{\rm Re}(H^{m}_{\frac{1}{2}1}H^{\dagger m'}_{\frac{1}{2}1})- 
{\rm Re}(H^{m}_{-\frac{1}{2}-1}H^{\dagger m'}_{-\frac{1}{2}-1})$} & 
\hfill\mbox{ \rm parity--odd}
\\
\end{array}\\
\end{equation}
Using this authors of Ref.~\cite{Gutsche:2013pp} give lepton-side forward-backward asymmetry as
\begin{equation}
A_{FB}^{\ell}(q^{2}) 
= - \frac{3}{2}\,\frac{v\cdot P^{12}}
{v^{2}\cdot\big(\,U^{11+22}
+L^{11+22}\big)+\frac{2m_{\ell}^{2}}{q^{2}}\cdot 3 \cdot
\big(U^{11}+L^{11}
+S^{22}\,\big) }\,.
\label{eq:afbDef}
\end{equation}
Using results from eqs.~\ref{eq:afbDef} and \ref{bjoint00} one can rewrite eq.~\ref{costheta2} as
\begin{align}
\frac{d\Gamma(\Lambda_{b}\to \Lambda \,\ell^{+}\ell^{-})}{dq^2d\cos\theta_\ell}=&
\frac{3}{8}\frac{d\Gamma}{dq^2}\left(1+\cos^2\theta_\ell\right)U^{11+22}+
\frac{d\Gamma}{dq^2}A_{FB}^\ell\cos\theta_\ell+
\frac{3}{8}\sin^2\theta_\ell v^2\left(L^{11+22}\right) \nonumber \\
&\left(U^{11}+L^{11}+S^{22}\right)\frac{3m_\ell^2}{q^2}\left(\frac{1}{8}-\frac{3}{8}\cos^2\theta_\ell\right)
\end{align}
 
For analysis, in analogy with $\Bz\to\Kstarz\mumu$ we neglect muon mass (good approximation except of
very low $q^2$). In such case, differential rate and angular distribution simplify to
\begin{eqnarray}
\frac{d\Gamma(\Lambda_b \to \Lambda \,\ell^{+}\ell^{-})}{d q^2}=
\frac{v^{2}}{2}\cdot\bigg( U^{11+22} + L^{11+22} \bigg)
\end{eqnarray}
and
\begin{align}
\frac{d\Gamma(\Lambda_{b}\to \Lambda \,\ell^{+}\ell^{-})}{dq^2d\cos\theta_\ell}=&
\frac{3}{8}\frac{d\Gamma}{dq^2}\left(1+\cos^2\theta_\ell\right)U^{11+22}+
\frac{d\Gamma}{dq^2}A_{FB}^\ell\cos\theta_\ell+
\frac{3}{8}\sin^2\theta_\ell v^2\left(L^{11+22}\right). 
\end{align}
This can be further simplified to
\begin{align}
\frac{d\Gamma(\Lambda_{b}\to \Lambda \,\ell^{+}\ell^{-})}{dq^2d\cos\theta_\ell}=
\frac{d\Gamma}{dq^2}&\left[
\frac{3}{8}\left(1+\cos^2\theta_\ell\right)\frac{U^{11+22}}{U^{11+22}+L^{11+22}}+A_{FB}^\ell\cos\theta_\ell\right . +
 \nonumber \\
& \left. \frac{3}{4}\sin^2\theta_\ell\frac{L^{11+22}}{U^{11+22}+L^{11+22}}\right].
\end{align}
The amplitude combination in the last term can be viewed as ratio between longitudinal and sum of
longitudinal and unpolarized transverse contributions and therefore we define longitudinal fraction 
\begin{equation}
f_L=\frac{L^{11+22}}{U^{11+22}+L^{11+22}},
\end{equation}
which leads to the distribution we use in the analysis
\begin{align}
\frac{d\Gamma(\Lambda_{b}\to \Lambda \,\ell^{+}\ell^{-})}{dq^2d\cos\theta_\ell}=
\frac{d\Gamma}{dq^2}&\left[  \frac{3}{8}\left(1+\cos^2\theta_\ell\right)(1-f_L)+A_{FB}^\ell\cos\theta_\ell +
   \frac{3}{4}\sin^2\theta_\ell f_L\right]. 
\end{align}

In Ref.~\cite{Gutsche:2013pp} gives also forward-backward asymmetry on hadron side. While in the
decay $\Bz\to\Kstarz\mumu$ this is zero, in case of $\Lb\to\Lz\mumu$ this is in principle non-zero
thanks to non-zero \Lz decay asymmetry parameter $\alpha_{\Lz}$. The proton helicity distribution is
given as
\begin{equation}
\label{pz2}
\frac{d\Gamma(\Lambda_{b}\to \Lambda(\to p \pi^{-})\ell^{+}\ell^{-})}
     {dq^2\,d\cos\theta_\Lambda} 
={\rm Br}(\Lambda \to p\pi^{-})
 \frac{d\Gamma(\Lambda_b \to \Lambda\, \ell^{+}\ell^{-})}{d q^2}
\Big(\frac{1}{2}+A_{FB}^h\cos\theta_\Lambda\Big) \,, 
\end{equation}
where 
\begin{equation}
A_{FB}^h=\frac{1}{2}\alpha_{\Lz}P_z^\Lz(q^2).
\end{equation} 
The quantity $P_z^\Lz(q^2)$ is defined as (eq.~$21$ in Ref.~\cite{Gutsche:2013pp})
\begin{equation}
\label{pz1}
P_{z}^{\Lambda}=\frac{
v^{2}\cdot\big(\,P^{11+22}
+L_{P}^{11+22}\big)+\frac{2m_{\ell}^{2}}{q^{2}}\cdot 3 \cdot 
\big(P^{11}+L_{P}^{11}
+S_{P}^{22}\,\big)
}
{v^{2}\cdot\big(\,U^{11+22}
+L^{11+22}\big)+\frac{2m_{\ell}^{2}}{q^{2}}\cdot 3 \cdot
\big(U^{11}+L^{11}
+S^{22}\,\big) 
} \, 
\end{equation}
where two parity-violating combinations of amplitudes
\begin{equation}
\label{helcom3}
\qquad\begin{array}{lr}
\mbox{$H^{mm'}_{L_{P}}=
{\rm Re}
(H^{m}_{\frac{1}{2}\,0}H^{\dagger m'}_{\frac{1}{2}\,0}-H^{m}_{-\frac{1}{2}\,0}
H^{\dagger m'}_{-\frac{1}{2}\,0})$} & 
\hfill\mbox{ \rm longitudinal--polarized}\,, 
\\
\mbox{$H^{mm'}_{S_{P}}=
{\rm Re}
(H^{m}_{\frac{1}{2}\,t}H^{\dagger m'}_{\frac{1}{2}\,t}-H^{m}_{-\frac{1}{2}\,t}
H^{\dagger m'}_{-\frac{1}{2}\,t})$}
 & 
\hfill\mbox{ \rm scalar--polarized}\,, 
\end{array}\\
\end{equation}
enter in addition to the ones we already discussed.

While lepton side observables are well known from meson decays, hadron side is unexplored. To get
feeling how sensitive $A_{FB}^h$ is to new physics, we plot together three different cases: first,
SM using predictions and numerical values from Ref.~\cite{Gutsche:2013pp}, scenario in which Wilson
coefficients are same as in the SM, just sign of the $C_7$ is flipped and finally modifying $C_7$,
$C_9$ and $C_{10}$ by amount compatible with fit to meson data including $P_5'$ in
Ref.~\cite{Descotes-Genon:2013wba}.
Resulting dependence of $P_z^\Lz$ is shown in Fig.~\ref{fig:PzLpred}.
%
\begin{figure}
\centering
\includegraphics[width=0.7\textwidth]{Lmumu/figs/AFBhPrediction}
\caption{Expectation for $P_z^\Lz$ which is proportional to $A_{FB}^h$ in the SM (black line), SM
like scenario with flipped sign on $C_7$ (red line) and scenario with all three Wilson coefficients
modified according to Ref.~\cite{Descotes-Genon:2013wba} (blue line). There is little difference
between those tree scenarious.}
\label{fig:PzLpred}
\end{figure}
%
As can be seen, there is little difference between three scenarious. We also tried to investigate
analytically. In that case we put lepton mass to zero and neglect $C_7$ terms, which are small at
high $q^2$ where we have most of the signal. In such case, both numerator and denumerator in
eq.~\ref{pz1} are proportional to $(|C_9|^2+|C_{10}|^2)$ and thus dependence on Wilson coefficients
drops out in first order. So only difference can come from $C_7$ and its interference with other two Wilson
coefficients and thus effects are expected to be small. From this we conclude that main help of this
variable is in constraining form factors in the studied decay. 

To incorporate effects of production polarization, we modify eq.~\ref{bjoint3} to  
\begin{eqnarray}
\label{bjoint4}
W(\theta,\theta_\ell,\phi_\ell,\theta_\Lambda,\phi_\Lambda)  &\propto& 
\sum_{\lambda_1,\lambda_{2},\lambda_j,\lambda'_j,J,J',m,m',\lambda_{\Lambda},
\lambda'_{\Lambda},\lambda_{p}} 
h^{m}_{\lambda_1\lambda_2}(J)h^{m'}_{\lambda_1\lambda_2}(J')
\rho_{\lambda_{j}-\lambda_{\Lambda},\lambda'_{j}-\lambda'_{\Lambda}}(\theta)
(-1)^{J+J'}
\nonumber\\ 
&\times&D^J_{\lambda_j,\lambda_1-\lambda_{2}}(\theta_\ell,\phi_\ell)
D^{J'}_{\lambda'_j,\lambda_1-\lambda_{2}}(\theta_\ell,\phi_\ell)
H^{m}_{\lambda_{\Lambda}\lambda_{j}}(J)
H^{m'\dagger}_{\lambda'_{\Lambda}\lambda'_{j}}(J')
\nonumber \\
&\times& 
D^{1/2}_{\lambda_{\Lambda}\lambda_{p}}(\theta_\Lambda,\phi_\Lambda)
D^{1/2}_{\lambda'_{\Lambda}\lambda_{p}}(\theta_\Lambda,\phi_\Lambda)
h^{B}_{\lambda_{p}0}h^{B\,\dagger}_{\lambda_{p}0}\,,
\end{eqnarray}
where
\begin{equation}
D^{J}_{m,n}(\theta_i,\phi_i)=e^{-i(m-n)\phi_i}d^J_{m,n}(\theta_i).
\end{equation}
In modified version, angle $\theta$ comes with production polarization through spin-density matrix
in eq.~\ref{eq:spinDensity}, while angle between decay planes $\chi$ is changed to the polar angles of proton and
positive muon. Integrating this expression over $\theta_\Lambda$, $\phi_\Lambda$, $\phi_\ell$ and $\theta$ yields
same distribution as in unpolarized case (eq.~\ref{costheta2}). Therefore, in case of uniform
efficiency, lepton side forward-backward asymmetry $A_{FB}^\ell$ is unaffected by production
polarization. To estimate effect of production polarization, we also derive two-dimensional
distribution in $\theta$ and $\theta_\ell$, which in limit of massless leptons becomes (up to constant
multiplicative factor)
%
\begin{align}
\frac{d\Gamma(\Lambda_{b}\to \Lambda \,\ell^{+}\ell^{-})}{dq^2d(\cos\theta) d(\cos\theta_\ell)}=&
\frac{d\Gamma}{dq^2}\left\{  \frac{3}{8}\left(1+\cos^2\theta_\ell\right)(1-f_L)+A_{FB}^\ell\cos\theta_\ell +
   \frac{3}{4}\sin^2\theta_\ell f_L+\right. \nonumber \\
&P_b\cos(\theta)\left[ -\frac{3}{4}\sin(\theta_\ell)^2O_{Lp}+
  \frac{3}{8}\left(1+\cos(\theta_\ell)^2\right)O_P\right. \nonumber \\
&\left.\left.-\frac{3}{8}\cos(\theta_\ell)O_{U12} \right]\right\}\,,
\label{eq:lepton2D}
\end{align}
where we have defined
\begin{align}
O_{Lp}=&\frac{L_P^{11}+L_P^{22}}{U^{11+22}+L^{11+22}}, \nonumber \\
O_P=&\frac{P^{11}+P^{22}}{U^{11+22}+L^{11+22}}, \nonumber \\
O_{U12}=&\frac{U^{12}}{U^{11+22}+L^{11+22}}. \nonumber
\end{align}
In the massless approximation we are using two of these quantities are related to hadron side
forward-backward asymmetry as
\begin{equation}
\frac{1}{2}\alpha_\Lambda \left(O_P+O_{Lp}\right)=A_{FB}^h\,.
\end{equation}

Following the same steps as in case of lepton side $A_{FB}^\ell$, after integrating over four angles we
find that hadron side $A_{FB}^h$ is also unaffected by production polarization in case of uniform
efficiency. The two dimensional distribution in $\theta$ and $\theta_\Lambda$ has form (again omitting some
numerical constants)
\begin{align}
\frac{d\Gamma(\Lambda_{b}\to \Lambda \,\ell^{+}\ell^{-})}{dq^2d(\cos\theta) d(\cos\theta_B)}=
\frac{d\Gamma}{dq^2}&\left[1+2A_{FB}^h\cos\theta_B+P_b\left(O_P-O_{Lp}\right)\cos\theta\right.\nonumber \\
&\left.+\alpha_\Lz P_b\left(1-2f_L\right)\cos\theta\cos\theta_B\right].
\label{eq:hadron2D}
\end{align}

It should be noted that fact that in case of uniform efficiency both angular observables are
unaffected by production polarization comes from the fact that terms proportional to $\sin\theta$
cancel out when integrating over $\phi_\ell$ and $\phi_\Lambda$ and remaining terms containing production
polarization go as $\cos\theta$, which after integrating over it yields zero.

In order to use two-dimensional distributions, we need expectation for three additional observables,
which do not enter one-dimensional distributions. To obtain expactation, we use form factors and
numerical inputs from Ref.~\cite{Gutsche:2013pp}. In this calculation we use massless lepton limit
(all definitions would be slightly altered if lepton mass is taken into account) and compared to the
Ref.~\cite{Gutsche:2013pp} we turn off long-distance contributions by setting dileptonic decay
widths of \jpsi and $\psi(2S)$ to zero. In order to obtain binned values, we need to integrate
corresponding amplitudes over \qsq interval and this is done separately for those in the numerator
and denominator rather than integrating ratio of amplitudes. Results are in tables
\ref{tab:obsGutsche1}. Alternatively we calculate also expectation using lattice QCD form factors
from Ref.~\cite{Detmold:2012vy} by substituting those instead of form factors from Ref.~\cite{Gutsche:2013pp}.
These results are in table \ref{tab:lQCD1}.
%
\begin{table}
\begin{center}
\begin{tabular}{lcccccc}\hline
\qsq [$GeV^2/c^2$]  & $A_{FB}^\ell$ & $P_z^\Lz$  & $f_L$   & $O_P$  & $O_{Lp}$ & $O_{U12}$ \\ \hline
0.1 -- 2.0          &  0.082     & -0.9998    & 0.537   & -0.463 & -0.537   &  0.055  \\ 
2.0 -- 4.0          & -0.032     & -0.9996    & 0.858   & -0.142 & -0.857   & -0.021  \\ 
4.0 -- 6.0          & -0.153     & -0.9991    & 0.752   & -0.247 & -0.752   & -0.102  \\ 
11.0 -- 12.5        & -0.348     & -0.9834    & 0.508   & -0.478 & -0.505   & -0.239  \\ 
15.0 -- 16.0        & -0.384     & -0.9374    & 0.428   & -0.524 & -0.413   & -0.280  \\ 
16.0 -- 18.0        & -0.377     & -0.8807    & 0.399   & -0.513 & -0.368   & -0.294  \\ 
18.0 -- 20.0        & -0.297     & -0.6640    & 0.361   & -0.404 & -0.260   & -0.314  \\ \hline 
1.0 -- 6.0          & -0.040     & -0.9994    & 0.830   & -0.170 & -0.830   & -0.027  \\ 
15.0 -- 20.0        & -0.339     & -0.7830    & 0.385   & -0.461 & -0.322   & -0.302  \\ \hline
\end{tabular}
\end{center}
\caption{Prediction for angular observables entering two-dimensional angular distributions.
Prediction is based on covariant quark model form factors from Ref.~\cite{Gutsche:2013pp}.}
\label{tab:obsGutsche1}
\end{table}
%
%
\begin{table}
\begin{center}
\begin{tabular}{lcccccc}\hline
\qsq [$GeV^2/c^2$]  & $A_{FB}^\ell$ & $P_z^\Lz$  & $f_L$   & $O_P$  & $O_{Lp}$ & $O_{U12}$ \\ \hline
0.1 -- 2.0          &  0.023     & -0.963    & 0.904   & -0.096 & -0.867   &  0.015  \\ 
2.0 -- 4.0          & -0.085     & -0.962    & 0.870   & -0.128 & -0.834   & -0.058  \\ 
4.0 -- 6.0          & -0.163     & -0.962    & 0.771   & -0.224 & -0.739   & -0.111  \\ 
11.0 -- 12.5        & -0.316     & -0.934    & 0.541   & -0.427 & -0.507   & -0.227  \\ 
15.0 -- 16.0        & -0.346     & -0.866    & 0.450   & -0.468 & -0.398   & -0.271  \\ 
16.0 -- 18.0        & -0.336     & -0.799    & 0.416   & -0.455 & -0.344   & -0.288  \\ 
18.0 -- 20.0        & -0.260     & -0.585    & 0.368   & -0.354 & -0.231   & -0.311  \\ \hline 
1.0 -- 6.0          & -0.086     & -0.962    & 0.865   & -0.132 & -0.829   & -0.059  \\ 
15.0 -- 20.0        & -0.306     & -0.721    & 0.402   & -0.415 & -0.306   & -0.295  \\ \hline
\end{tabular}
\end{center}
\caption{Prediction for angular observables entering two-dimensional angular distributions.
Prediction is based on lattice QCD form factors from Ref.~\cite{Detmold:2012vy}.}
\label{tab:lQCD1}
\end{table}

For completeness, two-dimensional distribution in $\cos\theta_L$-$\cos\theta_B$ has form
\begin{align}
\frac{d\Gamma(\Lambda_{b}\to \Lambda \,\ell^{+}\ell^{-})}{dq^2d(\cos\theta_B) d(\cos\theta_L)}=&
\frac{3}{8}+\frac{6}{16}\cos^2\theta_L(1-f_L)-\frac{3}{16}\cos^2\theta_L f_L
+A_{FB}^l\cos\theta_L+ \nonumber \\
& \left(\frac{3}{2}A_{FB}^h-\frac{3}{8}\alpha_\Lz O_P\right)\cos\theta_B
-\frac{3}{2}A_{FB}^h\cos^2\theta_L\cos\theta_B-\frac{3}{16}f_L+ \nonumber \\
& \frac{9}{16}f_L\sin^2\theta_L+\frac{9}{8}\alpha_\Lz \cos^2\theta_L\cos\theta_B O_P- \nonumber \\
& \frac{3}{2}\alpha_\Lz \cos\theta_L\cos\theta_B O_{U12}.
\label{eq:2DcosThetaLandB}
\end{align}
It does not need any additional inputs compared to previous two-dimensional distributions.

\clearpage
