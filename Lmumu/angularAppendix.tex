\chapter{Decay models}

\section{$\Lb\ra\Lz\mumu$ distribution}
\label{ap:LbLmumuAngular}

%Our more realistic model is based on Ref.~\cite{Aliev:2005np}, which is for standard model same
%model as is implemented in EvtGen. We start from expressions for amplitudes for given helicity state,
%which are given as
%%
%% 
%\begin{eqnarray}
%%1{++}
%{\mathcal{M}}_{+1/2}^{++} &=& 2 m_{\mu} \sin\theta \Big( H_{+1/2,+1}^{(1)} +
%H_{+1/2,+1}^{(2)}\Big) + 2 m_{\mu} \cos\theta \Big( H_{+1/2,0}^{(1)} +
%H_{+1/2,0}^{(2)}\Big) \nonumber \\
%&+& 2 m_{\mu} \Big( H_{+1/2,t}^{(1)} -
%H_{+1/2,t}^{(2)}\Big)~, \nonumber \\
%%2{+-}
%{\cal M}_{+1/2}^{+-} &=& - \sqrt{q^2} (1-\cos\theta) \Big[
%(1-v) H_{+1/2,+1}^{(1)} + (1+v) H_{+1/2,+1}^{(2)}\Big] -
%\sqrt{q^2} \sin\theta  \Big[(1-v) H_{+1/2,0}^{(1)} \nonumber \\
%&+& (1+v) H_{+1/2,0}^{(2)}\Big]~, \nonumber \\
%%3{-+}
%{\cal M}_{+1/2}^{-+} &=& \sqrt{q^2} (1+\cos\theta) \Big[
%(1+v) H_{+1/2,+1}^{(1)} + (1-v) H_{+1/2,+1}^{(2)}\Big] -
%\sqrt{q^2} \sin\theta  \Big[(1+v) H_{+1/2,0}^{(1)} \nonumber \\
%&+& (1-v) H_{+1/2,0}^{(2)}\Big]~, \nonumber \\
%%4{--}
%{\cal M}_{+1/2}^{--} &=& - 2 m_{\mu} \sin\theta \Big(
%H_{+1/2,+1}^{(1)} +
%H_{+1/2,+1}^{(2)}\Big) - 2 m_{\mu} \cos\theta \Big( H_{+1/2,0}^{(1)} +
%H_{+1/2,0}^{(2)}\Big) \nonumber \\
%&+& 2 m_{\mu} \Big( H_{+1/2,t}^{(1)} -
%H_{+1/2,t}^{(2)}\Big)~, \nonumber \\
%%5{++}
%{\cal M}_{-1/2}^{++} &=& - 2 m_{\mu} \sin\theta \Big(
%H_{-1/2,-1}^{(1)} +
%H_{-1/2,-1}^{(2)}\Big) + 2 m_{\mu} \cos\theta \Big( H_{-1/2,0}^{(1)} +
%H_{-1/2,0}^{(2)}\Big) \nonumber \\
%&+& 2 m_{\mu} \Big( H_{-1/2,t}^{(1)} -
%H_{-1/2,t}^{(2)}\Big)~, \nonumber \\
%%6{+-}
%{\cal M}_{-1/2}^{+-} &=& - \sqrt{q^2} (1+\cos\theta) \Big[
%(1-v) H_{-1/2,-1}^{(1)} + (1+v) H_{-1/2,-1}^{(2)}\Big] -
%\sqrt{q^2} \sin\theta  \Big[(1-v) H_{-1/2,0}^{(1)} \nonumber \\
%&+& (1+v) H_{-1/2,0}^{(2)}\Big]~, \nonumber \\
%%7{-+}
%{\cal M}_{-1/2}^{-+} &=& \sqrt{q^2} (1-\cos\theta) \Big[
%(1+v) H_{-1/2,-1}^{(1)} + (1-v) H_{-1/2,-1}^{(2)}\Big] -
%\sqrt{q^2} \sin\theta  \Big[(1+v) H_{-1/2,0}^{(1)} \nonumber \\
%&+& (1-v) H_{-1/2,0}^{(2)}\Big]~, \nonumber \\
%%8{--}
%{\cal M}_{-1/2}^{--} &=& 2 m_{\mu} \sin\theta \Big( H_{-1/2,-1}^{(1)} +
%H_{-1/2,-1}^{(2)}\Big) - 2 m_{\mu} \cos\theta \Big( H_{-1/2,0}^{(1)} +
%H_{-1/2,0}^{(2)}\Big) \nonumber \\
%&+& 2 m_{\mu} \Big( H_{-1/2,t}^{(1)} -
%H_{-1/2,t}^{(2)}\Big)~.
%\label{eq:LbLmumuHelicity}
%\end{eqnarray}
%%
%Here $M^{\lambda_1\lambda_2}_{\lambda_{\Lb}}$ is amplitude for muon helicities $\lambda_1$ and
%$\lambda_2$ and \Lb helicity $\lambda_{\Lb}$, $H_{\lambda_{\Lb},\lambda}$ contains dependence on
%form factors and Wilson coefficients with $\lambda$ being helicity of "virtual boson" ($\pm1$ and
%$0$ for spin-1 and $t$ for spin-0). The angle $\theta$ is helicity angle of the muon and
%$v=\sqrt{1-4m^2_\mu/\qsq}$.
%In order to obtain full angular distribution, each of the terms
%in each expression above is multiplied by appropriate Wigner function to take into account proton
%helicity angle and angle between \mumu and \Lz decay planes. To obtain angular and \qsq
%distribution, we need to square amplitudes of eq.~\ref{eq:LbLmumuHelicity} with incoherent sum over
%muon helicities. The sum over \Lb helicities is coherent and each term is multiplied by appropriate
%term of spin density matrix in order to take into account production polarization. Spin density
%matrix has form
%%
%\begin{equation}
%P_s=\frac{1}{2}\left(\begin{array}{cc} P_b^{++} & P_b^{+-} \\ P_b^{-+} & P_b^{--}\end{array}\right)=
%\frac{1}{2}\left(\begin{array}{cc} 1+P_b\cos\theta & P_b\sin\theta \\ P_b\sin\theta &
%1-P_b\cos\theta \end{array}\right),
%\end{equation}
%%
%where $P_b$ is production polarization and $\theta$ is angle with same definition as for
%\Lb\ra\jpsi\Lz decay.
%So the full rate can be expressed as
%\begin{equation}
%  |M|^2=\sum_{\lambda_p,\lambda_1,\lambda_2}\left|\sum_{\lambda_{\Lb},\lambda^{'}_{\Lb}}
%M^{\lambda_1\lambda_2}_{\lambda_{\Lb}}P_b^{\lambda_{\Lb}\lambda^{'}_{\Lb}}M^{\lambda_1\lambda_2}_{\lambda^{'}_{\Lb}}\right|^2.
%\end{equation}
%For $H_{\lambda_{\Lb},\lambda}$ expression we refer reader to the Ref.~\cite{Aliev:2005np}.
%For calculation, we use Wilson coefficients used in EvtGen decay model Lb2Lll,
%which are based on Refs.~\cite{Buras:1994dj,Buras:1993xp} with input values corresponding to world
%averages in 2004 \cite{PDBook2004}. Original HQET form factors of Ref.~\cite{Aliev:2005np} are used
%in our evaluations.

The \qsq and angular dependancies of the $\Lb\to\Lz\mumu$ decays are modelled based on
Ref.~\cite{Gutsche:2013pp}, where the angular distribution for unpolarised \Lb production
is defined as
\begin{eqnarray}
W(\theta\ell,\theta_{B},\chi)  &\propto& 
\sum_{\lambda_1,\lambda_{2},\lambda_j,\lambda'_j,J,J',m,m',\lambda_{\Lambda},
\lambda'_{\Lambda},\lambda_{p}} 
h^{m}_{\lambda_1\lambda_2}(J)h^{m'}_{\lambda_1\lambda_2}(J')
e^{i(\lambda_{j}-\lambda'_{j})\chi}
\nonumber\\ 
&\times&
\delta_{\lambda_{j}-\lambda_{\Lambda},\lambda'_{j}-\lambda'_{\Lambda}}
\delta_{JJ'}
d^J_{\lambda_j,\lambda_1-\lambda_{2}}(\theta\ell)
d^{J'}_{\lambda'_j,\lambda_1-\lambda_{2}}(\theta\ell)
H^{m}_{\lambda_{\Lambda}\lambda_{j}}(J)
H^{m'\dagger}_{\lambda'_{\Lambda}\lambda'_{j}}(J')
\nonumber \\
&\times& 
d^{1/2}_{\lambda_{\Lambda}\lambda_{p}}(\theta_{B})
d^{1/2}_{\lambda'_{\Lambda}\lambda_{p}}(\theta_{B})
h^{B}_{\lambda_{p}0}h^{B\,\dagger}_{\lambda_{p}0}\,.
\end{eqnarray}
In this formula $\theta_\ell$ and $\theta_B$ correspond to the lepton and proton helicity angles, $\chi$ is angle between
dimuon and \Lz decay planes (for unpolarised production we are sensitive only to difference in
azimuthal angles), $d^J_{i,j}$ are Wigner d-functions and $h$, $h^B$ and $H$ are helicity amplitudes
for virtual dimuon, \Lz and \Lb decays. The sum runs over all possible helicities with the dimuon being
allowed in spin 0 and 1 states ($J$ and $J'$). The $m$ and $m'$ indices run over the vector %($1$ in following)
and axial-vector %($2$)
current contributions.

The production polarisation is introduced by removing
$e^{i(\lambda_{j}-\lambda'_{j})\chi}$ from the expression, swapping small Wigner d-functions
$d^J_{i,j}$ to the corresponding capital ones $D^J_{i,j}$ which are related as
\begin{equation}
D^J_{i,j}(\theta,\phi)=d^J_{i,j}(\theta)e^{i\phi(i-j)}
\end{equation}
and substitute spin density matrix for
$\delta_{\lambda_{j}-\lambda_{\Lambda},\lambda'_{j}-\lambda'_{\Lambda}}\delta_{JJ'}$. The spin density matrix
itself is given by
\begin{equation}
\label{eq:spinDensity}
\rho_{\lambda_{j}-\lambda_{\Lambda},\lambda'_{j}-\lambda'_{\Lambda}}=
\frac{1}{2}\left(\begin{array}{cc} 1+P_b\cos\theta & P_b\sin\theta \\ P_b\sin\theta &
1-P_b\cos\theta \end{array}\right).
\end{equation}
Those changes lead to the formula
\begin{eqnarray}
W(\theta\ell,\theta_{B},\chi)  &\propto& 
\sum_{\lambda_1,\lambda_{2},\lambda_j,\lambda'_j,J,J',m,m',\lambda_{\Lambda},
\lambda'_{\Lambda},\lambda_{p}} 
h^{m}_{\lambda_1\lambda_2}(J)h^{m'}_{\lambda_1\lambda_2}(J')
\nonumber\\ 
&\times&
\rho_{\lambda_{j}-\lambda_{\Lambda},\lambda'_{j}-\lambda'_{\Lambda}}
D^J_{\lambda_j,\lambda_1-\lambda_{2}}(\theta\ell,\phi_L)
D^{J'}_{\lambda'_j,\lambda_1-\lambda_{2}}(\theta\ell,\phi_L)
H^{m}_{\lambda_{\Lambda}\lambda_{j}}(J)
H^{m'\dagger}_{\lambda'_{\Lambda}\lambda'_{j}}(J')
\nonumber \\
&\times& 
D^{1/2}_{\lambda_{\Lambda}\lambda_{p}}(\theta_{B},\phi_B)
D^{1/2}_{\lambda'_{\Lambda}\lambda_{p}}(\theta_{B},\phi_B)
h^{B}_{\lambda_{p}0}h^{B\,\dagger}_{\lambda_{p}0}\,.
\end{eqnarray}
The lepton amplitudes come directly from Ref.~\cite{Gutsche:2013pp}, eq.~3.
%and we refer interested reader directly to source.
The \Lz decay amplitudes are related to the \Lz decay asymmetry parameter as
\begin{equation}
\alpha_{\Lz}\,=\,\frac{|h^{B}_{{\frac{1}{2}}0}|^{2}-
|h^{B}_{-{\frac{1}{2}}0}|^{2}}
{|h^{B}_{{\frac{1}{2}}0}|^{2}
+|h^{B}_{-{\frac{1}{2}}0}|^{2}}\,.
\end{equation}
Finally, the \Lb decay amplitudes receive contributions from vector and axial-vector currents
 and can be written as
\begin{equation}
H^m_{\lambda_2,\lambda_j} = H^{Vm}_{\lambda_2,\lambda_j}
                          - H^{Am}_{\lambda_2,\lambda_j}\,.
\end{equation}
%Amplitudes for some helicities combinations can be related as
%\begin{align}
% &H^{Vm}_{-\lambda_2,-\lambda_j} = H^{Vm}_{\lambda_2,\lambda_j}\,,
%\nonumber\\
% &H^{Am}_{-\lambda_2,-\lambda_j} = - H^{Am}_{\lambda_2,\lambda_j}\,.
%\end{align}
Finally, the remaining amplitudes are expressed in terms of form factors
(Ref.~\cite{Gutsche:2013pp}, eq.~C6) as
\begin{align}
H^{Vm}_{\frac{1}{2} t} &=&
\sqrt{\frac{Q_+}{q^2}} \,
\biggl( M_- \, F_1^{Vm} + \frac{q^2}{M_1} \, F_3^{Vm} \biggr)\,, \nonumber\\
H^{Vm}_{\frac{1}{2} 1} &=& \sqrt{2 Q_-} \,
\biggl( F_1^{Vm} + \frac{M_+}{M_1} \, F_2^{Vm} \biggr)\,, \nonumber\\
H^{Vm}_{\frac{1}{2} 0} &=& \sqrt{\frac{Q_-}{q^2}} \,
\biggl( M_+ \, F_1^{Vm} + \frac{q^2}{M_1} \, F_2^{Vm} \biggr)\,, \nonumber\\
&&\\
H^{Am}_{\frac{1}{2} t} &=& \sqrt{\frac{Q_-}{q^2}} \,
\biggl( M_+ \, F_1^{Am} - \frac{q^2}{M_1} \, F_3^{Am} \biggr)\,, \nonumber\\
H^{Am}_{\frac{1}{2} 1} &=& \sqrt{2 Q_+} \,
\biggl( F_1^{Am} - \frac{M_-}{M_1} \, F_2^{Am} \biggr)\,, \nonumber\\
H^{Am}_{\frac{1}{2} 0} &=& \sqrt{\frac{Q_+}{q^2}} \,
\biggl( M_- \, F_1^{Am}  - \frac{q^2}{M_1} \, F_2^{Am} \biggr)\,, \nonumber
\end{align}
where $M_\pm = M_1 \pm M_2$, $Q_\pm = M_\pm^2 - q^2$. %They contain both Wilson coefficients as well as $q^2$ dependence.
The form factors $F$ are expressed in terms of dimensioneless quantities in
eqs.~C8 and C9 in Ref.~\cite{Gutsche:2013pp}. In our actual implementation form factors calculated in the covariant
quark model~\cite{Gutsche:2013pp} are used and for the numerical values of the Wilson coefficients
Ref.~\cite{Gutsche:2013pp} is used. 

To assess effect of different form factors on efficiency calculations, an alternative set of form 
factors is implemented, based on the LQCD calculation from Ref.~\cite{Detmold:2012vy}.
%We keep same implementation and numerical values of various parameters
%as for Ref.~\cite{Gutsche:2013pp}, just exchange form factors.
The form factors relations are found by comparing eqs.~66 and $68$ in Ref.~\cite{Gutsche:2013pp} to eq.~$51$ in
Ref.~\cite{Detmold:2012vy}. Denoting LQCD form factors by $F_i^L$ and dimensionless covariant quark model ones by
$f_i^{XX}$ we have
\begin{align}
  f_1^V&=c_\gamma(F_1^L+F_2^L), \nonumber \\
  f_2^V&=-2c_\gamma F_2^L, \nonumber \\
  f_3^V&=c_v(F_1^L+F_2^L), \nonumber \\
  f_1^A&=c_\gamma(F_1^L-F_2^L), \nonumber \\
  f_2^A&=-2c_\gamma F_2^L, \nonumber \\
  f_3^A&=-c_v(F_1^L-F_2^L), \nonumber \\
  f_1^{TV}&=c_\sigma F_2^L, \nonumber \\
  f_2^{TV}&=-c_\sigma F_1^L, \nonumber \\
  f_1^{TA}&=c_\sigma F_2^L, \nonumber \\
  f_2^{TA}&=-c_\sigma F_1^L, \nonumber 
\end{align} 
where
\begin{align}
c_\gamma&=1-\frac{\alpha_s(\mu^2)}{\pi}\left[\frac{4}{3}+\ln\left(\frac{\mu}{m_b}\right)\right],\nonumber \\
c_v&=\frac{2}{3}\frac{\alpha_s(\mu^2)}{\pi}, \nonumber \\
c_\sigma&=1-\frac{\alpha_s(\mu^2)}{\pi}\left[\frac{4}{3}+\frac{5}{3}\ln\left(\frac{\mu}{m_b}\right)\right].
\end{align}
In the calculations $\mu=m_b$ is used. For the strong coupling constant, we start from the world average value at
the $Z$ mass, $\alpha_s(m_Z^2)=0.1185 \pm 0.0006$~\cite{PDG2014}, and we translate it to the scale $m_b^2$ by
\begin{equation}
\alpha_s(\mu^2)=\frac{\alpha_s(m_Z^2)}{1+\frac{\alpha_s(m_Z^2)}{12\pi}
\left(33-2n_f\right)\ln\left(\frac{\mu^2}{m_Z^2}\right)},
\end{equation}
where $n_f=5$. The LQCD form factors $F_1^L$ and $F_2^L$ can be then taken directly from Ref.~\cite{Detmold:2012vy} and
plugged into the code implementing the calculation from Ref.~\cite{Gutsche:2013pp}.


\section{Bi-dimensional distribution parameters}
Expectations values for parameters in the bi-dimensional angular distribution for the \Lb\to\Lz\mumu decay
 calculated using form factors and numerical inputs from Ref.~\cite{Gutsche:2013pp}.
 
\begin{table}[h!]
\begin{center}
\begin{tabular}{lcccccc}\hline
\qsq [$GeV^2/c^2$]  & $A_{FB}^\ell$ & $P_z^\Lz$  & $f_L$   & $O_P$  & $O_{Lp}$ & $O_{UVA}$ \\ \hline
0.1 -- 2.0          &  0.082     & -0.9998    & 0.537   & -0.463 & -0.537   &  0.055  \\ 
2.0 -- 4.0          & -0.032     & -0.9996    & 0.858   & -0.142 & -0.857   & -0.021  \\ 
4.0 -- 6.0          & -0.153     & -0.9991    & 0.752   & -0.247 & -0.752   & -0.102  \\ 
V.0 -- VA.5        & -0.348     & -0.9834    & 0.508   & -0.478 & -0.505   & -0.239  \\ 
15.0 -- 16.0        & -0.384     & -0.9374    & 0.428   & -0.524 & -0.413   & -0.280  \\ 
16.0 -- 18.0        & -0.377     & -0.8807    & 0.399   & -0.513 & -0.368   & -0.294  \\ 
18.0 -- 20.0        & -0.297     & -0.6640    & 0.361   & -0.404 & -0.260   & -0.314  \\ \hline 
1.0 -- 6.0          & -0.040     & -0.9994    & 0.830   & -0.170 & -0.830   & -0.027  \\ 
15.0 -- 20.0        & -0.339     & -0.7830    & 0.385   & -0.461 & -0.3A   & -0.302  \\ \hline
\end{tabular}
\end{center}
\caption{Prediction for angular observables entering two-dimensional angular distributions.
Prediction is based on covariant quark model form factors from Ref.~\cite{Gutsche:2013pp}.}
\label{tab:obsGutsche1}
\end{table}



\section{$\Lb\ra\jpsi\Lz$ distribution}
\label{ap:LbJpsiLAngular}

%\newcommand{\jnd}{\mbox{${\frac{1}{2}}$}}
%\newcommand{{\frac{3}{2}}}{\mbox{${\frac{3}{2}}$}}
%\newcommand{\tnsd}{\mbox{${3 \over\sqrt{2}}$}}
%\newcommand{\jnst}{\mbox{${1 \over {(4 \pi)}^{3}}$}}



%\begin{table}
%\begin{tabular}{lccc}
%             \hline\hline
%              i & $f_{1i}$
%           & $f_{2i}$                & $F_{i}$
%                             \\
%             \hline
%              0 & $a_{+}a_{+}^{*}+a_{-}a_{-}^{*}+b_{+}b_{+}^{*}+b_{-}b_{-}^{*}$
%           & 1                       & 1
%                                 \\
%             \hline
%              1 & $a_{+}a_{+}^{*}-a_{-}a_{-}^{*}+b_{+}b_{+}^{*}-b_{-}b_{-}^{*}$
%           & $P_{b}$                 & $\cos\theta$
%                               \\
%              2 & $a_{+}a_{+}^{*}-a_{-}a_{-}^{*}-b_{+}b_{+}^{*}+b_{-}b_{-}^{*}$
%           & $\alpha_{\Lambda}$      & $\cos\theta_{1}$
%                               \\
%              3 & $a_{+}a_{+}^{*}+a_{-}a_{-}^{*}-b_{+}b_{+}^{*}-b_{-}b_{-}^{*}$
%           & $P_{b}\alpha_{\Lambda}$ & $\cos\theta\cos\theta_{1}$
%                             \\
%             4 & $-a_{+}a_{+}^{*}-a_{-}a_{-}^{*}+{\frac{1}{2}} b_{+}b_{+}^{*}+{\frac{1}{2}}
% b_{-}b_{-}^{*}$ & 1                       & $d_{00}^{2}(\theta_{2})$
%                                       \\
%              5 & $-a_{+}a_{+}^{*}+a_{-}a_{-}^{*}+{\frac{1}{2}} b_{+}b_{+}^{*}-{\frac{1}{2}}
% b_{-}b_{-}^{*}$ & $P_{b}$                 & $d_{00}^{2}(\theta_{2})\cos\theta$
%                                 \\
%              6 & $-a_{+}a_{+}^{*}+a_{-}a_{-}^{*}-{\frac{1}{2}} b_{+}b_{+}^{*}+{\frac{1}{2}}
% b_{-}b_{-}^{*}$ & $\alpha_{\Lambda}$      &
% $d_{00}^{2}(\theta_{2})\cos\theta_{1}$                                  \\
%              7 & $-a_{+}a_{+}^{*}-a_{-}a_{-}^{*}-{\frac{1}{2}} b_{+}b_{+}^{*}-{\frac{1}{2}}
% b_{-}b_{-}^{*}$ & $P_{b}\alpha_{\Lambda}$ &
% $d_{00}^{2}(\theta_{2})\cos\theta\cos\theta_{1}$                      \\
%             \hline
%              8 & $-3Re(a_{+}a_{-}^{*})$
% & $P_{b}\alpha_{\Lambda}$ &
% $\sin\theta\sin\theta_{1}\sin^{2}\theta_{2}\cos\phi_{1}$
% \\
%              9 & $ 3Im(a_{+}a_{-}^{*})$
% & $P_{b}\alpha_{\Lambda}$ &
% $\sin\theta\sin\theta_{1}\sin^{2}\theta_{2}\sin\phi_{1}$
%  \\
%             10 & $-{\frac{3}{2}} Re(b_{-}b_{+}^{*})$
% & $P_{b}\alpha_{\Lambda}$ &
% $\sin\theta\sin\theta_{1}\sin^{2}\theta_{2}cos(\phi_{1}+2\phi_{2})$
%    \\
%             11 & $ {\frac{3}{2}} Im(b_{-}b_{+}^{*})$
% & $P_{b}\alpha_{\Lambda}$ &
% $\sin\theta\sin\theta_{1}\sin^{2}\theta_{2}sin(\phi_{1}+2\phi_{2})$
%    \\
%             \hline
%             12 & $-{\frac{3}{\sqrt{2}}}  Re(b_{-}a_{+}^{*}+a_{-}b_{+}^{*})$
% & $P_{b}\alpha_{\Lambda}$     &
% $\sin\theta\cos\theta_{1}\sin\theta_{2}\cos\theta_{2}\cos\phi_{2}$
%       \\
%             13 & $ {\frac{3}{\sqrt{2}}}  Im(b_{-}a_{+}^{*}+a_{-}b_{+}^{*})$
% & $P_{b}\alpha_{\Lambda}$     &
% $\sin\theta\cos\theta_{1}\sin\theta_{2}\cos\theta_{2}\sin\phi_{2}$
%        \\
%             14 & $-{\frac{3}{\sqrt{2}}}  Re(b_{-}a_{-}^{*}+a_{+}b_{+}^{*})$
% & $P_{b}\alpha_{\Lambda}$     &
% $\cos\theta\sin\theta_{1}\sin\theta_{2}\cos\theta_{2}\cos(\phi_{1}+\phi_{2})$
%                   \\
%             15 & $ {\frac{3}{\sqrt{2}}}  Im(b_{-}a_{-}^{*}+a_{+}b_{+}^{*})$
% & $P_{b}\alpha_{\Lambda}$     &
% $\cos\theta\sin\theta_{1}\sin\theta_{2}\cos\theta_{2}\sin(\phi_{1}+\phi_{2})$
%                   \\
%             \hline
%             16 & $ {\frac{3}{\sqrt{2}}}  Re(a_{-}b_{+}^{*}-b_{-}a_{+}^{*})$            &
% $P_{b}$                  &
%$\sin\theta\sin\theta_{2}\cos\theta_{2}\cos\phi_{2}$
%                   \\
%             17 & $-{\frac{3}{\sqrt{2}}}  Im(a_{-}b_{+}^{*}-b_{-}a_{+}^{*})$            &
% $P_{b}$                  &
%$\sin\theta\sin\theta_{2}\cos\theta_{2}\sin\phi_{2}$
 %                  \\
%             18 & $ {\frac{3}{\sqrt{2}}}  Re(b_{-}a_{-}^{*}-a_{+}b_{+}^{*})$            &
% $\alpha_{\Lambda}$       &
% $\sin\theta_{1}\sin\theta_{2}\cos\theta_{2}\cos(\phi_{1}+\phi_{2})$
%       \\
%             19 & $-{\frac{3}{\sqrt{2}}}  Im(b_{-}a_{-}^{*}-a_{+}b_{+}^{*})$            &
% $\alpha_{\Lambda}$       &
% $\sin\theta_{1}\sin\theta_{2}\cos\theta_{2}\sin(\phi_{1}+\phi_{2})$
%       \\
%             \hline\hline
%             \end{tabular}
%\caption{Different terms describing angular distributions of \Lb\ra\jpsi\Lz decays by
%eq.~\ref{eq:jpsiLambdaRate}.}
%\label{tab:LbJpsiLAngular}
%\end{table}
%
%
The angular distribution of the \Lb\ra\jpsi\Lz decay is modelled using Ref.~\cite{Hrivnac:1994jx}.
The differential rate is written as
%
    \begin{equation}\label{eq:jpsiLambdaRate}
       w(\Omega ,\Omega_{1}, \Omega_{2}) =  {\frac{1}{(4\pi)}^{3}}
       \sum_{i=0}^{i=19} f_{1i} f_{2i}(P_{b} ,\alpha_{\Lambda}) F_{i}(\theta,
       \theta_{1},\theta_{2},\phi_{1},\phi_{2}),
      \end{equation}
where $f_{1i}$, $f_{2i}$ and $F_i$ are listed in Tab.~\ref{tab:LbJpsiLAngular}. The expression uses
four observables (angles) and depends on four complex amplitudes $a_+$, $a_-$, $b_+$, $b_-$ and two
real valued parameters for the production polarisation, $P_b$, and the \Lz decay asymmetry, $\alpha_\Lz$.
The angle $\theta$ is the angle of the \Lz momentum in \Lb rest frame with respect to
the vector $\vec{n}=\frac{\vec{p}_{inc} \times \vec{p}_{\Lb}} { |\vec{p}_{inc} \times \vec{p}_{\Lb}|}$, where
$\vec{p_{inc}}$ and $\vec{p}_{\Lb}$ are the momenta of incident proton and \Lb in the center of mass system.
The angles $\theta_1$ and $\phi_1$ are polar and azimuthal angle of the proton coming from the \Lz decay
in the \Lz rest frame with axis defined as
    $z_{1}\uparrow\uparrow\vec{p}_{\Lz}$,
    $y_{1}\uparrow\uparrow\vec{n}\times\vec{p}_{\Lz}$.
Finally, the angles $\theta_2$ and $\phi_2$ are the angles of the momenta of the muons in \jpsi rest frame
with axes defined as 
    $z_{2}\uparrow\uparrow\vec{p}_{\jpsi}$,
    $y_{2}\uparrow\uparrow\vec{n}\times\vec{p}_{\jpsi}$.

The distribution depends on the \Lz decay asymmetry parameter, $\alpha_\Lz$, the production polarisation
$P_b$ and four complex amplitudes. The $\alpha_\Lz$ is measured to be $0.642 \pm 0.013$ for \Lz.
% but for $\bar\Lz$ the precision is worse and we use up to the sign same value as for \Lz.
The production polarisation $P_b$ and magnitudes of $a_+$, $a_-$, $b_+$ and $b_-$ are measured in
Ref.~\cite{LHCb-PAPER-2012-057}. Phases are not measured therefore, as default all phases are set to
zero and then they are randomly varied to calculate the systematic uncertainty.


