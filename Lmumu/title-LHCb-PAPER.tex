% $Id: title-LHCb-PAPER.tex 61931 2014-10-14 09:51:37Z roldeman $
% ===============================================================================
% Purpose: LHCb-PAPER journal paper title page template
% Author: 
% Created on: 2010-09-25
% ===============================================================================

%%%%%%%%%%%%%%%%%%%%%%%%%
%%%%%  TITLE PAGE  %%%%%%
%%%%%%%%%%%%%%%%%%%%%%%%%
\begin{titlepage}
\pagenumbering{roman}

% Header ---------------------------------------------------
\vspace*{-1.5cm}
\centerline{\large EUROPEAN ORGANIZATION FOR NUCLEAR RESEARCH (CERN)}
\vspace*{1.5cm}
\hspace*{-0.5cm}
\begin{tabular*}{\linewidth}{lc@{\extracolsep{\fill}}r}
\ifthenelse{\boolean{pdflatex}}% Logo format choice
{\vspace*{-2.7cm}\mbox{\!\!\!\includegraphics[width=.14\textwidth]{lhcb-logo.pdf}} & &}%
{\vspace*{-1.2cm}\mbox{\!\!\!\includegraphics[width=.12\textwidth]{lhcb-logo.eps}} & &}%
\\
 & & CERN-PH-EP-2015-078 \\  % ID 
 & & LHCb-PAPER-2015-009 \\  % ID 
 & & 23 March 2015  \\
\end{tabular*}

\vspace*{1.5cm}

% Title --------------------------------------------------
{\bf\boldmath\huge
\begin{center}
  Differential branching fraction
  and angular analysis of
  \decay{\Lb}{\Lz\mumu} decays
\end{center}
}


\vspace*{0.7cm}

% Authors -------------------------------------------------
\begin{center}
The LHCb collaboration\footnote{Authors are listed at the end of this paper.}
\end{center}

%\vspace{\fill}

% Abstract -----------------------------------------------
\begin{abstract}
  \noindent
  The differential branching fraction of the rare decay
  \decay{\Lb}{\Lz\mumu} is measured as a function of \qsq, the square
  of the dimuon invariant mass.  The analysis is performed using
  proton-proton collision data, corresponding to an integrated
  luminosity of 3.0\invfb, collected by the \lhcb experiment. 
  Evidence of signal is observed in the \qsq region below the square
  of the \jpsi mass. Integrating
  over $15 < \qsq < 20$ \gevgevcccc 
 the branching fraction
 is measured as
\begin{equation}
\deriv\BF(\decay{\Lb}{\Lz\mumu})/\deriv\qsq = (1.18 \;^{+\,0.09}_{-\,0.08} \pm 0.03 \pm 0.27 )
\times 10^{-7}\;(\gevgevcccc)^{-1},
\nonumber
\end{equation}
 \noindent where the uncertainties are statistical, 
 systematic and due to the normalisation mode,
 \decay{\Lb}{\jpsi\Lz}, respectively.
 In the \qsq\ intervals where the signal is observed, angular
 distributions are studied and the forward-backward asymmetries
 in the dimuon ($A_{\rm FB}^{\ell}$) and hadron ($A_{\rm FB}^{h})$ systems
 are measured for the first time.
 In the range $15 < \qsq < 20$ \gevgevcccc they are found to be
\begin{equation}
\begin{split}
A_{\rm FB}^{\ell} & = -0.05 \; \pm 0.09 \; \text{(stat)} \; \pm  0.03 \; \text{(syst)} \text{\;and} \\
A_{\rm FB}^{h} & = -0.29 \; \pm 0.07 \; \text{(stat)} \; \pm  0.03 \; \text{(syst)}.
\end{split}
\nonumber
\end{equation}

\end{abstract}

\vspace*{1.0cm}

\begin{center}
  Submitted to JHEP
\end{center}

\vspace{\fill}

{\footnotesize 
\centerline{\copyright~CERN on behalf of the \lhcb collaboration, licence \href{http://creativecommons.org/licenses/by/4.0/}{CC-BY-4.0}.}}
\vspace*{2mm}

\end{titlepage}


%%%%%%%%%%%%%%%%%%%%%%%%%%%%%%%%
%%%%%  EOD OF TITLE PAGE  %%%%%%
%%%%%%%%%%%%%%%%%%%%%%%%%%%%%%%%

%  empty page follows the title page ----
\newpage
\setcounter{page}{2}
\mbox{~}

\cleardoublepage







