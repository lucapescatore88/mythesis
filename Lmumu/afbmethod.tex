\chapter{Angular analysis of $\protect\Lb\to\Lz\mumu$ decays}
\label{sec:ang_ana}

The angular distribution of $\Lb\to\Lz\mumu$ decays can be described 
as a function of three angles and \qsq  when neglecting the production polarisation of the \Lb.
 The two angles which are relevant for the
analysis in this chapter and are defined in Fig.~\ref{fig:Lb_angles}: $\theta_\ell$ is the angle between 
the positive (negative) muon direction in the dimuon rest frame and the dimuon system direction in the \Lb (\Lbbar) 
rest frame; similarly, $\theta_h$ is defined as the angle between the proton and the \Lz baryon directions, 
in the \Lz and \Lb rest frames. The third angle is the angle between the dimuon and \Lz decay planes, which is integrated
over in this analysis. % (for unpolarized production we are sensitive only to difference in azimuthal angles)
This chapter describes a measurement of two forward-backward asymmetries, namely those in the leptonic
($\afbl$) and in the hadronic ($\afbh$) systems. These forward-backward asymmetries
are defined as
\begin{align}
A_{\rm FB}^i(\qsq)&=\frac{\int_0^1 \frac{\deriv^2\Gamma}{\deriv\qsq\,\deriv\!\cos\theta_i} \deriv\!\cos\theta_i-
               \int^0_{-1} \frac{\deriv^2\Gamma}{\deriv\qsq\,\deriv\!\cos\theta_i} \deriv\!\cos\theta_i}{\deriv\Gamma / \deriv \qsq},
\label{eq:afbTh}
\end{align}
where $i$=$h$ or $\ell$, $\deriv^2\Gamma/\deriv \qsq\,\deriv\!\cos\theta_i$ is the two-dimensional differential rate and
$\mathrm{d}\Gamma / \mathrm{d}\qsq$ is rate integrated over the angles. 

\begin{figure}[h!]
\centering
\includegraphics[width=0.5\textwidth]{Lmumu/figs/angles.jpeg}
\caption{Graphical representation of the angles for the \decay{\Lb}{\Lz\mumu} decay.}
\label{fig:Lb_angles}
\end{figure}

The $\afbl$ observable was previously measured by LHCb also for $\Bz\ra\Kstarz\mumu$ decays
which proceed through the same quark level transition as $\Lb\to\Lz\mumu$ decays. In contrast, the hadronic
asymmetry, $\afbh$, is interesting only in the \Lb case as it is zero by definition
in the \Bz case, due to the strong decay of the \Kstarz.

\section{One-dimensional angular distributions}

This section describes the derivation of the functional form of the differential distributions 
as a function of $\cos\theta_\ell$ and $\cos\theta_h$, which are used to measure
the observables. The content of this section is based on the calculations in Ref.~\cite{Gutsche:2013pp}. 

For unpolarised \Lb production,
%
% the most general angular distribution can be written as 
%\begin{eqnarray}
%\label{bjoint3}
%W(\theta_\ell,\theta_h,\chi)  &\propto& 
%\sum_{\lambda_1,\lambda_{2},\lambda_j,\lambda'_j,J,J',m,m',\lambda_{\Lambda},
%\lambda'_{\Lambda},\lambda_{p}} 
%h^{m}_{\lambda_1\lambda_2}(J)h^{m'}_{\lambda_1\lambda_2}(J')
%e^{i(\lambda_{j}-\lambda'_{j})\chi}
%\nonumber\\ 
%&\times&
%\delta_{\lambda_{j}-\lambda_{\Lambda},\lambda'_{j}-\lambda'_{\Lambda}}
%\delta_{JJ'}
%d^J_{\lambda_j,\lambda_1-\lambda_{2}}(\theta_\ell)
%d^{J'}_{\lambda'_j,\lambda_1-\lambda_{2}}(\theta_\ell)
%H^{m}_{\lambda_{\Lambda}\lambda_{j}}(J)
%H^{m'\dagger}_{\lambda'_{\Lambda}\lambda'_{j}}(J')
%\nonumber \\
%&\times& 
%d^{1/2}_{\lambda_{\Lambda}\lambda_{p}}(\theta_h)
%d^{1/2}_{\lambda'_{\Lambda}\lambda_{p}}(\theta_h)
%h^{B}_{\lambda_{p}0}h^{B\,\dagger}_{\lambda_{p}0}\,.
%\end{eqnarray}
%where $\theta_\ell$, $\theta_h$ correspond to lepton and proton helicity angle, $\chi$ is angle between
%dimuon and \Lz decay planes (for unpolarized production we are sensitive only to difference in
%azimuthal angles), $d^J_{i,j}$ are Wigner d-functions and $h$, $h^B$ and $H$ are helicity amplitudes
%for virtual dimuon, \Lz and \Lb decays.
%The sum runs over all possible helicities with the dimuon being allowed in spin 0 and 1 states 
%($J$ and $J'$). The indeces $m$ and $m'$ run over the vector and axial-vector current contributions.
%
%Substituting for the dimuon amplitudes and integrating over three angles, one can obtain the differential
%branching fraction (eq. $V$ in Ref.~\cite{Gutsche:2013pp})
integrating over the three angles, the differential branching fraction is given in Eq.~11 of Ref.~\cite{Gutsche:2013pp} as
\begin{eqnarray}
\label{bjoint00}
\frac{\mathrm{d}\Gamma(\Lambda_b \to \Lambda \,\ell^{+}\ell^{-})}{\mathrm{d} \qsq}=
\frac{v^{2}}{2}\cdot\bigg( U^{V+A} + L^{V+A} \bigg)
+\frac{2m_\ell^{2}}{q^{2}}\cdot\frac{3}{2}\cdot
\bigg( U^{V} + L^{V} + S^{A} \bigg)\,, 
\end{eqnarray}
%where we have adopted the notations
%$\mathrm{d}\Gamma_X^{mm'}/d \qsq=X^{mm'}$ and $X^{V+A}=X^{V}+X^{A}$.
%The bilinear expressions $H^{mm'}_{X}$ ($X=U,L,S$) are defined by
%\begin{equation}
%\label{helcom2}
%\qquad
%\begin{array}{lr}
%\mbox{$ H^{mm'}_U = 
%{\rm Re}(H^{m}_{\frac{1}{2}1} H^{\dagger m'}_{\frac{1}{2}1}) + 
%{\rm Re}(H^{m}_{-\frac{1}{2}-1} H^{\dagger m'}_{-\frac{1}{2}-1}) $}  & 
%\hfill\mbox{ \rm unpolarized-transverse}\,, 
%\\
%\mbox{$ H^{mm'}_L = 
%{\rm Re}(H^{m}_{\frac{1}{2}0} H^{\dagger m'}_{\frac{1}{2}0}) + 
%{\rm Re}(H^{m}_{-\frac{1}{2}0} H^{\dagger m'}_{-\frac{1}{2}0}) $}     & 
%\hfill\mbox{ \rm longitudinal}\,, 
%\\
%\mbox{$ H^{mm'}_S =  
%{\rm Re}(H^{m}_{\frac{1}{2}t}H^{\dagger m'}_{\frac{1}{2}t}) +  
%{\rm Re}(H^{m}_{-\frac{1}{2}t}H^{\dagger m'}_{-\frac{1}{2}t})$} &  
%\hfill\mbox{ \rm scalar}\,. 
%\end{array}\\
%\end{equation}
and the lepton helicity angle differential distribution, given in Eq.~15, has the form
\begin{eqnarray}
\label{costheta2}
\frac{\mathrm{d}\Gamma(\Lambda_{b}\to \Lambda \,\ell^{+}\ell^{-})}{\mathrm{d}\qsq\,\mathrm{d}\!\cos\theta_\ell} 
&=&\,
v^{2}\cdot\bigg[\frac{3}{8}\,(1+\cos^2\theta_\ell)\cdot
\frac{1}{2} U^{V+A}  
\ + \ \frac{3}{4}\,\sin^2\theta_\ell\cdot
\frac{1}{2} L^{V+A} \bigg]\label{distr2}\nonumber\\[2mm]
&-&\,v \cdot\frac{3}{4}\cos\theta_\ell\cdot P^{VA} 
\ + \ \frac{2m_{\ell}^{2}}{q^{2}}\cdot \frac{3}{4}\cdot
\bigg[ U^{V}+ L^{V} + S^{A} \bigg]\,.
\end{eqnarray}
%Here in addition to amplitudes combinations entering differential branching fraction one more
%parity--odd term contributes
%\begin{equation}
%\label{helcom2a}
%\qqua\mathrm{d}\begin{array}{lr}
%\mbox{$H^{mm'}_P =  
%{\rm Re}(H^{m}_{\frac{1}{2}1}H^{\dagger m'}_{\frac{1}{2}1})- 
%{\rm Re}(H^{m}_{-\frac{1}{2}-1}H^{\dagger m'}_{-\frac{1}{2}-1})$} & 
%\hfill\mbox{ \rm parity--odd}
%\\
%\end{array}\\
%\end{equation}
%
In these espressions $m_\ell$ is the mass of the lepton and $v = \sqrt{1-4m_\ell^2/\qsq}$; $U$ denotes
the unpolarised-transverse contributions, $L$ the longitudinal
contributions and $S$ the scalar contribution. The superscripts $V$ and $A$ represent respectively
vector and axial-vector currents, with $X^{V+A} = X^{V} + X^{A}$. The authors of Ref.~\cite{Gutsche:2013pp} 
subsequently define the lepton-side forward-backward asymmetry as
\begin{equation}
A_{\rm FB}^{\ell} 
= - \frac{3}{2}\,\frac{v\cdot P^{VA}}
{v^{2}\cdot\big(\,U^{V+A}
+L^{V+A}\big)+\frac{2m_{\ell}^{2}}{q^{2}}\cdot 3 \cdot
\big(U^{V}+L^{V}
+S^{A}\,\big) }\,.
\label{eq:afbDef}
\end{equation}
%
%
%
%Using these results as a starting point one can rewrite Eq.~\ref{costheta2} as
%\begin{align}
%\frac{\mathrm{d}\Gamma(\Lambda_{b}\to \Lambda \,\ell^{+}\ell^{-})}{\mathrm{d}\qsq\mathrm{d}\!\cos\theta_\ell}=&
%\frac{3}{8}\frac{\mathrm{d}\Gamma}{\mathrm{d}\qsq}\left(1+\cos^2\theta_\ell\right)U^{V+A}+
%\frac{\mathrm{d}\Gamma}{\mathrm{d}\qsq}\afbl\cos\theta_\ell+
%\frac{3}{8}\sin^2\theta_\ell v^2\left(L^{V+A}\right) \nonumber \\
%&\left(U^{V}+L^{V}+S^{A}\right)\frac{3m_\ell^2}{\qsq}\left(\frac{1}{8}-\frac{3}{8}\cos^2\theta_\ell\right)
%\end{align}
%
For this analysis the massless leptons limit, $m_\ell \rightarrow 0$, is used, which is a good
approximation except at very low \qsq. Combining the previous equations and working in the massless 
limit the differential rates simplify to
\begin{eqnarray}
%\frac{\mathrm{d}\Gamma(\Lambda_b \to \Lambda \,\ell^{+}\ell^{-})}{d \qsq}=
\frac{\mathrm{d}\Gamma}{\mathrm{d} \qsq}=
\frac{v^{2}}{2}\cdot\bigg( U^{V+A} + L^{V+A} \bigg)
\label{eq:totalBR}
\end{eqnarray}
and
\begin{align}
\frac{\mathrm{d}\Gamma}{\mathrm{d}\qsq\,\mathrm{d}\!\cos\theta_\ell}=&
\frac{v^2}{2} \left[ \frac{3}{8}\left(1+\cos^2\theta_\ell\right)U^{V+A}+
\afbl\cos\theta_\ell(U^{V+A} + L^{V+A})+
\frac{3}{4}\sin^2\theta_\ell \left(L^{V+A}\right) \right].
\label{eq:differentialBR}
\end{align}
%
%
Equations~\ref{eq:totalBR} and \ref{eq:differentialBR} can be then combined to achieve the form
\begin{align}
%\frac{\mathrm{d}\Gamma(\Lambda_{b}\to \Lambda \,\ell^{+}\ell^{-})}{\mathrm{d}\qsq\mathrm{d}\!\cos\theta_\ell}=
\frac{\mathrm{d}\Gamma}{\mathrm{d}\qsq\,\mathrm{d}\!\cos\theta_\ell}=
\frac{\mathrm{d}\Gamma}{\mathrm{d}\qsq}&\left[
\frac{3}{8}\left(1+\cos^2\theta_\ell\right)\frac{U^{V+A}}{U^{V+A}+L^{V+A}}+\afbl\cos\theta_\ell\right . +
 \nonumber \\
& \left. \frac{3}{4}\sin^2\theta_\ell\frac{L^{V+A}}{U^{V+A}+L^{V+A}}\right].
\end{align}
The amplitude combination in the last term can be viewed as the ratio between the longitudinal and the sum of
longitudinal and unpolarised contributions and therefore one can define the longitudinal fraction
\begin{equation}
\fl=\frac{L^{V+A}}{U^{V+A}+L^{V+A}},
\end{equation}
which leads to the functional form used in the analysis:
\begin{align}
\frac{\mathrm{d}\Gamma}{\mathrm{d}\qsq\,\mathrm{d}\!\cos\theta_\ell}=
\frac{\mathrm{d}\Gamma}{\mathrm{d}\qsq}&\left[  \frac{3}{8}\left(1+\cos^2\theta_\ell\right)(1-\fl)+\afbl\cos\theta_\ell +
   \frac{3}{4}\sin^2\theta_\ell \fl\right]. 
   \label{eq:afbLTh}
\end{align}

Using the same steps the proton helicity distribution is given in Ref.~\cite{Gutsche:2013pp} as
\begin{equation}
\frac{\mathrm{d}\Gamma(\Lambda_{b}\to \Lambda(\to p \pi^{-})\ell^{+}\ell^{-})}
     {\mathrm{d}\qsq\,\mathrm{d}\!\cos\theta_h} 
={\rm Br}(\Lambda \to p\pi^{-})
 \frac{\mathrm{d}\Gamma(\Lambda_b \to \Lambda\, \ell^{+}\ell^{-})}{\mathrm{d}\qsq}
\Big(\frac{1}{2}+\afbh\cos\theta_h\Big) \,,
\label{eq:afbBTh}
\end{equation}
and $\afbh$ is defined as
\begin{equation}
\afbh=\frac{1}{2}\alpha_{\Lz}P_z^\Lz(\qsq),
\end{equation} 
where $P_z^\Lz(\qsq)$ is the polarisation of the daughter baryon, \Lz,
and $\alpha_{\Lz} = 0.642 \pm 0.013$~\cite{PDG2014} is the \Lz decay asymmetry parameter.

%is defined as (eq.~$21$ in Ref.~\cite{Gutsche:2013pp})
%\begin{equation}
%\label{pz1}
%P_{z}^{\Lambda}=\frac{
%v^{2}\cdot\big(\,P^{V+A}
%+L_{P}^{V+A}\big)+\frac{2m_{\ell}^{2}}{q^{2}}\cdot 3 \cdot 
%\big(P^{V}+L_{P}^{V}
%+S_{P}^{A}\,\big)
%}
%{v^{2}\cdot\big(\,U^{V+A}
%+L^{V+A}\big)+\frac{2m_{\ell}^{2}}{q^{2}}\cdot 3 \cdot
%\big(U^{V}+L^{V}
%+S^{A}\,\big) 
%} \,.
%\end{equation}

%where two parity-violating combinations of amplitudes
%\begin{equation}
%\label{helcom3}
%\qqua\mathrm{d}\begin{array}{lr}
%\mbox{$H^{mm'}_{L_{P}}=
%{\rm Re}
%(H^{m}_{\frac{1}{2}\,0}H^{\dagger m'}_{\frac{1}{2}\,0}-H^{m}_{-\frac{1}{2}\,0}
%H^{\dagger m'}_{-\frac{1}{2}\,0})$} & 
%\hfill\mbox{ \rm longitudinal--polarized}\,, 
%\\
%\mbox{$H^{mm'}_{S_{P}}=
%{\rm Re}
%(H^{m}_{\frac{1}{2}\,t}H^{\dagger m'}_{\frac{1}{2}\,t}-H^{m}_{-\frac{1}{2}\,t}
%H^{\dagger m'}_{-\frac{1}{2}\,t})$}
% & 
%\hfill\mbox{ \rm scalar--polarized}\,, 
%\end{array}\\
%\end{equation}
%enter in addition to the ones we already discussed.

%While lepton side observables are well known from meson decays, hadron side is unexplored. To get
%feeling how sensitive $\afbh$ is to new physics, we plot together three different cases: first,
%SM using predictions and numerical values from Ref.~\cite{Gutsche:2013pp}, scenario in which Wilson
%coefficients are same as in the SM, just sign of the $C_7$ is flipped and finally modifying $C_7$,
%$C_9$ and $C_{10}$ by amount compatible with fit to meson data including $P_5'$ in
%Ref.~\cite{Descotes-Genon:2013wba}.
%Resulting dependence of $P_z^\Lz$ is shown in Fig.~\ref{fig:PzLpred}.
%
%\begin{figure}
%\centering
%\includegraphics[width=0.7\textwidth]{Lmumu/figs/AFBhPrediction}
%\caption{Expectation for $P_z^\Lz$ which is proportional to $\afbh$ in the SM (black line), SM
%like scenario with flipped sign on $C_7$ (red line) and scenario with all three Wilson coefficients
%modified according to Ref.~\cite{Descotes-Genon:2013wba} (blue line). There is little difference
%between those tree scenarious.}
%\label{fig:PzLpred}
%\end{figure}
%
%As can be seen, there is little difference between three scenarious. We also tried to investigate
%analytically. In that case we put lepton mass to zero and neglect $C_7$ terms, which are small at
%high $\qsq$ where we have most of the signal. In such case, both numerator and denumerator in
%eq.~\ref{pz1} are proportional to $(|C_9|^2+|C_{10}|^2)$ and thus dependence on Wilson coefficients
%drops out in first order. So only difference can come from $C_7$ and its interference with other two Wilson
%coefficients and thus effects are expected to be small. From this we conclude that main help of this
%variable is in constraining form factors in the studied decay. 


The above expressions assume that \Lb is produced unpolarised, which is supported by the recent LHCb
measurement in Ref.~\cite{LHCb-PAPER-2012-057}. Possible effects due to a non-zero production polarisation
are investigated as systematic uncertainties (see Sec.~\ref{sec:ang_pol_sys}).


\section{Multi-dimensional angular distributions}
\label{sec:multidim_ang_distrib}

The equations were modified to take into account the effects of the production polarisation.
%
%, we modify Eq.~\ref{bjoint3} to  
%\begin{eqnarray}
%\label{bjoint4}
%W(\theta,\theta_\ell,\phi_\ell,\theta_h,\phi_\Lambda)  &\propto& 
%\sum_{\lambda_1,\lambda_{2},\lambda_j,\lambda'_j,J,J',m,m',\lambda_{\Lambda},
%\lambda'_{\Lambda},\lambda_{p}} 
%h^{m}_{\lambda_1\lambda_2}(J)h^{m'}_{\lambda_1\lambda_2}(J')
%\rho_{\lambda_{j}-\lambda_{\Lambda},\lambda'_{j}-\lambda'_{\Lambda}}(\theta)
%(-1)^{J+J'}
%\nonumber\\ 
%&\times&D^J_{\lambda_j,\lambda_1-\lambda_{2}}(\theta_\ell,\phi_\ell)
%D^{J'}_{\lambda'_j,\lambda_1-\lambda_{2}}(\theta_\ell,\phi_\ell)
%H^{m}_{\lambda_{\Lambda}\lambda_{j}}(J)
%H^{m'\dagger}_{\lambda'_{\Lambda}\lambda'_{j}}(J')
%\nonumber \\
%&\times& 
%D^{1/2}_{\lambda_{\Lambda}\lambda_{p}}(\theta_h,\phi_\Lambda)
%D^{1/2}_{\lambda'_{\Lambda}\lambda_{p}}(\theta_h,\phi_\Lambda)
%h^{B}_{\lambda_{p}0}h^{B\,\dagger}_{\lambda_{p}0}\,,
%\end{eqnarray}
%where
%\begin{equation}
%D^{J}_{m,n}(\theta_i,\phi_i)=e^{-i(m-n)\phi_i}d^J_{m,n}(\theta_i).
%\end{equation}
In the modified version, an angle $\theta$ is defined as the angle between the \Lz direction
in the \Lb rest frame and the vector $\hat{n} = \hat{p}_{inc} \times \hat{p}_{\Lb}$, where $\hat{p}_{inc}$
represents the direction of the incoming proton; this angle is sensitive to the production polarisation.
%through the spin-density matrix. % in Eq.~\ref{eq:spinDensity}.
%, while the angle between the decay planes $\chi$ is changed to the polar angles of proton and positive muon.
Integrating over all the angles except $\theta_\ell$ results in 
the same distribution as in the unpolarised case (Eq.~\ref{costheta2}). Therefore, in the case of uniform
efficiency, the lepton side forward-backward asymmetry, $\afbl$, is unaffected by the production
polarisation. To be able to estimate the effect of the production polarisation in the case of non-uniform efficiency, 
the differential distribution in $\theta$ and $\theta_\ell$ is derived, which in the massless leptons
limit becomes (up to a  constant multiplicative factor)
%
\begin{align}
\renewcommand{\arraystretch}{1.5}
\frac{\mathrm{d}\Gamma(\Lambda_{b}\to \Lambda \,\ell^{+}\ell^{-})}{\mathrm{d}\qsq\,\mathrm{d}\!\cos\theta\, \mathrm{d}\!\cos\theta_\ell}=&
\frac{\mathrm{d}\Gamma}{\mathrm{d}\qsq}\left\{  \frac{3}{8}\left(1+\cos^2\theta_\ell\right)(1-\fl)+\afbl\cos\theta_\ell +
   \frac{3}{4}\sin^2\theta_\ell \,\fl+\right. \nonumber \\[8pt]
&P_b\cos\theta\left[ -\frac{3}{4}\sin\theta_\ell^2\,O_{Lp}+
  \frac{3}{8}\left(1+\cos\theta_\ell^2\right)O_P\right. \nonumber \\[8pt]
&\left.\left.-\frac{3}{8}\cos\theta_\ell\,O_{UVA} \right]\right\}\,,
\label{eq:lepton2D}
\end{align}
where three more observables are defined
\begin{align}
O_{Lp}=&\frac{L_P^{V}+L_P^{A}}{U^{V+A}+L^{V+A}}, \nonumber \\
O_P=&\frac{P^{V}+P^{A}}{U^{V+A}+L^{V+A}}, \nonumber \\
O_{UVA}=&\frac{U^{VA}}{U^{V+A}+L^{V+A}}. \nonumber
\end{align}
%
In the massless leptons approximation two of these quantities are related to the hadron side
forward-backward asymmetry as
\begin{equation}
\frac{1}{2}\alpha_\Lambda \left(O_P+O_{Lp}\right)=\afbh\,.
\end{equation}
%
Following the same steps as for the lepton case, after integrating over all the angles except $\theta_h$ one finds
that the hadron side asymmetry, $\afbh$, is also unaffected by the production polarisation in case of uniform
efficiency and the differential distribution in $\theta$ and $\theta_h$ has the form
\begin{align}
\frac{\mathrm{d}\Gamma(\Lambda_{b}\to \Lambda \,\ell^{+}\ell^{-})}{\mathrm{d}\qsq\,\mathrm{d}\!\cos\theta\,\mathrm{d}\!\cos\theta_h}=
\frac{\mathrm{d}\Gamma}{\mathrm{d}\qsq}&\left[1+2\afbh\cos\theta_h+P_b\left(O_P-O_{Lp}\right)\cos\theta\right.\nonumber \\
&\left.+\alpha_\Lz P_b\left(1-2\fl\right)\cos\theta\cos\theta_h\right].
\label{eq:hadron2D}
\end{align}

%It should be noted that fact that in case of uniform efficiency both angular observables are
%unaffected by production polarization comes from the fact that terms proportional to $\sin\theta$
%cancel out when integrating over $\phi_\ell$ and $\phi_\Lambda$ and remaining terms containing production
%polarization go as $\cos\theta$, which after integrating over it yields zero.
%
In order to use these distributions, expectations for the three additional observables,
which do not enter one-dimensional distributions, are needed.
Expectations are calculated using form factors and numerical inputs from Ref.~\cite{Gutsche:2013pp}
and are listed in Appendix~\ref{ap:LbLmumuAngular}.
%
%In this calculation the massless lepton limit
%is used and compared to the
%Ref.~\cite{Gutsche:2013pp} we turn off long-distance contributions by setting dileptonic decay
%widths of \jpsi and $\psi(2S)$ to zero. In order to obtain binned values, we need to integrate
%corresponding amplitudes over \qsq interval and this is done separately for those in the numerator
%and denominator rather than integrating ratio of amplitudes. Results are in tables
%\ref{tab:obsGutsche1}. Alternatively we calculate also expectation using lattice QCD form factors
%from Ref.~\cite{Detmold:20VAvy} by substituting those instead of form factors from Ref.~\cite{Gutsche:2013pp}.
%These results are in table \ref{tab:lQCD1}.
%
%
%
%
%\begin{table}
%\begin{center}
%\begin{tabular}{lcccccc}\hline
%\qsq [$GeV^2/c^2$]  & $\afbl$ & $P_z^\Lz$  & $\fl$   & $O_P$  & $O_{Lp}$ & $O_{UVA}$ \\ \hline
%0.1 -- 2.0          &  0.023     & -0.963    & 0.904   & -0.096 & -0.867   &  0.015  \\ 
%2.0 -- 4.0          & -0.085     & -0.962    & 0.870   & -0.VA8 & -0.834   & -0.058  \\ 
%4.0 -- 6.0          & -0.163     & -0.962    & 0.771   & -0.A4 & -0.739   & -0.V1  \\ 
%V.0 -- VA.5        & -0.316     & -0.934    & 0.541   & -0.427 & -0.507   & -0.A7  \\ 
%15.0 -- 16.0        & -0.346     & -0.866    & 0.450   & -0.468 & -0.398   & -0.271  \\ 
%16.0 -- 18.0        & -0.336     & -0.799    & 0.416   & -0.455 & -0.344   & -0.288  \\ 
%18.0 -- 20.0        & -0.260     & -0.585    & 0.368   & -0.354 & -0.231   & -0.3V  \\ \hline 
%1.0 -- 6.0          & -0.086     & -0.962    & 0.865   & -0.132 & -0.829   & -0.059  \\ 
%15.0 -- 20.0        & -0.306     & -0.721    & 0.402   & -0.415 & -0.306   & -0.295  \\ \hline
%\end{tabular}
%\end{center}
%\caption{Prediction for angular observables entering two-dimensional angular distributions.
%Prediction is based on lattice QCD form factors from Ref.~\cite{Detmold:20VAvy}.}
%\label{tab:lQCD1}
%\end{table}

For completeness, the differential distribution in $\cos\theta_\ell$ and $\cos\theta_h$ has the form
\begin{align}
\frac{\mathrm{d}\Gamma(\Lambda_{b}\to \Lambda \,\ell^{+}\ell^{-})}{\mathrm{d}\qsq\,\mathrm{d}\!\cos\theta_h\, \mathrm{d}\!\cos\theta_\ell}=&
\frac{3}{8}+\frac{6}{16}\cos^2\theta_\ell\,(1-\fl)-\frac{3}{16}\cos^2\theta_\ell\, \fl
+\afbl\cos\theta_\ell+ \nonumber \\[8pt]
& \left(\frac{3}{2}\afbh-\frac{3}{8}\alpha_\Lz O_P\right)\cos\theta_h
-\frac{3}{2}\afbh\cos^2\theta_\ell\cos\theta_h-\frac{3}{16}\fl+ \nonumber \\[8pt]
& \frac{9}{16}\fl\sin^2\theta_\ell+\frac{9}{8}\alpha_\Lz \cos^2\theta_\ell\cos\theta_h O_P- \nonumber \\[8pt]
& \frac{3}{2}\alpha_\Lz \cos\theta_\ell\cos\theta_h O_{UVA}.
\label{eq:2DcosThetaLandB}
\end{align}
%It does not need any additional inputs compared to previous two-dimensional distributions.


\section{Angular resolution}
\label{sec:and_resolution}

This section describes a study of the angular resolution performed in order to achieve a better understanding
of detector and reconstruction effects. This is then used to study systematic uncertainties (see Sec.~\ref{sec:ang_pol_sys}).
The study is performed by analysing simulated events and comparing generated and reconstructed quantities.
%Plots shown are done on long-long events only but we performed the same study separately on down-down events too.
%In Fig.~\ref{fig:resolutionvsq2ang} the same difference is shown also as a function of \qsq revealing again no bias.
Figures~\ref{fig:resolutionvsq2ang} and~\ref{fig:resolutionvsq2ang_DD} show the difference between true and measured 
angular observables, $\cos \theta_\ell$ and $\cos \theta_h$, as a function of the observables themselves.
These distributions are centred at zero indicating no bias in the measurement.
The spread of these distributions around the central value can be interpeted as an estimate of the angular resolution.
Taking vertical slices of the plots in Figs.~\ref{fig:resolutionvsq2ang} and~\ref{fig:resolutionvsq2ang_DD}  one obtains approximately Gaussian
distributions centred at zero. These are fit with a single gaussian and its width
is interpreted as the angular resolution. Table~\ref{tab:resolutions} reports the average resolutions
for the two angular observables separately for long and downstream candidates. Candidates built
from long tracks are characterised by a better angular resolution due to a better momentum and vertex position resolutions.
%Furthermore also the resolution of the lepton observable is better than the hadronic one.

\begin{table}[b]
\centering
\caption{Average angular resolutions for downstream and long candidates.} % integrated over the full interval and the full available \qsq.}
\begin{tabular}{$c|^c^c}
\rowstyle{\bfseries}
Observable      & Downstream & Long     \\ \hline
$\cos \theta_\ell$ & 0.015 & 0.010 \\
$\cos \theta_h$ & 0.066 & 0.014 \\
\end{tabular}
\label{tab:resolutions}
\end{table}
%
\begin{figure}
\centering
\includegraphics[width=0.9\textwidth]{Lmumu/figs/resolution/RmT_vs_cosThetaL_LL.pdf}
\includegraphics[width=0.9\textwidth]{Lmumu/figs/resolution/RmT_vs_cosThetaB_LL.pdf}
%\includegraphics[width=0.8\textwidth]{Lmumu/figs/resolution/RmT_vs_cosThetaL_DD.pdf}
%\includegraphics[width=0.8\textwidth]{Lmumu/figs/resolution/RmT_vs_cosThetaB_DD.pdf}
%\includegraphics[width=0.49\textwidth]{Lmumu/figs/resolution/RmTcosThetaL_vs_q2_DD.pdf}
%\includegraphics[width=0.49\textwidth]{Lmumu/figs/resolution/RmTcosThetaL_vs_q2_LL.pdf}
 \caption{Difference between generated and reconstructed angular observables as a function of the 
 observables themselves for long candidates: for $\cos\theta_\ell$ (top) and $\cos\theta_h$ (bottom). 
 The spread of these distributions can be interpreted as the angular resolution.
 %As the plots are made using fully selected rare samples the bottom plots present empty bands
 %corresponding to the charmonium vetoes.
 }
\label{fig:resolutionvsq2ang}
\end{figure}
%
\begin{figure}
\centering
%\includegraphics[width=0.8\textwidth]{Lmumu/figs/resolution/RmT_vs_cosThetaL_LL.pdf}
%\includegraphics[width=0.8\textwidth]{Lmumu/figs/resolution/RmT_vs_cosThetaB_LL.pdf}
\includegraphics[width=0.9\textwidth]{Lmumu/figs/resolution/RmT_vs_cosThetaL_DD.pdf}
\includegraphics[width=0.9\textwidth]{Lmumu/figs/resolution/RmT_vs_cosThetaB_DD.pdf}
%\includegraphics[width=0.49\textwidth]{Lmumu/figs/resolution/RmTcosThetaL_vs_q2_DD.pdf}
%\includegraphics[width=0.49\textwidth]{Lmumu/figs/resolution/RmTcosThetaL_vs_q2_LL.pdf}
 \caption{Difference between generated and reconstructed angular observables as a function of the 
 observables themselves for downstream candidates: for $\cos\theta_\ell$ (top) and $\cos\theta_h$ (bottom). 
 The spread of these distributions can be interpreted as the angular resolution.
 %As the plots are made using fully selected rare samples the bottom plots present empty bands
 %corresponding to the charmonium vetoes.
 }
\label{fig:resolutionvsq2ang_DD}
\end{figure}
%
%
%
%Figure~\ref{fig:avgResol} shows the angular resolution in two-dimensional bins of \qsq and angular observables.
%\begin{figure}
%\centering
%\includegraphics[width=0.49\textwidth]{Lmumu/figs/resolution/resolution2D_cosThetaL_DD.pdf}
%\includegraphics[width=0.49\textwidth]{Lmumu/figs/resolution/resolution2D_cosThetaB_DD.pdf} \\
%\includegraphics[width=0.49\textwidth]{Lmumu/figs/resolution/resolution2D_cosThetaL_LL.pdf}
%\includegraphics[width=0.49\textwidth]{Lmumu/figs/resolution/resolution2D_cosThetaB_LL.pdf}
%\caption{Angular resolution for $\cos \theta_\ell$ (left plots) and  $\cos \theta_h$ (right plots)
%as a function of the angular observables and \qsq for downstream (upper plots) and
%long (lower plots) candidates. White bands correspond to the \jpsi and \psitwos resonances
%which are excluded from the study.}
%\label{fig:avgResol}
%\end{figure}


