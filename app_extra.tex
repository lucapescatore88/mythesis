\chapter{Extra}

\subsection{Quantum Electrodynamics: an example of gauge field theory }

{\em Possible addition around Sec.~\ref{sec:EMandEW}.}



The EM theory results from requiring the fermion Lagrangian to be invariant under local gauge transformations.
The electron and positron free fermionic fields are defined as $\psi(x)$ and $\bar{\psi(x)}$, where x is a relativistic four vector.
A local gauge transformation can always be written as:

\begin{align}
\psi(x) \rightarrow \psi(x') = e^i{\alpha(x)}\psi(x)
\end{align}

where $\alpha(x)$ can be any function of space and/or time. The free fermion Lagrangian, given by

\begin{equation}
\mathcal{L} = -i \bar{\psi(x)} \gamma^{\mu} \partial_\mu \psi(x) - m\bar{\psi(x)} \psi
\end{equation}

is not invariant under such transformations. Greek indices denote space-time directions and imply summation and $\gamma^i$ are the Dirac matrices. If we apply the local gauge transformation and we subtract the initial Lagrangian we get a remaining term

\begin{equation}
\Delta \mathcal{L} = \mathcal{L}' - \mathcal{L} = -i \bar{\psi}  \gamma^{\mu} \psi \partial_\mu \alpha(x)
\end{equation}

In order to make the Lagrangian invariant we can introduce a vector field $A$, which transforms as described by Eq. \ref{gauge invariance},  to %compensate for the remaining term: this is the photon field.

\begin{equation}
\label{gauge invariance}
A'_\mu = A_\mu -\frac{1}{e}\partial_\mu \alpha(x)
\end{equation}

Redefining then the field derivative $D_\mu = \partial_\mu - ieA_\mu$, we obtain the invariant Lagrangian:

\begin{equation}
\mathcal{L} = -i \bar{\psi(x)} \gamma^{\mu} D_\mu \psi(x)  - m\bar{\psi(x)} \psi = -i \bar{\psi(x)} (\gamma^{\mu} \partial_\mu  - m)\psi(x) + e\bar{\psi}  \gamma^{\mu} \psi A_\mu
\end{equation}

In this Lagrangian the first term describes the behaviours of a free fermion and the second term,
$\bar{\psi}  \gamma^{\mu} \psi A_\mu$, the electromagnetic interaction through the photon represented by $A_\mu$.


\section{Anomalies}

{\em Possible addition around Sec.~\ref{sec:exp_status}.}

Various anomalies were observed in the past years with respect to SM predictions.
This section reports a brief review of these anomalies, limiting to $B$ physics.

The measurement of the CKM matrix elements $V_{ub}$ and $V_{cb}$
is vital for analysis in the flavour sector. Both these quantities can be measured
using tree level transitions, which are assumed to be free from NP.
Decays such as $B\to D^{*}\ell\nu$ are used to measure $V_{cb}$
and $B\to\pi\ell\nu$ for $V_{ub}$ as well as inclusive decays.
several measurements, mainly from BaBar and LHCb~\cite{Aaij:2015bfa,}, observe a discrepancy
at $2\sigma$ level between the values found using the exclusive and inclusive approaches.
This has recently increased to $3\sigma$ level due to improvements in form factor calculations~\cite{Crivellin:2014zpa}.
NP can modify the values of the CMK matrix elements as described in Ref.~\cite{}.

Secondly a series of anomalies was found in recent LHCb measurements of semileptonic $B$ decays.
The branching ratios of the $\decay{B}{K\mumu}$, $\decay{B}{\Kstarz\mumu}$
and $\decay{\Bs}{\phi\mumu}$~\cite{LHCB-PAPER-2014-006,LHCB-PAPER-2013-017,LHCB-PAPER-2013-019}
are all found to be slightly below the predicted values. Although taken by itself each measurements
does not present relevant discrepancies, the systematic deviation seems to indicate a more general picture.
Angular analysis were also performed for these decays and, while most observables are found
to agree with SM predictions, the measurement of the $P'_5$ observable in $\decay{B}{\Kstarz\mumu}$
resulted in a local $3.7\sigma$ deviation with respect to predictions~\cite{}.
At the same time the measurement of the $R_K$ ratio, between the branching fractions
of the \Bz\to\Kstarz\mumu and \Bz\to\Kstarz\ee decays, showed a $2.6\sigma$ deviation from unity,
indicating the possibility of a violation of lepton flavour universality.
Authors of Ref.~\cite{Altmannshofer:2014rta} performed a global fit taking into account of several measurements
and found that a model with a NP component in $C_9$ is preferred with respect to the SM at $4.3\sigma$ level.
Finally, one more discrepancy linked to this picture is the branching fraction of the $h\to\mu\tau$ decay,
which is found to be different from zero at $2.4\sigma$ level.


 








