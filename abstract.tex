\chapter*{ABSTRACT}

Flavour Changing Neutral Currents are transitions between different quarks with the same charge such as 
$\bquark\to\squark$ processes. These are forbidden at tree level in the Standard Model (SM) but
can happen through loop diagrams, which causes the branching ratio of this type of decays
to be small, typically \mbox{$\sim10^{-6}$} or less. Particles beyond the SM can contribute in the loops enhancing
the branching fractions of these decays, which are therefore very sensitive new physics.
In this work two analysis of semileptonic $\bquark\to\squark\ll$ decays are presented.
First, $\Lb\to\Lz\mumu$ decays are analysed to measure their branching fraction as a function
of the square of the dimuon invariant mass, \qsq. Furthermore, an angular analysis of these decays is performed for the first time. 
Secondly, $\Bz\to\Kstarz\ll$ decays are analysed measuring the \RKst ratio between the muon, $\Bz\to\Kstarz\mumu$, 
and electron, $\Bz\to\Kstarz\ee$, channels, which is interesting as it is largely free
from uncertainties due to the knowledge of the hadronic matrix elements.
This thesis is organised in the following way. Chapter~\ref{ch:introduction} introduces the Standard Model and 
the concept of flavour and explains how rare decays can help us in the quest for physics beyond the SM.
Chapter~\ref{ch:detector} describes the LHCb detector, which was used to collect the data analysed in this thesis.
This chapter also includes studies performed to validate the hadronic physics in LHCb simulation software.
Chapter~\ref{sec:Lmumu_intro} deals with the measurement of the differential branching fraction of the $\Lb\to\Lz\mumu$ 
decay, while Chapter~\ref{sec:ang_ana} describes its angular analysis.
Finally, Chapter~\ref{sec:RKst_theory} reports the measurement of the $R_{\Kstarz}$ ratio.

\clearpage

\chapter*{DECLARATION OF AUTHORS CONTRIBUTION}

I am one of the main authors of the two analysis reported in Chapters~\ref{sec:Lmumu_intro},
\ref{sec:ang_ana} and~\ref{sec:RKst_theory}. For the analysis 
of $\Lb\to\Lz\mumu$ decays I collaborated with Michal Kreps, who took care of implementing the decay
model to re-weight the simulation and provided simulated samples. 
Furthermore, I want to thank him for the advice given throughout. 
The work on this analysis was published and can be found at Ref.~\cite{Aaij:2015xza}.
For the $R_{\Kstarz}$ analysis, described in Ch.~\ref{sec:RKst_theory}, I actively participated
in most stages of the analysis collaborating with Simone Bifani. In particular I took care of
the production of various simulated samples, participated in the definition of the selection,
in the yields extraction and I provided a fit and data-reduction framework.
Finally, as a service work for the LHCb experiment, I developed the tools used to perform
the validation studies described in Sections~\ref{sec:validation}--\ref{sec:radlength_conlsusions}.
%Furthermore, I contributed to the LHCb experiment in the role of ``Monte Carlo liaison" for two years.
%This is a connection role between the physics analysis groups and the simulation team providing
%simulated samples vital for most analysis. Finally, I have taken several shifts in the control room 
%during the Run II data taking in 2015, checking the smooth running of the detector.

\clearpage
\chapter*{ACKNOWLEDGEMENTS}

I thank everybody, evvvvvvvverybody!
%First of all I would like to thank Nigel, who always supported me in these years and
%granted be many good opportunities. I think I could not hope for a better supervisor.
%A big thank you also to Simone, with whom I collaborated for the $R_{\Kstarz}$ analysis
%and from who taught me a lot. Thanks also to the people of the Birmingham LHCb group:
%Cristina, Jimmy and also to Michal, who adopted us, Birmingham students, for a while.
%A special `thank you' goes to Pete, who shared with me this three years experience.
%I think it would have been a very different and less interesting experience without him.
%A `thank you' also goes to the members of the LHCb collaboration and in particular of the Rare
%Decays Working Group.; in particular to the working group conveners Gaia, Tom and Marco
%and to Gloria, who patiently guided me through the depts of the LHCb software.
%Finally, I'm grateful to the Vincenzo and the LHCb Bologna group, who kindly hosted me
%for a few months and in particular to Umberto for all the wisdom he shared. 
%I want also to thank the LTA folks, who where with me during the long period I spent at CERN
%and especially Mark and Lewis, adventure companions. And speaking about CERN people
%a great `thank you' to Lorenzo, because when it's 1pm I always feel that I should be in front of the trays.
%Going now to who is always waiting for me in Italy when I go back, a big `thank you' to my
%family for all their support and all the italian food they brought me while I was living abroad.
%Thank you my dad Orazio, my mum Paola and my sisters Giulia and
%Silvia. A big `thank you' also to my friends Ivan, Enrico, Martina, Federico, Valentina, Letizia
%and all the others. And finally, last but not least, a giant thank you to Lucia, who is the engine
%of my life and to whom this thesis is dedicated.

\cleardoublepage
~

\begin{flushright}
  \emph{A Lucia, \\
  perch\'{e} quando tutto perde di senso \\
  tu sei il mio piccolo mondo felice.}
  
  \vspace{10cm}
  
   \emph{
   Nec per se quemquam tempus sentire fatendumst \\
   semotum ab rerum motu placidaque quiete. \\
   (Lucretius, De rerum natura, vv. 462-463 )
   %Nel niente c'\`{e} una via che conduce \\
   %lontano dalla polvere del mondo.\\
   %(F. Bertossa)
   } 
\end{flushright}

\cleardoublepage
