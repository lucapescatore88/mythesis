\chapter{Testing lepton flavour universality with $R_{\Kstarz}$}
\label{sec:RKst_theory}

Lepton Flavour Universality (LFU) is the equality of the weak coupling constants for all leptons.
%and can be broken in NP scenarios. 
FCNC processes, which are forbidden in the SM at tree level and proceed only via loop diagrams,
are ideal to study LFU as new physics contributing in the loops could break the flavour symmetry.
 
In this work $b \rightarrow s\mumu (\ee)$ decays are studied to test LFU between electrons and muons. 
In particular, the \Bz meson semileptonic decays $\Bz\to\Kstarz\ll$ are considered.
Figure~\ref{fig:RKpenguins} shows the possible Feynman diagrams  producing such decays while 
Fig.~\ref{fig:NPpenguins} illustrates how these Feynman diagrams may include new particles. 
A series of recent LHCb measurements~\cite{TomRDreview} points to a tension with SM predictions, which makes
these processes particularly interesting as they can provide independent verifications of the existing discrepancy.
%
\begin{figure}[h]
\centering \includegraphics[width=0.8\textwidth]{RKst/figs/penguins3.png}
\caption{Loop diagrams producing $\Bz \to \Kstarz\ll$ decays.}
\label{fig:RKpenguins}
\end{figure}

In order to exploit the sensitivity of loop diagrams, in 2004 Hiller and Kruger proposed the measurement 
of the $R_H$ ratios~\cite{Hiller:2003js}, defined as
\begin{equation}
\label{eqRX}
R_{H} = \frac{\int_{\qsq_{min}}^{\qsq_{max}} \frac{ d\BR(\Bd \to H \mumu) }{\mathrm{d}\qsq}\mathrm{d}\qsq}{ \int_{\qsq_{min}}^{\qsq_{max}}\frac{ d\BR(\Bz \to H\epem) }{\mathrm{d}\qsq} \mathrm{d}\qsq} ,
\end{equation}
where $H$ can be an inclusive state containing an $s$ quark ($X_s$) or an $s$-quark resonance such as $K$ or $\Kstarz$.
In this quantity the differential branching fraction is integrated over the dilepton invariant mass squared, \qsq, from 
$\qsq_{min} = 4m_{\mu}^2$, which is the threshold for the $\mu\mu$ process, up to \mbox{$\qsq_{max} = (m_{\Bz} - m_H)^2$.} 

%The notation $\BR(X \rightarrow ~final ~state)$ denotes the fraction of $X$ particles which decays in the 
%given final state, this is called ``branching ratio". For example $\BR(\Bz\to\Kstar\mumu)$ is the
%fraction of \Bz particles which decays into $\Kstar\mumu$ with respect to all allowed \Bz decays.

The advantage of using ratios of branching fractions as observables is that, in the theoretical prediction, hadronic 
uncertainties cancel out. Furthermore, some of the experimental systematic uncertainties also approximately cancel in 
the ratios, improving the precision of the the measurement. For example, the measured quantities are the number of 
$\mu\mu$ and $ee$ decays recorded in a certain period of time. The luminosity, $\mathcal{L}$, is then used to obtain a
cross section, $\sigma$, using $R = \mathcal{L}\sigma$, where $R$ is the rate at which the decays occur. 
However, the luminosity measurement, usually a source of systematic uncertainty, appears on both
sides of the ratio and therefore cancels out.

\begin{figure}[h]
\centering \includegraphics[width=0.8\textwidth]{RKst/figs/penguins.png}
\caption{Example of penguin diagrams, on the left involving SM particles and on the right 
involving new possible particles.}
\label{fig:NPpenguins}
\end{figure}

Since the SM assumes lepton flavours universality, the predicted value of the ratio 
is $R_H = 1$, when the leptons are massless. Taking into account effects of order 
$m_\mu^2 / m_b^2$ Hiller and Kruger calculate that in the SM and in the full \qsq range~\cite{Hiller:2003js}:
%
\begin{align}
R_{X_s} = 0.987 \pm 0.006, \\
R_K = 1.0000 \pm 0.0001, \\
R_{\Kstarz} = 0.991 \pm 0.002; \\
\end{align}
%
\noindent
under the assumptions that:
%
\begin{itemize}
\item right-handed currents are negligible;
\item (pseudo-)scalar couplings are proportional to the lepton mass;
\item there are no CP-violating phases beyond the SM.
\end{itemize}

The measurement of the $R_H$ ratios is of particular interest after the recent
measurement of the branching fraction of the $\Bs\to\mumu$ decay~\cite{CMS:2014xfa}, 
where no evidence of new physics was found. In fact the quantities $(R_H - 1)$ and
$\BR(\Bs \to \mumu)$ remain proportional with
%
\begin{equation}
\frac{R_H - 1}{\BR(\Bs \to \mumu)} \sim 2 \cdot 10^{-5}.
\end{equation}
%
A joint measurement of these two quantities can give much information and constrain MFV models.
If $R_H = 1$ and $\BR(\Bs \to \mumu)$ is close to the SM prediction as it is measured to be at present, 
this will allow strong constraints to be established on extensions of the SM.
If instead $R_H > 1$ and the equation above is not verified, this would mean that one of the
assumptions listed above are not verified, which can happen is some extensions of the SM, such
as Super-symmetric models with broken R-parity.
%A series of recent LHCb measurements~\cite{TomRDreview} shows tensions with
%SM predictions, which makes it interesting to further investigate these processes.

\section{Combining ratios}

The full power of the $R_H$ ratios in understanding new physics scenarios comes from
their combinations.  In Ref.~\cite{Hiller:2014ula} Hiller and Schmaltz propose the measurement 
of the double ratios, $X_H = R_H / R_K$, which not only can test LFU but also allow
to disentangle the nature of the new physics that lies behind it. These ratios are in fact sensitive
to FCNCs of right-handed currents. Furthermore, in Ref.~\cite{Hiller:2014ula} the study is extended
to \Bs decays such as $\Bs\to\phi\ll$ and $\Bs\to\eta\ll$.

Parity and Lorentz invariance require that the Wilson Coefficients with left-handed chirality ($C$)
and their right-handed counterparts ($C'$) appear in the decay amplitude of exclusive decays in
specific combinations, \emph{e.g.}:
\begin{equation}
\begin{array}{ll}
C+C': & K, \Kstarz_{\perp}, ...  \\
C-C': & K_0(1430), \Kstarz_\parallel, ...
\end{array}
\end{equation}
where the labels for the \Kstarz meson represent its longitudinal (0), parallel $(\parallel)$ and
perpendicular $(\perp)$ transversity components. The $C$ contributions are universal for
all decays and therefore $X_H$ double ratios are sensitive to right-handed currents.
In fact the $R_H$ ratios can be expressed in terms of their deviation from unity as
\begin{equation}
\begin{array}{rl}
R_K \simeq 			& 1+ \Delta_+, 		\\
R_{K_0(1430)} \simeq 	& 1+ \Delta_-,		\\
R_{\Kstarz} \simeq 		& 1+ p(\Delta_- - \Delta_+) + \Delta_+,
\end{array}
\end{equation}
where the $\Delta_\pm$ quantities are combinations of Wilson coefficients
described in Eq.~10 of Ref.~\cite{Hiller:2014ula} and the parameter $p$ is the polarisation of \Kstarz
that in Ref.~\cite{Hiller:2014ula} is determined to be close to 1, simplifying the formula to $R_{\Kstarz} \simeq 1+ \Delta_-$.
In particular one can make the following observations: 
\begin{itemize}
\item $R_K < 1$, as it is measured to be, and $X_{\Kstarz} > 1$ points to dominant BSM contributions into $C_{LR}$ (see definition in Sec.~\ref{sec:operators});
\item a SM-like $R_K \sim 1$ together with $X_{\Kstarz} \neq 1$ requires BSM with $C_{LL} + C_{RL} \simeq 0$;
\item $R_K \neq 1$ and $X_{\Kstarz} \simeq 1$ corresponds to new physics in $C_{LL}$.
\end{itemize}

\section{Experimental status}

The $R_K$ and \RKst ratios have been measured at the B-factories~\cite{Lees:2012tva,Wei:2009zv},
while the recent measurement from LHCb~\cite{Aaij:2014ora} represents the most precise determination 
of $R_K$ to date; measured values are summarised in Tab.~\ref{tab:expstatus}.
The LHCb measurement manifests a $2.6~\sigma$ deviation from the SM prediction. This is particularly interesting as this
discrepancy can be explained with a new physics contribution in $C_9$ which also explains other existing tensions~\cite{Altmannshofer:2014rta,Descotes-Genon:2013wba,Hurth:2016fbr}.
It is also worth mentioning the measurement of the 
\mbox{$\mathcal{B}(\bar{B}^0 \to D^{*+}\tau^{-}\bar{\nu}_{\tau})/\mathcal{B}(\bar{B}^0 \to D^{*+}\mu^{-}\bar{\nu}_{\mu})$} 
ratio, which also probes LFU and was measured to be $2.1~\sigma$ larger than the value expected from the assumption 
of lepton universality in the SM~\cite{Aaij:2015yra}.
By profiting from the large dataset collected during Run-I, the LHCb experiment is expected 
to reduce the uncertainty on \RKst by at least a factor of 2 with respect to the B-factories.
%
\begin{table}[b]
\renewcommand\arraystretch{1.5}
\centering
\caption{Experimental status of the $R_{K^{(*)}}$ measurements. } %by the BaBar~\cite{Lees:2012tva} and Belle~\cite{Wei:2009zv} experiment.}
\begin{tabular}{$c|^c^c^c}
\rowstyle{\bfseries}
 Ratio	& Belle 			& BaBar 		& LHCb \\
 \hline
$R_K$			& $1.06 \pm 0.48 \pm 0.05$	& $1.38^{+0.39+0.06}_{-0.41-0.07}$ & $0.745^{+0.090}_{-0.074} \pm 0.036$\\
$R_{\Kstarz}$	& $0.93 \pm 0.46 \pm 0.12$	& $0.98^{+0.30+0.08}_{-0.31-0.08}$ & ---\\
\end{tabular}
\label{tab:expstatus}
\end{table}
\clearpage

\section{Analysis strategy}

The aim of the analysis in this chapter is to measure the $R_{\Kstarz}$ ratio using $pp$ collision data
collected by the LHCb detector in 2011 and 2012, corresponding to 3 \invfb of integrated luminosity.
The $\Bz \to \Kstarz\mumu$ and $\Bz \to \Kstarz\epem$, ``rare channels", are
reconstructed via the $\Kstarz$ decay into a kaon and a pion with opposite charges.

The analysis has to separate signal candidates from background candidates which have similar observed properties. 
The selection presented in Sec.~\ref{sec:RKst_selection} aims to maximise the yield while minimising
the background contamination. Two types of backgrounds are identified: ``peaking background" and ``combinatorial background". 
The first comes from misreconstructed or partially reconstructed decays. Due to its specific kinematic properties, this type 
of background usually peaks in some variable such as the invariant
mass of all final particles and, therefore, these candidates can be removed using specific cuts. 
In contrast, the combinatorial background arises from the random combination of particles and can 
be reduced by selecting candidates with good-quality tracks and vertices.

To further reduce the systematic uncertainties the measurement is performed as the double ratio 
%
\begin{equation}
\label{eq:RKst}
R_{\Kstarz} = 
\frac{N_{\Bz\to\Kstarz \mumu}}{N_{\Bz\to\Kstarz\jpsi\to \mumu}} 
\cdot \frac{N_{\Bz\to\Kstarz \jpsi \to \ee}}{N_{\Bz\to\Kstarz\ee}}
\cdot \frac{\varepsilon_{\Bz\to\Kstarz \jpsi \to \mumu}}{\varepsilon_{\Bz\to\Kstarz \mumu}} 
\cdot \frac{\varepsilon_{\Bz\to\Kstarz \ee}}{\varepsilon_{\Bz\to\Kstarz\jpsi\to \ee}},
\end{equation}
%
where decays reaching the same final states as the rare channels via a \jpsi resonance, $\Bz\to\Kstarz(\jpsi\to\ll)$,
also referred to as ``charmonium" or ``resonant" channels, are used as control samples.
These decays are distinguished from the rare channel using the invariant mass of the dilepton pair.
%
%In Sec.~\ref{sec:RKst_efficiency} the efficiency of selecting and reconstructing each channel is extracted
%and, finally, in Sec.~\ref{sec:RKst_result} the $R_{\Kstar}$ ratio defined is built as the double ratio
%of rare and resonant channels:
%
As new physics is not expected to affect tree level $b\to\cquark\cquarkbar\squark$ processes, the ratio 
between the \jpsi channels, $r_{\jpsi}$, \mbox{is 1} and therefore \mbox{$\RKst^{'} = \RKst \cdot r_{\jpsi} = \RKst$}.
On the other hand, using the relative efficiencies between the rare and resonant channels
causes many systematic effects to cancel resulting in a better control of systematic uncertainties.  

For brevity, the rare channels will also be denoted as ``$\ell\ell$", or
specifically ``$ee$'' and ``$\mu\mu$'', and the resonant channels as ``$\jpsi(\ell\ell)$'',
or ``$\jpsi(ee)$'' and ``$\jpsi(\mu\mu)$''.

\section{Dilepton invariant mass intervals}
\label{sec:RKst_q2_choice}

Three \qsq intervals are considered in this work: 
\begin{itemize}
\item the ``low-$q^2$" region, [0.0004,1.1]~\gevgevcccc, where the $\bquark\to\squark\ll$ process is dominated by the photon pole;
\item the ``central-$q^2$" region, [1.1,6.0]~\gevgevcccc;
\item the ``high-$q^2$"region, above 15~\gevgevcccc.
\end{itemize}
%
The central-\qsq region is the most interesting place to look for new physics. In fact, at low \qsq values, below 
1~\gevgevcccc~the photon pole dominates leaving little space for new physics to be found. %~\ref{sec:theo_qsq}.
The choice of the lower limit of the low-\qsq interval is driven by the need to reject the background due to the 
$\Bz\to\Kstarz\gamma$ decay where the photon converts into electrons in the material of the detector.
The lower bound of the central interval is set at 1.1~\gevgevcccc, to exclude a possible contribution from $\phi\to\ll$ decays, 
which can dilute new physics effects, while the upper bound is chosen to be sufficiently far away from the \jpsi radiative
tail where predictions are less cleanly defined. The 6 -- 15~\gevgevcccc~region is characterised by the presence
of the narrow peaks of the \jpsi and \psitwos resonances. The lower bound of the high-\qsq region, where
the signal in the electron channel is still unobserved, is chosen to be sufficiently far from the \psitwos resonance.
Rare and normalisation channels are selected according to the \qsq interval they fall into (for details see Sec.~\ref{sec:RKst_selection}).

\subsection{Control channels}
Beyond the normalisation channels, $\jpsi(ee)$ and $\jpsi(\mu\mu)$, additional control channels
are used to perform cross-checks and better constrain some of the background components
in the electron fit; in particular, \BdToKstGee, also denoted as ``$\gamma (ee)$", where the photon 
converts into an \ee pair in the detector material and \BdToKstPsiee, also denoted as ``$\psitwos(ee)$".
All of the normalisation and control channels are distinguished by the \qsq interval
they fall into. % (for details see Sec.~\ref{sec:RKst_selection}).

\section{Data samples and simulation}

%The analysis in this chapter is based on a dataset corresponding to 3~\invfb of integrated
%luminosity collected by the LHCb detector in 2011 and 2012.
Simulated samples are used to study the properties of backgrounds, determine efficiencies and to train
a multivariate classifier. The hard interactions are generated with \textsc{Pythia8}, hadronic particles
are decayed using \textsc{EvtGen} and, finally, propagated into the detector using \textsc{Geant4} and reconstructed
with the same software used for data. Samples are generated with both 2011 and 2012, magnet up and down
conditions and are combined in the appropriate proportions, according to the data integrated luminosities.
The next section describes the corrections applied to the simulation to ensure that it provides a good description of data.

%distributions of key valiables were compared between data and simulation and correction 
%In Table \ref{TabMC} are reported the MC samples used
%together with the number of generated events and the branching fractions of the simulated processes

\subsection{Data-simulation corrections}
\label{sec:RKst_mc_weighting}

Since the multivariate classifier training (see Sec.~\ref{sec:RKst_mva}) and the calculation
of most of the efficiency components (see Sec.~\ref{sec:RKst_efficiency}) are obtained from
the study of simulated events it is important to verify that the simulation provides a reliable 
description of data. Two areas where this agreement is particularly important are 
the kinematics of the final particles and the occupancy of the detector.
The kinematics of the decays is characterised by the transverse momentum spectrum of
the \Bz. Discrepancies in this distribution also cause the spectra of the final particles
to differ from data and hence affect the efficiency determination as its value often
depends on the momentum of the final particles.
The occupancy of the detector is relevant as it is correlated to the invariant mass shape of the signal
due to the addition of energy clusters in the electromagnetic calorimeter,
which affects the momenta of the electrons especially when bremsstrahlung photons are emitted before the magnet.
The hit multiplicity in the SPD detector is used as a proxy for the detector occupancy.

Since it is important that these quantities are well modelled, the simulation is
reweighted so that their distributions in data and simulation match.
The weight is calculated using resonant $\decay{\Bz}{\Kstarz(\jpsi\to\ll)}$ candidates, for which the signal peak
is already visible in data after pre-selection (see Sec.~\ref{sec:RKst_selection}). However, the data still includes
a high level of background and distributions cannot be directly compared.
The $_s\mathcal{P}$lot technique~\cite{sPlot} is used to statistically subtract the background from
data and obtain pure signal distributions using the invariant mass as the control variable.
%This method is based on an estimation
%od the signal and background densities based on a fit to a control variable where
%the two are well distinct, usually the invariant mass.
Figure~\ref{fig:RKst_sW_mass} shows fits to the 4-body invariant mass of candidates after pre-selection.
Data and simulation are then compared and the ratio between the two distributions is used to reweight
the simulation. The discrepancy in the SPD multiplicity is solved as a first step and then the \Bz transverse momentum 
distributions are compared in data and simulation reweighted to account for the SPD multiplicity.

Distributions of \Bz transverse momentum and SPD multiplicity are reported in Fig.~\ref{fig:b0pt_nSPD_distrib}
and ratios of these distributions, which are used to reweight the simulation, are reported in 
Fig.~\ref{fig:b0pt_nSPD_ratios}. The weights for the SPD multiplicity are calculated
separately for 2011 and 2012 events, because distributions are significantly different
in the two years. The binnings are chosen to have approximately 
the same number of events in each bin to limit fluctuations.
Further corrections are made by reweighting the simulation for PID efficiency using the
\verb!PIDCalib! package as described in Sec.~\ref{sec:RKst_pid_eff} and, finally, 
$ee$ samples are also reweighted for L0 trigger efficiency as described in Sec.~\ref{sec:RKst_trigger_eff}.
Weights are always applied throughout unless specified.
%Finally in Fig.~\ref{fig:mc_data_comparison} distributions of other variables are compared
%between data and reweighted Monte Carlo and show that a good agreement is achieved.

 \begin{figure}[h!]
\centering
%\includegraphics[width=0.48\textwidth]{RKst/figs/sW/KstJPsEE_log_fitAndRes.pdf}
\includegraphics[width=0.72\textwidth]{RKst/figs/sW/KstJPsMM_log.pdf}
\caption{Fitted 4-body invariant mass distribution of $\jpsi(\mu\mu)$ candidates 
after pre-selection used to obtain $_s\mathcal{W}$eights.}
\label{fig:RKst_sW_mass}
\end{figure}

\begin{figure}[h!]
\centering
\includegraphics[width=0.49\textwidth]{RKst/figs/nspd_12.pdf}
\includegraphics[width=0.49\textwidth]{RKst/figs/bpt.pdf}
\caption{Distributions of number of SPD hits (left) and \Bz transverse momentum (right) in data and simulation.}
\label{fig:b0pt_nSPD_distrib}
\end{figure}

\begin{figure}[h!]
\centering
\includegraphics[width=0.49\textwidth]{RKst/figs/nspd_w.pdf}
\includegraphics[width=0.49\textwidth]{RKst/figs/bpt_w.pdf}
\caption{ Ratios of simulated over real data distributions used to correct the simulation
as a function of the number of SPD hits (left) and the \Bz transverse momentum (right). }
\label{fig:b0pt_nSPD_ratios}
\end{figure}



