\section{Result extraction}
\label{sec:RKst_result}

This section presents the final results of this analysis
together with the description of sanity checks performed to verify the stability
of the methods used.

\subsection{$R_{\jpsi}$ sanity check}
\label{sec:Rjpsi}

In order to cross-check the analysis procedure, the ratio between the
measured branching ratio of the electron and muon resonant channels is calculated:
%
\begin{equation}
R_{\jpsi} = \frac{\BR(\Bz\to\Kstarz\jpsi(\to \mumu))} {\BR(\Bz\to\Kstarz\jpsi(\to \epem))} 
= \frac{\varepsilon_{\jpsi(\mu\mu)} \cdot N_{\Bz\to\Kstarz\jpsi(\to \epem)}}{\varepsilon_{\jpsi(ee)} 
\cdot N_{\Bz\to\Kstarz\jpsi(\to \mumu)}}.
\end{equation}
%
Unlike absolute branching fractions calculations, the determination of $R_{\jpsi}$ represents a better
sanity test as it is not affected by uncertainties due to the knowledge of the amount of collected 
luminosity, $\mathcal{L}$, or the fragmentation fraction, $f_d$, the probability for a $\bquark$
quark to produce a \Bz meson. These quantities come with large uncertainties but they cancel
in the $R_{\jpsi}$ ratio.

%but this ratio
%is a better defined sanity check for our analysis. In fact the absolute branching fractions
%can be calculated from the raw yields as
%
%\begin{equation}
%\BR = \mathcal{L} \cdot \sigma_{\bquark\bquarkbar} \cdot f_d \cdot \varepsilon \cdot N_{raw}
%\end{equation}
%
%where $\mathcal{L}$ is the luminosity, $\sigma_{\bquark\bquarkbar}$ is the cross
%subsection for $\bquark\bquarkbar$ production and $f_d$ is the fragmentation fraction,
%the probability for a $\bquark$ quark to produce a \Bz meson.


Measured values of the $R_{\jpsi}$ ratio are reported in Tab.~\ref{tab:Rjpsi}, where the error 
shown is statistical only. For this purpose the trigger efficiencies are corrected using the 
factors obtained in Sec.~\ref{sec:tistos}. Note that systematic uncertainties, 
which cancel when doing the ratio between the rare and
resonant channels with same leptonic final state, do not cancel in this case.
A reasonable agreement with unity is found.

\begin{table}[h!]
\centering
 \caption{Fully corrected measured values of the ratio
 $R_{\jpsi}$ in the three electron trigger categories. }
\begin{tabular}{|c|c|}
\hline
Trigger & $R_{\jpsi}$ \\
\hline
L0E  	   & $ 1.028  \pm 0.022 $ \\
L0H     & $ 0.986  \pm 0.072 $ \\
L0I      & $ 0.973  \pm 0.128 $ \\
\hline
  \end{tabular}
 \label{tab:Rjpsi}
\end{table}

\subsection{\RKst result summary}

The ratio \RKst is extracted by dividing the $R_{ee}$ and $R_{\mu\mu}$
parameters described in Sec.~\ref{sec:rkst_fits}. These ratios are 
direct parameters of the fit but they can also be built from the yields
in Tab.~\ref{tab:RKst_yields} and the efficiencies in Tab.~\ref{tab:double_rel_eff}.
In summary the definition of the \RKst ratio is the following:
%
\begin{equation}
\RKst = \frac{R_{ee}}{R_{\mu\mu}}  
%= \frac{N_{\Bz\to\Kstar ee}}{N_{\Bz\to\Kstar\jpsi\to ee}} 
%\cdot \frac{N_{\Bz\to\Kstar \jpsi \mumu}}{N_{\Bz\to\Kstar\mumu}}
%\cdot \frac{\varepsilon_{\Bz\to\Kstar \jpsi \to ee}}{\varepsilon_{\Bz\to\Kstar ee}} 
%\cdot \frac{\varepsilon_{\Bz\to\Kstar \mumu}}{\varepsilon_{\Bz\to\Kstar\jpsi\to \mumu}}
= \frac{N_{ee}}{N_{\jpsi(ee)}} 
\cdot \frac{N_{\jpsi(\mu\mu)}}{N_{\mu\mu}}
\cdot \frac{\varepsilon_{\jpsi(ee)}}{\varepsilon_{ee}} 
\cdot \frac{\varepsilon_{\mu\mu}}{\varepsilon_{\jpsi(\mu\mu)}}.
\end{equation}
%
As the electron ratio $R_{ee}$ is a shared parameter in the simultaneous fit to the
three electron categories its value is already a combination of the three samples.
Results are shown in Tab.~\ref{tab:RKst_results}.

\begin{table}
\centering
 \caption{Measured values of $R_{ee}$, $R_{\mu\mu}$ and \RKst ratios.}
\begin{tabular}{|c|c|c|}
\hline
 Ratio 			& 1--6 GeV$^2/c^4$ & 15--20 GeV$^2/c^4$\\ \hline
$R_{ee}$ 	& $ 0.00305  \pm 0.00040  $ 	& $ 0.00406  \pm 0.00081 $ \\
$R_{\mu\mu}$ 	& $ 0.00242  \pm 0.00011 $ 	& $ 0.00225  \pm 0.00010 $ \\
\hline $R_{\Kstarz}$ 	& blind 	& blind  \\
\hline 
 \end{tabular}
 \label{tab:RKst_results}
\end{table}

\subsection{Branching ratios and expectations}

Multiplying the ratios $R_{ee}$ and $R_{\mu\mu}$ by the measured $\Bz\to\Kstar (\jpsi\to\ell^+\ell^-)$~\cite{PDG2014}
branching ratios one can obtain absolute branching ratios for the rare channes:
%
\begin{equation}
\begin{split}
&\BR(\decay{\Bz}{\Kstarz(\jpsi\to\ell^+\ell^-)}) = \BR(\decay{\Bz}{\Kstarz\jpsi}) \times \BR(\decay{\jpsi}{\ll}) \\
& = (1.32 \pm 0.06) 10^{-3} \times (5.96 \pm 0.03) 10^{-2} = (7.87 \pm 0.36) \times 10^{-5}
\end{split}
\end{equation}
%
Table~\ref{tab:RKst_abs_BR} reports absolute branching ratio values obtained for the rare channels
in the considered \qsq intervals, where the errors are statistical only.
%
The results for the central-\qsq interval can be compared also with SM predictions obtained from Ref.~\cite{Ali:2002jg}.
This paper reports predicted branching ratios in the \mbox{$1 < \qsq < 6$~\gevgevcccc} interval
for the electron and muon rare channels. These are rescaled to the range $1.1 < \qsq < 6$ \gevgevcccc
using simulation. Finally, using the measured value of the measured \decay{\Bz}{\Kstarz(\jpsi\to\ell^+\ell^-)}
decay, the predicted ratio is found to be $0.75 \pm 0.14$,
which is in agreement with our measurement within one standard deviation.
Table~\ref{tab:RKst_expectations} also lists observed and expected ratios of rare
over resonant raw numbers of candidates ($N_{\ell\ell} / N_{\jpsi}$). In this table the observed ratios
are simply obtained dividing the rare and resonant yields in Tab.~\ref{tab:RKst_yields}, while
the expected ones are obtained by dividing the predicted rare channel branching ratios by 
the measured \decay{\Bz}{\Kstar(\jpsi\to\ll)} branching ratios, rescaled by the
relative efficiencies given in Tab.~\ref{tab:RKst_releff}.  

\begin{table}[h]
\centering
\caption{Measured absolute branching ratio of the rare $\mu\mu$ and $ee$ channels in
the central and high \qsq regions. Errors shown are statistical only. }
\begin{tabular}{|c|c|c|}
\hline
 Channel 		& 1--6 GeV$^2/c^4$ & 15--20 GeV$^2/c^4$\\ \hline
$ee$ 		& $( 1.80  \pm  0.24 )\times 10^{-7}$ 	& $( 3.19  \pm  0.64 )\times 10^{-7}$ \\
 $\mu\mu$ 	& $( 2.07  \pm  0.10 )\times 10^{-7}$ 	& $( 1.92  \pm  0.09 )\times 10^{-7}$ \\
\hline 
 \end{tabular}

\label{tab:RKst_abs_BR}
\end{table}
%
\begin{table}[h]
\centering
 \caption{Expected and observed ratios of raw event yields, $N_{\ell\ell} / N_{J/\psi}$. }
\begin{tabular}{|c|c|c|c|}
\hline
 Sample 			& Expected 			& Observed 			& Obs / exp ratio \\ \hline
$\mu\mu$ 	& $ 0.00253  \pm  0.00084 $ 	& $ 0.00188  \pm  0.00009 $ 	& $ 0.74309  \pm  0.24866 $ \\
\hline
$ee$ (L0E) 	& $ 0.00269  \pm  0.00084 $ 	& $ 0.00271  \pm  0.00035 $ 	&  \\
$ee$ (L0H) 	& $ 0.00723  \pm  0.00227 $ 	& $ 0.00732  \pm  0.00098 $ 	& $ 1.00826  \pm  0.34265 $ \\
$ee$ (L0I) 	& $ 0.00383  \pm  0.00120 $ 	& $ 0.00388  \pm  0.00051 $ 	&  \\
\hline 
 \end{tabular}
 \label{tab:RKst_expectations}
\end{table}

\clearpage
